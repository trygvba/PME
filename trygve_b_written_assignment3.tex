\documentclass[11pt, a4paper]{article}
\usepackage{mathptmx}
\usepackage{mathtools}
\usepackage{amssymb}
%\usepackage{amsfonts}
%\usepackage{amsmath}
\usepackage{fullpage}
\usepackage{enumerate}
\usepackage{amsthm}
\usepackage{remreset}
\usepackage[utf8]{inputenc}
\usepackage[numbers]{natbib}
\usepackage{lmodern}
\usepackage{setspace}


\begin{document}
\onehalfspacing

\title{Written assignment 3 in SPRÅK3501}
\author{Trygve Bærland}
\maketitle


\section*{Abstract for my project.}
This project is an exposition of the General Porous Medium Equation, which contains both the porous medium equation and the heat equation as special cases as well as the Stefan problem when considering the degenerate case. The first part is devoted to presenting some theory on classical solution with attention paid to important estimates. The comparison principle and $L^1$-contractivity, as well as estimates on both the spatial and temporal derivatives are derived. Concluding the discussion of classical solutions is an existence result under quite strict assumptions on the initial data, loading and nonlinearity.

 The theory is then extended to a weak setting, and a definition of weak solutions is provided. Again, proving well posedness will be the main focus, and to that end the concept of weak energy solutions is introduced. In this framework existence, uniqueness and stability is proved, and attention is paid to showing that the estimates for classical solutions are inherited by weak solutions. This is first done for the nondegenerate case, and the project report concludes by extending the well posedness results to the degenerate case. 

%%%%%%%%%%%%%%%%%%%%%%%%%%%%%%
%	BIBLIOGRAPHY
%%%%%%%%%%%%%%%%%%%%%%%%%%%%%%

\newpage
\bibliographystyle{plain}
\bibliography{project_bibliography.bib}
\end{document}