\documentclass[11pt, a4paper]{article}
\usepackage{mathptmx}
\usepackage{mathtools}
\usepackage{amssymb}
%\usepackage{amsfonts}
%\usepackage{amsmath}
\usepackage{fullpage}
\usepackage{enumerate}
\usepackage{amsthm}
\usepackage{remreset}
\usepackage[utf8]{inputenc}

\usepackage{lmodern}

\begin{document}
\title{Stuff that may well go into the project report}
\author{Trygve Bærland}
\maketitle
\abstract{Funker dette?}
\section{Basics}
\newtheorem{theorem}{Theorem}
\newtheorem{mydef}{Definition}
\newtheorem{ex}{Example}

Let $\Omega \subset \mathbb{R}^d$ be an open, simply connected domain. Then the filtration-, or general porous medium equation is
\begin{equation}
\label{GPME}
	\partial_t u = \Delta \Phi(u) + f,
\end{equation}
 where $u, f: \Omega \times (0,\infty) \to \mathbb{R}$, and $\phi: \mathbb{R} \to \mathbb{R}$. More specific assumptions on the nonlinearity $\Phi$  and the forcing term $f$ will follow.

\subsection{Problem statement}
I will consider on the following Dirichlet problem:
\begin{equation}
\label{HDP}
	\begin{cases}
		\partial_t u = \Delta \Phi(u) + f, \quad \text{ in } Q_T \\
		u(x,0) = u_0(x), \quad \text{ in } \Omega \\
		u(x,t) = 0, \quad \text{ on } \partial\Omega \times (0,T).
	\end{cases}
\end{equation}
Here $Q_T = \Omega \times (0,T)$, where $T$ may very well be infinite.
\subsection{Weak formulation}
\begin{mydef}[Weak solutions]
A locally integrable function $u$ defined in $Q_T$ is said to be a \textbf{weak solution} of \ref{HDP} if
\begin{enumerate}[i)]
	\item $u \in L^1(Q_T)$ and $\Phi(u) \in L^1(0,T : W_0^{1,1}(\Omega))$\footnote{$f:\Omega \to \mathbb{R}$ is an element of $W^{k,p}(\Omega)$ if all weak derivatives with $|\alpha| \leq k$ exists and are in $L^p(\Omega)$. $W_0^{k,p}$ is the closure of $C^\infty_c(\Omega)$ in $W^{k,p}(\Omega)$.};
	\item $u$ satisfies
		\begin{equation}
		\label{weak}
		\iint_{Q_T} \{\nabla\Phi(u)\cdot \nabla\eta - u\partial_t \eta \}dxdt = \int_\Omega u_0(x)\eta(x,0)dx + \iint_{Q_T}f\eta dxdt
		\end{equation}
		for any $\eta \in C^1(\overline{Q}_T)$ that vanishes on the boundary and has compact support.
\end{enumerate}
\end{mydef}
%%%%%%%%%%%%%%%%%%%%%%%%%%%%%%%%%%%%%%%%%%%%%%%%%%%%%%%%%%%%%%%
%					ESTIMATES
%%%%%%%%%%%%%%%%%%%%%%%%%%%%%%%%%%%%%%%%%%%%%%%%%%%%%%%%%%%%%%%
\section{Some estimates}
Now we need to make some assumptions on $\Phi$ and $f$:
\begin{enumerate}[i)]
	\item The function $\Phi: \mathbb{R} \to \mathbb{R}$ is continuous, strictly increasing and we assume $\Phi(0) = 0$ without any loss of generality.
\end{enumerate}
For now the assumption that $\Phi'(u) > 0$ is kind of strong, but I hope to generalise (or find analogues for) most of the estimates to the case $\Phi'(u) \geq 0$. In that case (\ref{GPME}) changes from nodegenerate parabolic to degenerate parabolic. But hopes are high.
%%%%%%%%%%%%%%%%%%%%%%%%%%%%%%%%%%%%%%%%%%%%%%%%%%%%%%%%%%%%%%%%
%					UNIQUENESS
%%%%%%%%%%%%%%%%%%%%%%%%%%%%%%%%%%%%%%%%%%%%%%%%%%%%%%%%%%%%%%%%
In the following proof of uniqueness of weak solutions I'll try out assuming that $\Phi'(u) \geq 0$.
\section{Uniqueness}
\begin{theorem}[Uniqueness]
	Assuming $u \in L^2(Q_T)$ and that $\Phi \in L^2(0,T : H_0^1(\Omega))$, (\ref{HDP}) has at most one weak solution.
\end{theorem}
\begin{proof}
This is a pretty neat proof using a smart choice of $\eta$. So let's assume $u_1$ and $u_2$ are weak solutions. Then
\begin{equation*}
\iint_{Q_T}\nabla(\Phi(u_2) - \Phi(u_1)) \cdot \nabla \eta - (u_2 - u_1)\partial_t \eta dxdt = 0.
\end{equation*}
Now the inspired move\footnote{Of course, this isn't my proof so that sort of horn-tooting is in good taste.} is choosing
\begin{equation*}
\eta(x,t) = \begin{cases}
		\int_t^T(\Phi(u_2) - \Phi(u_1))ds, \quad &\text{ if } 0 < t < T \\
		0 \quad &\text{ if } t\geq T.
	\end{cases}
\end{equation*}
It is then easily verified that
\begin{equation*}
\begin{cases}
	\partial_t \eta &= -(\Phi(u_2) - \Phi(u_1) ) \\
	\nabla \eta &= \int_t^T\nabla (\Phi(u_2) - \Phi(u_1))ds,
\end{cases}
\end{equation*}
which put into the integral equality above yields
\begin{equation}
\label{unique}
\begin{aligned}
	&\iint_{Q_T}(\Phi(u_2) - \Phi(u_1))(u_2 - u_1)dxdt \\ 
	&+ \iint_{Q_T}\nabla(\Phi(u_2(t)) - \Phi(u_1(t))) \cdot \left(\int_t^T 					\nabla(\Phi(u_2(s)) - \Phi(u_1(s))ds\right)dtdx = 0.
\end{aligned}
\end{equation}
Let's take these in order:
\begin{enumerate}[i)]
	\item In the first integral we notice that $u_2 > u_1 \Rightarrow \Phi(u_2) \geq \Phi(u_1)$ and vice versa, so this integral is nonnegative.
	
	\item For the second integral it is helpful to first only consider integration of the temporal dimensions, leaving us with
	\begin{align*}
	&\int_0^T\int_t^T \nabla(\Phi(u_2(t)) - \Phi(u_1(t)))\cdot \nabla(\Phi(u_2(s))-\Phi(u_1(s)))dsdt \\
	&= \int_0^T\int_0^s\nabla(\Phi(u_2(t)) - \Phi(u_1(t)))\cdot \nabla(\Phi(u_2(s))-\Phi(u_1(s)))dtds.
	\end{align*}
	Due to the symmetry about the line $s=t$, we have
	\begin{align*}
	&\int_0^T\int_0^s\nabla(\Phi(u_2(t)) - \Phi(u_1(t)))\cdot \nabla(\Phi(u_2(s))-\Phi(u_1(s)))dtds \\
	&= \frac{1}{2}\int_0^T \int_0^T \nabla(\Phi(u_2(t)) - \Phi(u_1(t)))\cdot \nabla(\Phi(u_2(s))-\Phi(u_1(s)))dtds \\
	&= \frac{1}{2}\left( \int_0^T \nabla(\Phi(u_2)-\Phi(u_1))dt\right)^2,
	\end{align*}
	which is also nonnegative, and therefore the second integral in (\ref{unique}) must also be nonnegative.
\end{enumerate}
From this we may conclude that both integral terms in (\ref{unique}) are zero, which in turn implies that $(\Phi(u_2)-\Phi(u_1))(u_2 - u_1) = 0$ a.e. in $Q_T$. Wherever $\Phi'(u) > 0$ we have $u_2 = u_1$, but more care is needed when $u_1$ and $u_2$ take values where $\Phi'(u) = 0$. So suppose $u_1 \neq u_2$ on a set $E \subset Q_T$ of positive measure. That means $\nabla(\Phi(u_2) - \Phi(u_1)) = 0$ a.e. in $E$, and so we can deduce that 
\begin{equation*}
\iint_{Q_T}(u_2 - u_1)\partial_t \eta dxdt = 0
\end{equation*}
for all test functions $\eta$ with support in $E$. This gives us that $u_2 = u_1$ a.e. in $E$ as well, so $u_2 = u_1$. Whoop-dee-doo!
\end{proof}
\end{document}



