\documentclass[11pt, a4paper]{article}
\usepackage{mathptmx}
\usepackage{mathtools}
\usepackage{amssymb}
%\usepackage{amsfonts}
\usepackage{amsmath}
\usepackage{fullpage}
\usepackage{enumerate}
\usepackage{amsthm}
\usepackage{remreset}
\usepackage[utf8x]{inputenc}
\usepackage[toc,page]{appendix}
\usepackage[numbers]{natbib}

\usepackage{lmodern}

\begin{document}
\title{Draft for project}
\author{Trygve Bærland}
\maketitle
\abstract{For now, heavily reliant on \citep{vazquez2007porous}.}

% My text categories:
\theoremstyle{plain}
\newtheorem{theorem}{Theorem}[section]
\newtheorem{proposition}{Proposition}[section]
\newtheorem{cor}{Corollary}[section]
\newtheorem{lemma}{Lemma}[section]

\theoremstyle{definition}
\newtheorem{mydef}{Definition}[section]
\newtheorem{example}{Example}[section]

%\theoremstyle{plain}
\newtheorem{obs}{Observation}
\newtheorem{rem}{Remark}


% Reset numbering for each section
\numberwithin{equation}{section}
%%%%%%%%%%%%%%%%%%%%%%%%%%%%%%%%%%%%%%%%%%%%%%%%%%%%%%%
%				INTRODUCTION
%%%%%%%%%%%%%%%%%%%%%%%%%%%%%%%%%%%%%%%%%%%%%%%%%%%%%%%
\section{Introduction}

Let $\Omega \subset \mathbb{R}^d$ be an open, bounded, and simply connected domain. Then the filtration, or general porous medium equation is

\begin{equation}
\label{GPME}
	\partial_t u = \Delta \Phi(u) + f,
\end{equation}

 where $u, f: \Omega \times (0,\infty) \to \mathbb{R}$, and $\phi: \mathbb{R} \to \mathbb{R}$. For now we assume that $\Phi$ is continuous and nondecreasing.


\subsection{Problem statement}
I will consider on the following Dirichlet problem:
\begin{equation}
\label{HDP}
	\begin{cases}
		\partial_t u = \Delta \Phi(u) + f, \quad \text{ in } Q_T \\
		u(x,0) = u_0(x), \quad \text{ in } \Omega \\
		u(x,t) = 0, \quad \text{ on } \partial\Omega \times [0,T].
	\end{cases}
\end{equation}
Here $Q_T = \Omega \times (0,T]$, where $T$ may very well be infinite. Furthermore we denote by $\Gamma_T = \overline{Q}_T\backslash Q_T$ the parabolic boundary.
%%%%%%%%%%%%%%%%%%%%%%%%%%%%%%%%%%%%%%%%%%%%%%%%%%%%%%%
%				CLASSICAL SOLUTIONS
%%%%%%%%%%%%%%%%%%%%%%%%%%%%%%%%%%%%%%%%%%%%%%%%%%%%%%%
\section{Classical solutions}
In this section it is helpful to reformulate (\ref{HDP}) as 
\begin{equation}
	\begin{cases}
		\partial_t u + Lu &= f, \quad \text{in } Q_T \\
		u(x,0) &= u_0(x), \quad \text{in } \Omega \\
		u &= 0, \quad \text{on } \partial \Omega \times [0,T]
	\end{cases}
\end{equation}
where $L$ is a differential operator that can be written in the divergence form
\begin{equation}
Lu = -\sum_{i=1}^d \partial_{x_i}(a^i(x,t,u,\nabla u)) + a(x,t,u,\nabla u).
\end{equation}

The reason for this reformulation is that many of the results presented in this section supposes a particular condition on the coefficients $a^i(x,t,u,p)$. The condition is that there are constants $0 < \nu < \mu < \infty $ so that
\begin{equation}
\label{uniform_parabolicity}
\nu |\xi|^2 \leq \sum_{i,j=1}^d \frac{\partial a^i(x,t,u,p)}{\partial p_j}\xi_i \xi_j \leq \mu |\xi|^2 
\end{equation}
for every $\xi \in \mathbb{R}^d$. If this condition is satisfied we say that the operator $\partial_t + L$ is \textbf{uniformly parabolic}.

In our setting it's easy enough to check that
\begin{equation*}
a^i(x,t,u,p) = \Phi'(u)p_i,
\end{equation*}
and so (\ref{uniform_parabolicity}) translates to
\begin{equation}
\label{uniform_parabGPME}
\nu \leq \Phi'(u) \leq \mu.
\end{equation}
We will in this section assume this to be true. I.e. that $\Phi'(u)$ is bounded and strictly positive.
%%%%%%%%%%%%%%%%%%%%%%%%%%%%%%%%%%%%%%%%%%%%%%%%%%%%%
%			MAXIMUM PRINCIPLE:
%%%%%%%%%%%%%%%%%%%%%%%%%%%%%%%%%%%%%%%%%%%%%%%%%%%%%
\subsection{Maximum principle}
The first main estimate we'll deal with is a $L^\infty$ bound on a classical solution. To that end, the following strong maximum principle will be helpful (cf. \citep[p. 396]{evans}).

\begin{theorem}[Strong maximum principle]
\begin{enumerate}[i)]
	Suppose $\Omega$ is connected, and $u\in C^{2,1}(Q_T)\cap C(\overline{Q}_T)$, then:
	\item If
	\begin{equation*}
		\partial_tu + Lu \leq 0, \quad \text{ in } Q_T, 
	\end{equation*}
	and $u$ attains its maximum over $\overline{Q}_T$ at point $(x_0,t_0)\in Q_T$, then
	$u$ is constant on $Q_{t_0}$.
	
	\item Analagously, if
	\begin{equation*}
		\partial_tu + Lu \geq 0, \quad \text{ in } Q_T, 
	\end{equation*}
	and $u$ attains its minimum over $\overline{Q}_T$ at a points  $(x_0,t_0) \in Q_T$, then $u$ is constant on $Q_{t_0}$.
\end{enumerate}
\end{theorem}

Using this theorem we can prove the comparison principle:
\begin{lemma}[Comparison principle]
\label{lemma:comparison}
Let $u,v \in C^{2,1}(Q_T)\cap C(\overline{Q}_T)$ be two solutions of (\ref{HDP}) with initial data $u_0$, $v_0$ and loading $f$, $\tilde{f}$ respectively. Further, assume that $v_0 \geq u_0$ in $\Omega$ and  $\tilde{f} \geq f$ in $Q_T$. Then either $u < v$ in $Q_T$ or $u=v$.
\end{lemma}
\begin{proof}
We set $w=v-u$ and see that $w$ then solves
\begin{align*}
\partial_t w &= \Delta(\Phi(v) - \Phi(u)) + (\tilde{f}-f) \\
			&= \Delta(\Phi(w+u) - \Phi(v-w)) + (\tilde{f}-f).  
\end{align*}
and $w(x,0) = v_0(x) - u_0(x)$ as well as satisfies the homogeneous Dirichlet boundary condition. We need to check that the problem satisfies (\ref{uniform_parabolicity}), and to that end we notice that
\begin{equation*}
\Delta(\Phi(w+u) - \Phi(v-w)) = \nabla \cdot \left(\Phi'(w+u)(\nabla w + \nabla u) - \Phi'(v-w)(\nabla v - \nabla w) \right).
\end{equation*}
Treating $u$ and $v$ as given parameters of the IVBP we get
\begin{equation*}
a^i = (\Phi'(w+u) + \Phi'(v-w))\partial_{x_i}w,
\end{equation*}
and since (\ref{uniform_parabGPME}) is satisfied the problem is uniformly parabolic.
We have
\begin{equation*}
\partial_tw - \Delta(\Phi(w+u) - \Phi(v-w)) = \tilde{f}-f \geq 0 
\end{equation*}
and we can use the second part of the strong maximum principle to deduce that
\begin{equation*}
w \geq \underset{\overline{Q_T}}{\mathrm{min}}w = \underset{\Gamma_T}{\mathrm{min}}w.
\end{equation*}
By the initial- and boundary data of $w$ we see that
\begin{equation*}
\underset{\Gamma_T}{\mathrm{min}}w = 0,
\end{equation*}
and so
\begin{equation*}
w \geq 0 \quad \Rightarrow  \quad v \geq u.
\end{equation*}

We conclude the proof by seeing that if $u=v$ at some points in $Q_T$ the strong maximum principle implies that $u=v$ everywhere. 
\end{proof}

\setcounter{rem}{0}
\begin{rem}
We could improve somewhat on this result by having $v \geq u$ on $\partial \Omega \times [0,T]$. All the steps in the proof works, in particular
\begin{equation*}
\underset{\Gamma_T}{\mathrm{min}}w \geq 0
\end{equation*} 
still holds.
\end{rem}

Let's now use this lemma to get an $L^\infty$-bound on classical solutions.
\begin{lemma}
If $u \in C^{2,1}(Q_T)\cap C(\overline{Q}_T)$ is a solution of (\ref{HDP}), and $u_0$ is bounded in $\Omega$ and $f$ bounded in $Q_T$, then
	\begin{equation}
		\label{infinity_bound}
		||u||_{L^\infty(Q_T)} \leq ||u_0||_{L^\infty(\Omega)} + T||f||_{L^\infty(Q_T)}.
	\end{equation}
\end{lemma}

\begin{proof}
We start by defnining $M_1 = \mathrm{max}\{0, \mathrm{sup}_\Omega(u_0) \}$, and $N_1 = \mathrm{max}\{0, \mathrm{sup}_{Q_T}(f) \}$. With $v(x,t) = M_1 + N_1t$, it solves
\begin{equation*}
\begin{cases}
	\partial_t v &= \Delta\Phi(v) +N_1 \quad \text{in } Q_T\\
	v(x,0) &= M_1 \quad \text{in } \Omega \\
	v(x,t) &= M_1 + N_1t \quad \text{on } \partial \Omega \times [0,T].
\end{cases}
\end{equation*}
Since $u_0 \leq M_1$ and $f\leq N_1$ we can use the comparison principle to conclude that
\begin{equation*}
u(x,t) \leq M_1 + N_1t
\end{equation*}
in $Q_T$.

Similarly we define $M_2 = \mathrm{min}\{ 0, \mathrm{inf}_\Omega(u_0) \}$, and $N_2 = \mathrm{min}\{ 0, \mathrm{inf}_{Q_T}(f) \}$, to get the lower bound
\begin{equation*}
u(x,t) \geq M_2 + N_2t.
\end{equation*}

Because
\begin{equation*}
||u_0||_{L^\infty(\Omega)} = \mathrm{max}\{ \mathrm{sup}_\Omega(u_0), -\mathrm{inf}_\Omega (u_0)\} = \mathrm{max}\{|M_1|,|M_2|\},
\end{equation*}
and similarly for $f$, our two bounds implies that
\begin{equation*}
|u(x,t)| \leq ||u_0||_{L^\infty(\Omega)} + t||f||_{L^\infty(Q_T)},
\end{equation*}
which  is what we wanted to prove.

\end{proof}


%%%%%%%%%%%%%%%%%%%%%%%%%%%%%%%%%%%%%%%%%%%%%%%%%%%%%%%
%				L^1-CONTRACTIVITY
%%%%%%%%%%%%%%%%%%%%%%%%%%%%%%%%%%%%%%%%%%%%%%%%%%%%%%%
\subsection{$L^1$-contractivity}
In this section we let $u,\hat{u} \in C^{2,1}(Q_T)\cap C(\overline{Q}_T)$ be two smooth solutions of the homogeneous Dirichlet problem (\ref{HDP}) with initial data $u_0$, $\hat{u}_0$ and loading $f$, $\hat{f}$ respectively. To reach our goal of deriving stability of solutions in $L^1$ we start off with the following lemma, from which the main result will be an almost immediate consequence.

\begin{lemma}
For $\tau, t \in [0,T]$ the inequality 
\begin{equation}
\label{l1_contr_part1}
\int_{\Omega}(u(t)-\hat{u}(t))_+dx \leq \int_{\Omega}(u(\tau)-\hat{u}(\tau))_+dx + \int_\tau^t\int_{\Omega}(f-\hat{f})_+dxdt.
\end{equation}
is satisfied.
\end{lemma}

\begin{proof}
Let $p: \mathbb{R} \to \mathbb{R}$ be a $C^1$ function satisfying $p(s) = 0$ for $s \leq 0$, $p'(s) \geq 0$ for all $s\in \mathbb{R}$, and $0 \leq p(s) \leq 1$ for $s > 0$. Define now $w = \Phi(u) - \Phi(\hat{u})$, and multiply $p(w)$ with the difference of (\ref{GPME}) for $u$ and $\hat{u}$ to get
\begin{equation*}
\partial_t(u-\hat{u})p(w) = p(w)\Delta w + p(w)f.
\end{equation*}
Integrating over $\Omega$, and using integration by parts on the first part on the right hand side yields
\begin{align*}
\int_{\Omega}\partial_t(u-\hat{u})p(w)dx &= -\int_{\Omega}p'(w)|\nabla w|^2dx + \int_{\Omega}p(w)(f-\hat{f}) dx \\
	&\leq \int_{\Omega}p(w)(f-\hat{f}) dx \\
	&\leq \int_{\Omega}(f-\hat{f})_+ dx,
\end{align*}
where the last step follows from the fact that $0 \leq p(w) \leq 1$. Here $(\cdot)_+ = \mathrm{max}\{\cdot, 0\}$.
Now, let $p$ converge to the Heaviside function $H(w)$. Then, it is worth noticing that $H(w) = H(u-\hat{u})$, because of the strict monotonicity of $\Phi$. Also see that $\partial_t(u-\hat{u})_+ = \partial_t(u-\hat{u})H(u-\hat{u})$. Thus
\begin{equation*}
\partial_t \int_{\Omega}(u-\hat{u})_+dx \leq \int_{\Omega}(f-\hat{f})_+ dx,
\end{equation*}
and integrating in time from $\tau$ to $t$ yields
\begin{equation*}
\int_{\Omega}(u(t)-\hat{u}(t))_+dx \leq \int_{\Omega}(u(\tau)-\hat{u}(\tau))_+dx + \int_\tau^t\int_{\Omega}(f-\hat{f})_+dxdt.
\end{equation*}
\end{proof}


\begin{cor}[$L^1$-contractivity]
For two smooth solutions, $u$ and $\hat{u}$ of the homoegenous Dirichlet problem (\ref{HDP}) with different initial data and loading, the $L^1$-estimate
\begin{equation}
\label{l1_contractivity}
||u(t) - \hat{u}(t)||_{L^1(\Omega)} \leq ||u_0 - \hat{u}_0||_{L^1(\Omega)} + ||f-\hat{f}||_{L^1(Q_t)}
\end{equation}
holds for every $t\in [0,T]$.
\end{cor}

\begin{proof}
Setting $\tau = 0$ we have from the previous lemma that
\begin{equation*}
\int_\Omega  (u(t)-\hat{u}(t))_+dx \leq \int_\Omega (u_0 - \hat{u}_0)_+dx + \int_0^t \int_\Omega (f-\hat{f})_+ dxdt,
\end{equation*}
and by interchanging $u$ and $\hat{u}$ we also have 
\begin{equation*}
\int_\Omega  (\hat{u}(t)-u(t))_+dx \leq \int_\Omega (\hat{u}_0 - u_0)_+dx + \int_0^t \int_\Omega (\hat{f}-f)_+ dxdt.
\end{equation*}

Since $|a| = (a)_+ + (-a)_+$, the sum of these inequalities yields the desired result.
\end{proof}

\setcounter{obs}{0}
\begin{obs}

\end{obs}













 

%%%%%%%%%%%%%%%%%%%%%%%%%%%%%%%%%%%%%%%%%%%%%%%%%%%%%%%
%				WEAK SOLUTIONS
%%%%%%%%%%%%%%%%%%%%%%%%%%%%%%%%%%%%%%%%%%%%%%%%%%%%%%%
\section{Weak solutions}



%%%%%%%%%%%%%%%%%%%%%%%%%%%%%%%%%%%%%%%%%%%%%%%%%%%%%%%
%				BIBLIOGRAPHY
%%%%%%%%%%%%%%%%%%%%%%%%%%%%%%%%%%%%%%%%%%%%%%%%%%%%%%%
\newpage
\bibliographystyle{plainnat}
\bibliography{project_bibliography.bib}
\end{document}
