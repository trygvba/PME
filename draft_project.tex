\documentclass[11pt, a4paper]{article}
\usepackage{mathptmx}
\usepackage{mathtools}
\usepackage{amssymb}
%\usepackage{amsfonts}
\usepackage{amsmath}
\usepackage{fullpage}
\usepackage{enumerate}
\usepackage{amsthm}
\usepackage{remreset}
\usepackage[utf8x]{inputenc}
\usepackage[toc,page]{appendix}
\usepackage[numbers]{natbib}

\usepackage{lmodern}

\begin{document}
\title{Draft for project}
\author{Trygve Bærland}
\maketitle
\abstract{For now, heavily reliant on \citep{vazquez2007porous}.}

% My text categories:
\theoremstyle{plain}
\newtheorem{theorem}{Theorem}[section]
\newtheorem{proposition}{Proposition}[section]
\newtheorem{cor}{Corollary}[section]
\newtheorem{lemma}{Lemma}[section]

\theoremstyle{definition}
\newtheorem{mydef}{Definition}[section]
\newtheorem{example}{Example}[section]

%\theoremstyle{plain}
\newtheorem{obs}{Observation}
\newtheorem{rem}{Remark}


% Reset numbering for each section
\numberwithin{equation}{section}
%%%%%%%%%%%%%%%%%%%%%%%%%%%%%%%%%%%%%%%%%%%%%%%%%%%%%%%
%				INTRODUCTION
%%%%%%%%%%%%%%%%%%%%%%%%%%%%%%%%%%%%%%%%%%%%%%%%%%%%%%%
\section{Introduction}

Let $\Omega \subset \mathbb{R}^d$ be an open, bounded, and simply connected domain. Then the filtration, or general porous medium equation is

\begin{equation}
\label{GPME}
	\partial_t u = \Delta \Phi(u) + f,
\end{equation}

 where $u, f: \Omega \times (0,\infty) \to \mathbb{R}$, and $\phi: \mathbb{R} \to \mathbb{R}$. For now we assume that $\Phi$ is continuous and nondecreasing.

\subsection{Basic sets and notation}
Before delving into the actual the actual analysis of our problem, let's just get assure ourselves that we're on the same page when it comes to notation of various sets and function spaces.

The begin, we let $\Omega \subset \mathbb{R}^d$ be an open, bounded and simply connected set, with boundary $\partial \Omega$. This will denote our spatial domain, and so we let write $Q_T = \Omega \times (0,T]$ to mean the space-time cylinder up to a time $T$, which can be both finite or infinite, and it will be stated clearly when we need to assume $T < \infty$.  The last set pertaining to the space-time domain we'll need is the parabolic boundary $\Gamma_T = (\partial \Omega \times [0,T]) \cup (\Omega\times \{t=0\})$.


Most, if not all, of the notation concerning function spaces will follow the notation in \citep{evans} (see in particular Ch. 5). 
 
\subsection{Problem statement}
We will consider on the following Dirichlet problem:
\begin{equation}
\label{HDP}
	\begin{cases}
		\partial_t u = \Delta \Phi(u) + f, \quad \text{ in } Q_T \\
		u(x,0) = u_0(x), \quad \text{ in } \Omega \\
		u(x,t) = 0, \quad \text{ on } \partial\Omega \times [0,T].
	\end{cases}
\end{equation}
Here $Q_T = \Omega \times (0,T]$, where $T$ may very well be infinite. Furthermore we denote by $\Gamma_T = \overline{Q}_T\backslash Q_T$ the parabolic boundary.
%%%%%%%%%%%%%%%%%%%%%%%%%%%%%%%%%%%%%%%%%%%%%%%%%%%%%%%
%				CLASSICAL SOLUTIONS
%%%%%%%%%%%%%%%%%%%%%%%%%%%%%%%%%%%%%%%%%%%%%%%%%%%%%%%
\section{Classical solutions}
In this section it is helpful to reformulate (\ref{HDP}) as 
\begin{equation}
	\begin{cases}
		\partial_t u + Lu &= f, \quad \text{in } Q_T \\
		u(x,0) &= u_0(x), \quad \text{in } \Omega \\
		u &= 0, \quad \text{on } \partial \Omega \times [0,T]
	\end{cases}
\end{equation}
where $L$ is a differential operator that can be written in the divergence form
\begin{equation}
Lu = -\sum_{i=1}^d \partial_{x_i}(a^i(x,t,u,\nabla u)) + a(x,t,u,\nabla u).
\end{equation}
We will need to assume that $a^i$ for every $i$ as well as $a$ are continuous functions with respect to all variables.

In addition we assume that there are constants $0 < \nu < \mu < \infty $ so that
\begin{equation}
\label{uniform_parabolicity}
\nu |\xi|^2 \leq \sum_{i,j=1}^d \frac{\partial a^i(x,t,u,p)}{\partial p_j}\xi_i \xi_j \leq \mu |\xi|^2 
\end{equation}
for every $\xi \in \mathbb{R}^d$. If this condition is satisfied we say that the operator $\partial_t + L$ is \textbf{uniformly parabolic}.

In our setting it's easy enough to check that
\begin{equation*}
a^i(x,t,u,p) = \Phi'(u)p_i,
\end{equation*}
and so (\ref{uniform_parabolicity}) translates to
\begin{equation}
\label{uniform_parabGPME}
\nu \leq \Phi'(u) \leq \mu.
\end{equation}
We will in this section assume this to be true. I.e. that $\Phi'(u)$ is bounded from above as well as away from zero.
%%%%%%%%%%%%%%%%%%%%%%%%%%%%%%%%%%%%%%%%%%%%%%%%%%%%%
%			MAXIMUM PRINCIPLE:
%%%%%%%%%%%%%%%%%%%%%%%%%%%%%%%%%%%%%%%%%%%%%%%%%%%%%
\subsection{Maximum principle}
The first main estimate we'll deal with is a $L^\infty$ bound on a classical solution. To that end, the following strong maximum principle will be helpful (cf. \citep[p. 396]{evans}).

\begin{theorem}[Strong maximum principle]
\begin{enumerate}[i)]
	Suppose $\Omega$ is connected, and $u\in C^{2,1}(Q_T)\cap C(\overline{Q}_T)$, then:
	\item If
	\begin{equation*}
		\partial_tu + Lu \leq 0, \quad \text{ in } Q_T, 
	\end{equation*}
	and $u$ attains its maximum over $\overline{Q}_T$ at point $(x_0,t_0)\in Q_T$, then
	$u$ is constant on $Q_{t_0}$.
	
	\item Analagously, if
	\begin{equation*}
		\partial_tu + Lu \geq 0, \quad \text{ in } Q_T, 
	\end{equation*}
	and $u$ attains its minimum over $\overline{Q}_T$ at a points  $(x_0,t_0) \in Q_T$, then $u$ is constant on $Q_{t_0}$.
\end{enumerate}
\end{theorem}

Using this theorem we can prove the comparison principle:
\begin{lemma}[Comparison principle]
\label{lemma:comparison}
Let $\Phi \in C^2(\mathbb{R})$ and satisfy (\ref{uniform_parabGPME}), and let $u,v \in C^{2,1}(Q_T)\cap C(\overline{Q}_T)$ be two solutions of the General Porous Medium Equation (\ref{GPME}) in $Q_T$ with initial data $u_0$, $v_0$, loading $f$, $\tilde{f}$ respectively as well as prescribed Dirchlet data on the boundary. Further, assume that $v\big|_{\Gamma_T} \geq u\big|_{\Gamma_T}$, and  $\tilde{f} \geq f$ in $Q_T$. Then $u \leq v$ in $Q_T$.
\end{lemma}
\begin{proof}
The strategy will be to set $w = v-u$ and derive a initial- and boundary value problem that  $w$ solves. First off, we see that
\begin{align*}
\partial_t w &= \Delta(\Phi(v) - \Phi(u)) + (\tilde{f}-f) \\
&= \Delta ( \Phi(w+u) - \Phi(v-w) ) + (\tilde{f}-f)
\end{align*}
So, when treating $u$ and $v$ as parameters, we can formulate the problem
\begin{equation}
\begin{cases}
\partial_t w &= \Delta ( \Phi(w+u) - \Phi(v-w) ) + (\tilde{f}-f), \quad \text{ in } Q_T \\
w(x,0) &= v_0(x) - u_0(x), \quad \text{ in } \Omega \\
w(x,t) &= v(x,t) - u(x,t), \quad \text{ on } \partial \Omega \times [0,T].
\end{cases}
\end{equation}
Now we want to use the strong maximum principle, so we should find out what $a^i$ and $a$ are. We have that
\begin{align*}
\Delta ( \Phi(w+u) - \Phi(v-w) )&= \sum_{i=1}^d \partial_{x_i}\left(\Phi'(w+u)(\partial_{x_i}w + \partial_{x_i}u) - \Phi'(v-w)(\partial_{x_i}v-\partial_{x_i}w) \right),
\end{align*} 
and seeing as $u$ and $v$ should be considered parameters to this problem we can establish that
\begin{equation*}
\begin{cases}
a^i(x,t,w,p) &= (\Phi'(w+u) + \Phi'(v-w))p_i \\
a(x,t,w,p) &= \sum_{i=1}^d \partial_{x_i}\left( \Phi'(u+w)\partial_{x_i}u - \Phi'(v-w)\partial_{x_i}v\right).
\end{cases}
\end{equation*}
The $a^i$'s are continuous from the assumptions on $\Phi$, and in addition it satisfies (\ref{uniform_parabGPME}) inherited from $\Phi$. The $a$ is continuous since both $u$ and $v$ are in $C^{2,1}(Q_T)$ and $\Phi \in C^2(\mathbb{R})$.

We can thus use the maximum principle together with $\tilde{f}-f \geq 0$ to deduce that
\begin{equation*}
w \geq \mathrm{min}_{\overline{Q}_T}w = \mathrm{min}_{\Gamma_T}w.
\end{equation*}
By assumption $\mathrm{min}_{\Gamma_T}w \geq 0$, so $v \geq u$. We conclude the proof that seeing that if $u=v$ at some internal point $(x_0,t_0) \in Q_T$, then $u=v$ in $Q_{t_0}$.
\end{proof}


Let's now use this lemma to get an $L^\infty$-bound on classical solutions.
\begin{lemma}
Assume $\Phi \in C^2(\mathbb{R})$ and satisfy (\ref{uniform_parabGPME}). If $u \in C^{2,1}(Q_T)\cap C(\overline{Q}_T)$ is a solution of (\ref{HDP}), and $u_0$ is bounded in $\Omega$ and $f$ bounded in $Q_T$, then
	\begin{equation}
		\label{infinity_bound}
		||u||_{L^\infty(Q_T)} \leq ||u_0||_{L^\infty(\Omega)} + T||f||_{L^\infty(Q_T)}.
	\end{equation}
\end{lemma}

\begin{proof}
We start by defnining $M_1 = ||u_0^+||_{L^\infty(\Omega)} $, and $N_1 = ||f^+||_{L^\infty(Q_T)}$, where $(\cdot)^+ = \mathrm{max} \{ 0,\cdot \}$. With $v(x,t) = M_1 + N_1t$, it solves
\begin{equation*}
\begin{cases}
	\partial_t v &= \Delta\Phi(v) +N_1 \quad \text{in } Q_T\\
	v(x,0) &= M_1 \quad \text{in } \Omega \\
	v(x,t) &= M_1 + N_1t \quad \text{on } \partial \Omega \times [0,T].
\end{cases}
\end{equation*}
Since $u_0 \leq M_1$ and $f\leq N_1$ we can use the comparison principle to conclude that
\begin{equation*}
u(x,t) \leq M_1 + N_1t
\end{equation*}
in $Q_T$.

Similarly we define $M_2 = -||u_0^-||_{L^\infty(\Omega)}$, and $N_2 = -||f^-||_{L^\infty(Q_T)}$. To avoid any possible confusion, here $(\cdot)^- = \mathrm{min}\{0, \cdot \}$. We use the same reasoning to get the lower bound
\begin{equation*}
u(x,t) \geq M_2 + N_2t.
\end{equation*}

Because
\begin{equation*}
||u_0||_{L^\infty(\Omega)} = \mathrm{max}\{ ||u_0^+||_{L^\infty(\Omega)}, ||u_0^-||_{L^\infty(\Omega)} \} = \mathrm{max}\{|M_1|,|M_2|\},
\end{equation*}
and similarly for $f$, our two bounds implies that
\begin{equation*}
|u(x,t)| \leq ||u_0||_{L^\infty(\Omega)} + t||f||_{L^\infty(Q_T)},
\end{equation*}
which  is what we wanted to prove.

\end{proof}


%%%%%%%%%%%%%%%%%%%%%%%%%%%%%%%%%%%%%%%%%%%%%%%%%%%%%%%
%				L^1-CONTRACTIVITY
%%%%%%%%%%%%%%%%%%%%%%%%%%%%%%%%%%%%%%%%%%%%%%%%%%%%%%%
\subsection{$L^1$-contractivity}
In this section we let $u,\hat{u} \in C^{2,1}(Q_T)\cap C(\overline{Q}_T)$ be two smooth solutions of the homogeneous Dirichlet problem (\ref{HDP}) with initial data $u_0$, $\hat{u}_0$ and loading $f$, $\hat{f}$ respectively. To reach our goal of deriving stability of solutions in $L^1$ we start off with the following lemma, from which the main result will be an almost immediate consequence.

\begin{lemma}
For $\tau, t \in [0,T]$ the inequality 
\begin{equation}
\label{l1_contr_part1}
\int_{\Omega}(u(t)-\hat{u}(t))_+dx \leq \int_{\Omega}(u(\tau)-\hat{u}(\tau))_+dx + \int_\tau^t\int_{\Omega}(f-\hat{f})_+dxdt.
\end{equation}
is satisfied.
\end{lemma}

\begin{proof}
Let $p: \mathbb{R} \to \mathbb{R}$ be a $C^1$ function satisfying $p(s) = 0$ for $s \leq 0$, $p'(s) \geq 0$ for all $s\in \mathbb{R}$, and $0 \leq p(s) \leq 1$ for $s > 0$. Define now $w = \Phi(u) - \Phi(\hat{u})$, and multiply $p(w)$ with the difference of (\ref{GPME}) for $u$ and $\hat{u}$ to get
\begin{equation*}
\partial_t(u-\hat{u})p(w) = p(w)\Delta w + p(w)f.
\end{equation*}
Integrating over $\Omega$, and using integration by parts on the first part on the right hand side yields
\begin{align*}
\int_{\Omega}\partial_t(u-\hat{u})p(w)dx &= -\int_{\Omega}p'(w)|\nabla w|^2dx + \int_{\Omega}p(w)(f-\hat{f}) dx \\
	&\leq \int_{\Omega}p(w)(f-\hat{f}) dx \\
	&\leq \int_{\Omega}(f-\hat{f})_+ dx,
\end{align*}
where the last step follows from the fact that $0 \leq p(w) \leq 1$. Here $(\cdot)_+ = \mathrm{max}\{\cdot, 0\}$.
Now, let $p$ converge to the Heaviside function $H(w)$. Then, it is worth noticing that $H(w) = H(u-\hat{u})$, because of the strict monotonicity of $\Phi$. Also see that $\partial_t(u-\hat{u})_+ = \partial_t(u-\hat{u})H(u-\hat{u})$. Thus
\begin{equation*}
\partial_t \int_{\Omega}(u-\hat{u})_+dx \leq \int_{\Omega}(f-\hat{f})_+ dx,
\end{equation*}
and integrating in time from $\tau$ to $t$ yields
\begin{equation*}
\int_{\Omega}(u(t)-\hat{u}(t))_+dx \leq \int_{\Omega}(u(\tau)-\hat{u}(\tau))_+dx + \int_\tau^t\int_{\Omega}(f-\hat{f})_+dxdt.
\end{equation*}
\end{proof}


\begin{cor}[$L^1$-contractivity]
For two smooth solutions, $u$ and $\hat{u}$ of the homoegenous Dirichlet problem (\ref{HDP}) with different initial data and loading, the $L^1$-estimate
\begin{equation}
\label{l1_contractivity}
||u(t) - \hat{u}(t)||_{L^1(\Omega)} \leq ||u_0 - \hat{u}_0||_{L^1(\Omega)} + ||f-\hat{f}||_{L^1(Q_t)}
\end{equation}
holds for every $t\in [0,T]$.
\end{cor}

\begin{proof}
Setting $\tau = 0$ we have from the previous lemma that
\begin{equation*}
\int_\Omega  (u(t)-\hat{u}(t))_+dx \leq \int_\Omega (u_0 - \hat{u}_0)_+dx + \int_0^t \int_\Omega (f-\hat{f})_+ dxdt,
\end{equation*}
and by interchanging $u$ and $\hat{u}$ we also have 
\begin{equation*}
\int_\Omega  (\hat{u}(t)-u(t))_+dx \leq \int_\Omega (\hat{u}_0 - u_0)_+dx + \int_0^t \int_\Omega (\hat{f}-f)_+ dxdt.
\end{equation*}

Since $|a| = (a)_+ + (-a)_+$, the sum of these inequalities yields the desired result.
\end{proof}

\setcounter{obs}{0}
\begin{obs}
From this corollary, we get out immediately that (\ref{HDP}) has at most one solution in $C^{2,1}(Q_T)\cap C(\overline{Q}_T)$.
\end{obs}

%%%%%%%%%%%%%%%%%%%%%%%%%%%%%%%%%%%%%%%%%%%%%%%
%			SPATIAL DER. BOUND
%%%%%%%%%%%%%%%%%%%%%%%%%%%%%%%%%%%%%%%%%%%%%%%
\subsection{Control of the spatial derivative}
This bound follows nicely the $L^1$-contractivity in that its derivation hinges on a clever choice of function to multiply (\ref{GPME}) with. Before proceeding, we define  $\Psi$ as the primitive of $\Phi$ with $\Psi(0) = 0$, i.e.
\begin{equation}
\label{phi_primitive}
	\Psi(s) = \int_0^s \Phi(u) ds,
\end{equation}

The bound on the spatial derivative is then summarised in the following lemma.

\begin{lemma}
For a solution $u\in C^{2,1}(Q_T) \cap C(\overline{Q}_T)$ of (\ref{HDP})
\begin{equation}
\label{spatial_derivative_bound}
\iint_{Q_T} |\nabla \Phi(u)|^2dxdt + \int_\Omega \Psi(u(T))dx = \int_\Omega \Psi(u_0)dx + \iint_{Q_T} f\Phi(u)dxdt.
\end{equation}
\end{lemma}

\begin{proof}
We start off by multiplying (\ref{GPME}) with $\Phi(u)$ to get
\begin{equation*}
\Phi(u)\partial_t u = \Phi(u)\Delta \Phi(u) + \Phi(u)f,
\end{equation*}
and we notice that $\Phi(u)\partial_t u = \partial_t \Psi(u)$. That means when we integrate over $Q_T$ we have
\begin{equation*}
\iint_{Q_T} \partial_t \Psi(u)dxdt = \int_\Omega \Psi(u(T))dx - \int_\Omega \Psi(u_0)dx.
\end{equation*}
On the diffusive term we can use integration by parts, so
\begin{equation*}
\iint_{Q_T}\Phi(u) \Delta \Phi(u) dxdt = -\iint_{Q_T} |\nabla \Phi(u)|^2dxdt + \int_0^T \int_{\partial\Omega} \Phi(u)\nabla \Phi(u) \cdot \hat{n}dS,
\end{equation*}
where $\hat{n}$ is the outward unit normal. By the Dirichlet condition
\begin{equation*}
\iint_{Q_T}\Phi(u) \Delta \Phi(u) dxdt = -\iint_{Q_T} |\nabla \Phi(u)|^2dxdt,
\end{equation*}
and notice that this step would have been valid for homogeneous Neumann condition as well. In toto 
\begin{equation*}
\iint_{Q_T} |\nabla \Phi(u)|^2dxdt + \int_\Omega \Psi(u(T))dx = \int_\Omega \Psi(u_0)dx + \iint_{Q_T} f\Phi(u)dxdt.
\end{equation*}
\end{proof}
\setcounter{obs}{0}

\begin{obs}
We can go even further and deal with the last term on the right hand side: Assuming $f,\Phi(u) \in L^2(Q_T)$, then
\begin{equation*}
\iint_{Q_T} f\Phi(u)dxdt \leq \left(\iint_{Q_T} f^2 dxdt\right)^\frac{1}{2}\left(\iint_{Q_T} \Phi(u)^2 dxdt\right)^\frac{1}{2},
\end{equation*}
by Cauchy-Schwarz'. With the inequality $\sqrt{ab} \leq \frac{1}{4c}a + cb$ for $c>0$
\begin{equation*}
\iint_{Q_T} f\Phi(u)dxdt \leq \frac{1}{4c}\iint_{Q_T}f^2dxdt + c\iint_{Q_T}\Phi(u)^2dxdt.
\end{equation*}
Lastly, because of the homogeneous Dirichlet conditions, we use the Poincaré inequality and set $c$ so that
\begin{equation}
\label{using_Poincare}
\iint_{Q_T} f\Phi(u)dxdt \leq C\iint_{Q_T}f^2dxdt + \frac{1}{2}\iint_{Q_T}|\nabla \Phi(u)|^2dxdt,
\end{equation}
and the final inequality becomes
\begin{equation}
\frac{1}{2}\iint_{Q_T}|\nabla \Phi(u)|^2dxdt + \int_\Omega \Psi(u(T))dx \leq \int_\Omega \Psi(u_0)dx + C\iint_{Q_T}f^2dxdt.
\end{equation}
\end{obs}
%%%%%%%%%%%%%%%%%%%%%%%%%%%%%%%%%%%%%%%%%%%%%%%%%%%%%%%%%%%%
%				TEMPORAL DER. BOUND
%%%%%%%%%%%%%%%%%%%%%%%%%%%%%%%%%%%%%%%%%%%%%%%%%%%%%%%%%%%%
\subsection{Control of the temporal derivative}
A similar, but perhaps a bit more involved argument can be used to make an estimate the temporal derivative.

Let $\zeta = \zeta(t)$ be a $C^1([0,T],[0,1])$ cutoff function so that $\zeta(0) = \zeta(T) = 1$. If we multiply (\ref{GPME}) by $\zeta \partial_t\Phi(u)$ and integrate over $Q_T$ we get
	\begin{equation}
	\label{temporal_bound_start}
	\iint_{Q_T}\zeta \Phi'(u)|\partial_t u|^2 dxdt = \iint_{Q_T} \left\{ \zeta \partial_t \Phi(u) \Delta \Phi(u)  + \zeta \partial_t \Phi(u) f \right\} dxdt.
	\end{equation}
	Let's start by considering the first term on the right hand side. By the product rule for the divergence
	\begin{equation*}
	\int_0^T \zeta \int_\Omega \partial_t \Phi(u) \Delta \Phi(u) dxdt = \int_0^T \zeta \int_\Omega \left\{ \nabla \cdot (\partial_t \Phi(u) \nabla \Phi(u)) - (\partial_t \nabla \Phi(u)) \cdot \nabla \Phi(u) \right\} dx dt.
	\end{equation*}
	So after using the divergence theorem together with the homogeneous boundary data, as well as noticing that $(\partial_t \nabla \Phi(u)) \cdot \nabla \Phi(u) = \frac{1}{2}\partial_t |\nabla \Phi(u)|^2$, we see that
	\begin{equation*}
	\int_0^T \zeta \int_\Omega \partial_t \Phi(u) \Delta \Phi(u) dxdt = -\iint_{Q_T}\frac{\zeta}{2}\partial_t |\nabla \Phi(u)|^2 dxdt.
	\end{equation*}
	Integrating by parts in the temporal dimension, we end up with
	\begin{equation*}
	\int_0^T \zeta \int_\Omega \partial_t \Phi(u) \Delta \Phi(u) dxdt = \iint_{Q_T} \frac{\partial_t \zeta}{2} |\nabla \Phi(u)|^2 dxdt - \frac{1}{2}\int_\Omega \left\{ |\nabla \Phi(u(T))|^2 - |\nabla \Phi(u_0)|^2 \right\} dx.
	\end{equation*}
	A similar inetgration by parts in the temporal dimension of the last term on the right hand side in (\ref{temporal_bound_start}) yields
	\begin{equation*}
	\iint_{Q_T} \zeta f \partial_t \Phi(u) dxdt = -\iint_{Q_T} \partial_t (\zeta f) \Phi(u) dxdt + \int_\Omega \left\{ f(T)\Phi(u(T)) - f(0)\Phi(u_0) \right\} dx.
	\end{equation*}
	For the last term in this integral we can do the same trick as we did previously for the energy estimate. We use Cauchy-Schwarz and the inequality $\sqrt{ab} \leq \frac{c^2}{4}a + \frac{1}{c^2}b$. Then, after using a Poincaré inequality we can choose $c$ so that the integrals of $|\nabla \Phi(u)|^2$ at the temporal endpoints are cancelled. In toto we end up with
	\begin{align*}
	\iint_{Q_T} \zeta \Phi'(u) |\partial_t u|^2 dxdt &\leq \iint_{Q_T} \left\{ \frac{\partial_t \zeta}{2}|\nabla \Phi(u)|^2 - \partial_t(\zeta f) \Phi(u) \right\} dxdt \\
	&+ \frac{C}{2}\int_\Omega \left\{ f^2(T) - f^2(0) \right\} dx	\\
	&= \iint_{Q_T} \left\{ \frac{\partial_t \zeta}{2}|\nabla \Phi(u)|^2 - \partial_t(\zeta f) \Phi(u) \right\} dxdt \\
	&+ C \iint_{Q_T} f\partial_t f dxdt.
	\end{align*}
	
	
A particular choice for $\zeta$, that's of interest is $\zeta(t) = 1$. Then we end up with the estimate
\begin{equation}
\label{temporal_bound}
\iint_{Q_T} \Phi'(u) |\partial_t u|^2 dxdt \leq \iint_{Q_T}\left\{ -\partial_t f \Phi(u)   +Cf\partial_tf \right\}dxdt, 
\end{equation}
and in addition if $\Phi'(u)$ is bounded over all relevant values of $u$, e.g. given by the $L^\infty$-bound, then what we have here is an $L^2$-estimate of $\partial_t \Phi(u)$.
	


%%%%%%%%%%%%%%%%%%%%%%%%%%%%%%%%%%%%%%%%%%%%%%%%%
%				EXISTENCE:
%%%%%%%%%%%%%%%%%%%%%%%%%%%%%%%%%%%%%%%%%%%%%%%%%
\subsection{Existence}
Now we move on to tackle one of the central questions with regards to the well-posedness of (\ref{HDP}), viz. existence, and in particular what sort of assumptions we need to put on $u_0$, $f$ and $\Phi$ to be guaranteed that a solution exists.

We will make good use of an admittedly quite harrowing theorem presented in \citep{ladyzhenskaya}. This theorem is quite general in its formulation and our job here will mainly be to translate all the conditions to the general porous medium equation, but before delving into this mammoth, let's set the stage, and introduce some notation used by Ladyzhenskaya et. al.

\subsubsection{Preliminaries}

As in the section on the maximum principle it will be useful to recast our PDE to the form
\begin{equation}
\label{lady_parabolic_eq}
	u_t - \sum_{i=1}^d\frac{d}{dx_i}a_i(x,t,u,\nabla u) + a(x,t,u,\nabla u) = 0
\end{equation}
in $Q_T$. In addition we formulate the initial value and boundary as
\begin{equation}
\label{lady_bc}
	u\big|_{\Gamma_T} = \psi\big|_{\Gamma_T}
\end{equation}
for some known function $\psi$.

Furthermore, we define
\begin{equation*}
A(x,t,u,\nabla u) = a(x,t,u,\nabla u) - \sum_{i=1}^d \left[ \frac{\partial a_i(x,t,u,\nabla u)}{\partial u} + \frac{\partial a_i(x,t,u,\nabla u)}{\partial x_i} \right]
\end{equation*}

By the function space $H^{l, l/2}$, we mean the Banach space of functions $u$ that are continuous in $\bar{Q}_T$. So are all partial derivatives of the form $\partial_t^r D^\alpha u$, with $2r+|\alpha| <l$. The norm is given as
\begin{equation*}
	|u|_{Q_T}^{(l)} = \langle u \rangle^{(l)}_{Q_T} + \sum_{j=0}^{[l]}\langle u \rangle^{(j)}_{Q_T},
\end{equation*}
with
\begin{align*}
\langle u \rangle^{(j)}_{Q_T} &= \sum_{2r+|\alpha|=j} ||\partial_t^r D^\alpha u||_{C(\bar{Q}_T)} \\
\langle u \rangle^{(l)}_{Q_T} &= \langle u \rangle^{(l)}_{x,Q_T} + \langle u \rangle^{(l/2)}_{t,Q_T} \\
\langle u \rangle^{(l)}_{x,Q_T} &= \sum_{2r+|\alpha|=[l]} |\partial_t^r D^{\alpha}u|_{C^{0,l-[l]}(\bar{Q}_T)} \\
 \langle u \rangle^{(l/2)}_{t,Q_T} &= \sum_{0 < l - 2r -\alpha < 2} |\partial_t^r D^\alpha u|_{C^{0,(l-2r-s)/2}(\bar{Q}_T)},
\end{align*}
with the Hölder semi-norm with respect to $x$ and $t$ respectively in the last two definitions.



\subsubsection{The big, general existence theorem}
We should now will at least partly prepared for the big existence theorem. It is presented immediately below, almost verbatim from \citep[ Theorem 6.1, p. 452]{ladyzhenskaya}.

\begin{theorem}[Existence of classical solutions]
\label{thm:classical_existence}
Suppose that the following conditions hold.
\begin{enumerate}[a)]
	\item For $(x,t) \in \overline{Q}_T$ 
	\begin{equation}
		\label{exist_cond1}
		\frac{\partial a_i(x,t,u,p)}{\partial p_j}\xi_i \xi_j \Bigg|_{p=0} \geq 0, \quad A(x,t,u,0)u \geq -b_1 u^2 - b2
	\end{equation}
	 holds for arbitrary $u$ and $\xi = (\xi_1, \xi_2, \ldots, \xi_d)$.
	
	\item For $(x,t) \in \overline{Q}_T$, $|u| \leq M$ and arbitrary $p$ the functions $a_i(x,t,u,p)$ and $a(x,t,u,p)$ are continuous, the $a_i(x,t,u,p)$ are differentiable with respect to $x$, $u$ and $p$ and the following inequalities are satisfied
	\begin{equation}
	\label{exist_cond3}
		\begin{cases}
			\nu\xi^2 \leq \sum_{i,j=1}^d\frac{\partial a_i(x,t,u,p)}{\partial p_j}\xi_i \xi_j \leq \mu \xi^2 \\
			\sum_{i=1}^d \left(|a_i| + |\frac{\partial a_i}{\partial u}|\right)(1+|p|) + \sum_{i,j=1}^d |\frac{\partial a_i}{\partial x_j}| + |a| \leq \mu (1 + |p|)^2. 		
		\end{cases}
	\end{equation}
	where $\nu$ and $\mu$ are positive numbers.
	
	\item For $(x,t) \in \overline{Q}_T$, $|u| \leq M$ and $|p| \leq M_1$, the functions $a_i$, $a$, $\frac{\partial a_i}{\partial p_j}$, $\frac{\partial a_i}{\partial u}$, and $\frac{\partial a_i}{\partial x_i}$ are continuous functions satisfying a Hölder condition in $x$, $t$, $u$ and $p$ with exponents $\beta$, $\frac{\beta}{2}$, $\beta$ and $\beta$ respectively.
	
	\item For $(x,t) \in \overline{Q}_T$, $|u| \leq M$ and $|p| \leq M_1$ the function $a(x,t,u,p)$ has the partial derivatives $\frac{\partial a}{\partial p_j}$ and $\frac{\partial a}{\partial u}$ and the functions $a_i$ and $a$ satisfy
	\begin{equation}
	\label{exist_cond4}
	\begin{split}
		&\Bigg| \frac{\partial a_i(x,t,u,p)}{\partial u}, \frac{a_i(x,t+h,u,p) - 		a_i(x,t,u,p)}{h}, \\
		&\frac{\partial a}{\partial p}, \frac{\partial a}{\partial u}, \frac{a(x,t+h,u,p) - a(x,t,u,p)}{h} \Bigg| \leq \varphi(x,t).
	\end{split}.
	\end{equation}
	Here $\varphi$ is required to satisfy
	\begin{equation*}
	\left( \int_0^T \left(\int_\Omega |\varphi(x,t)|^q dx\right)^\frac{r}{q} dt 			\right)^\frac{1}{r} < \infty,
	\end{equation*}
	i.e. $\varphi \in L^r(0,T: L^q(\Omega))$, where $q$ and $r$ satisfy
	\begin{equation*}
	\frac{1}{r} + \frac{d}{2q} = 1 - x_1
	\end{equation*}
	and
	\begin{equation*}
	\begin{cases}
		q\in \left[\frac{d}{2(1-x_1)},\infty\right], &r\in \left[\frac{1}{1-x_1},\infty \right], 0<x_1<1, \quad \text{ for } d \geq 2 \\
		q \in [1, \infty], &r \in \left[\frac{1}{1-x_1},\frac{2}{1-2x_1}\right], 0<x_1<\frac{1}{2}, \quad \text{ for } n=1
	\end{cases}
\end{equation*}
	
	\item $\psi(x,t) \in H^{2+\beta, 1+\beta/2}(\overline{Q}_T)$ and satisfy
	\begin{equation}
	\label{exist_cond5}
	\psi_t - \frac{d}{dx_i}a_i(x,t,\psi,\psi_x) + a(x,t,\psi,\psi_x) = 0.
	\end{equation}
	on $\partial \Omega \times \{t=0\}$.
	
	\item $\partial \Omega \in C^{2+\beta}$\footnote{I can't seem to find a definition of this sort of boundary in \citep{ladyzhenskaya}. However, \citep{evans} has something vaguely familiar, but here integer values of the exponent is assumed.}.
\end{enumerate}

Under these conditions there exists a unique solution of problem (\ref{lady_parabolic_eq})  with boundary condition (\ref{lady_bc}) from the class $H^{2+\beta, 1+\beta/2}(\overline{Q}_T)$. Moreover, this solution has derivatives $u_{xt}$ from $L^2(\overline{Q}_T)$.
\end{theorem}

\subsubsection{From general to specific: Existence for the GPME}
The general porous medium equation we are concerned with can be written
\begin{equation*}
u_t - \nabla \cdot (\nabla \Phi(u) ) - f(x,t) = 0,
\end{equation*}
which is of the form (\ref{lady_parabolic_eq}) with
\begin{equation}
a_i(x,t,u,\nabla u) = \Phi'(u) u_{x_i}, \quad a(x,t,u,\nabla u) = -f(x,t).
\end{equation}


Let's go trhough each condition in theorem \ref{thm:classical_existence} to see what assumptions we must make to have a classical solution.

\begin{enumerate}[a)]
	\item We consider fullfilling (\ref{exist_cond1}). We have
	\begin{equation*}
		\frac{\partial a_i(x,t,u,p)}{\partial p_j} = \delta_{ij} \Phi'(u),
	\end{equation*}
	and so, since $\Phi'(u) \geq 0$ is assumed,
	\begin{equation*}
		\frac{\partial a_i(x,t,u,p)}{\partial p_j}\Bigg|_{p=0}\xi_i\xi_j = \Phi'(u)\xi_i^2 \geq 0.
	\end{equation*}
	
	In the general porous medium equation
	\begin{equation*}
		A(x,t,u,p) = -f(x,t) - \sum_{i=1}^d \Phi''(u)p_i,
	\end{equation*}
	which means $A(x,t,u,0) = -f(x,t)$ (Also see that we need $\Phi \in C^2(\mathbb{R})$). Thus, according to (\ref{exist_cond1}) we need constants $b_1$ and $b_2$ so that
	\begin{equation*}
		f(x,t)u \leq b_1 u^2 + b_2.
	\end{equation*}
	We are able to choose $b_1$ and $b_2$ if, for example, $f(x,t)$ is bounded in $\overline{Q}_T$.
	
	\item We need $a_i(x,t,u,p)$ and $a(x,t,u,p)$ to be continuous. This translates to requiring that $\Phi'(u)$ and $f(x,t)$ are continuous. In addition $a_i(x,t,u,p)$ needs to be differentiable with respect to $x$, $t$, $u$ and $p$, which in our case means that $\Phi''(u)$ needs to be continuous.
	
	Considering the first inequality in (\ref{exist_cond3}) we have
	\begin{equation*}
		\sum_{i,j=1}^d \frac{\partial a_i(x,t,u,p)}{\partial p_j} \xi_i \xi_j = \sum_{i=1}^d \Phi'(u)\xi_i^2 =\Phi'(u)\xi^2.
	\end{equation*}
	Assuming that $\Phi'(u)$ is strictly positive and continuous for $|u|\leq M$ we can set
	\begin{equation}
	0 < \nu = \text{inf}_{|u| \leq M}\Phi'(u) < \text{sup}_{|u| \leq M}\Phi'(u) \leq \mu.
	\end{equation}
	
	After establishing that
	\begin{equation*}
		\begin{cases}
			a_i(x,t,u,p) &= \Phi'(u)p_i \\
			\frac{\partial a_i(x,t,u,p)}{\partial u} &= \Phi''(u)p_i \\
			\frac{\partial a_i(x,t,u,p)}{\partial x_j} &= 0,
		\end{cases}	
	\end{equation*}
	we put it into the left hand side of the second inequality of (\ref{exist_cond3}) as
	\begin{align*}
	&\sum_{i=1}^d\left( |a_i| + \left| \frac{\partial a_i}{\partial u}\right| \right)(1+|p|) + \sum_{i,j}^d\left| \frac{\partial a_i}{\partial x_j} \right| + |a| \\
	&= \sum_{i=1}^d\left( |\Phi'(u)| + |\Phi''(u)| \right)|p_i|(1+|p|) + |f(x,t)| \\
	&= (|\Phi'(u)| + |\Phi''(u)|)(1+|p|)\sum_{i=1}^d|p_i| + |f(x,t)|.
	\end{align*}
	By the equivalence of $l_p$-norms in finite dimensions $\sum_i |p_i| \leq C|p|$. Furthermore, we assume $\Phi''(u)$ to be continuous for $|u|\leq M$, then
	\begin{equation*}
	(|\Phi'(u)| + |\Phi''(u)|)(1+|p|)\sum_{i=1}^d|p_i| + |f(x,t)| \leq K(|p| + |p|^2) + |f(x,t)|,
	\end{equation*}
	where $K = C\cdot \text{sup}_{|u|\leq M}(|\Phi'(u)| + |\Phi''(u)|)$. Now we are almost done. All that remains is noticing that
	\begin{align*}
	K(|p| + |p|^2) + |f(x,t)| &\leq K(|p| + |p|^2) + ||f||_{\infty}(1+|p|) \\
	&\leq \text{max}\{ K, ||f||_\infty \} (1+|p|)^2.
	\end{align*}
	So we can set $\mu = \text{max}\{K, ||f||_\infty, \text{sup}_{|u|\leq M}\Phi'(u) \}$, and both inequalities in (\ref{exist_cond3}) are satisfied.
	
	\item To reiterate, we need the functions
	\begin{equation*}
		\begin{cases}
			a_i(x,t,u,p) &= \Phi'(u)p_i \\
			a(x,t,u,p) &= -f(x,t) \\
			\frac{\partial a_i(x,t,u,p)}{\partial p_j} &= \delta_{ij}\Phi'(u) \\
			\frac{\partial a_i(x,t,u,p)}{\partial u} &= \Phi''(u)p_i \\
			\frac{\partial a_i(x,t,u,p)}{\partial x_j} &= 0
		\end{cases}
	\end{equation*}
	to be Hölder continuous in $x$, $t$, $u$ and $p$ with exponents $\beta$, $\frac{\beta}{2}$, $\beta$ and $\beta$ respectively. If we choose $\beta = 1$ (i.e. Lipschitz continuous for all variables except $t$) the only additional assumption we need to make on $\Phi$ is that $\Phi^{(3)}(u)$ is bounded for $|u| \leq M$, which means that $\Phi \in W^{3,\infty}(\mathbb{R})$. The additional assumptions we need to make on $f$ is that it needs to be Lipschitz continuous with respect to $x$ and that
	\begin{equation*}
	|f(x,t) - f(x,s)| \leq C|t-s|^\frac{1}{2}
	\end{equation*}
	for some constant $C$. Note that for bounded domains, Lipschitz continuity implies Hölder continuity with lower exponent. So we could make the kind of a fortiori assumption that $f$ is Lipschitz continuous with respect to time. Or even that $f_t(x,t)$ is bounded.
	
	\item For this condition we see that it seems even more natural to impose that $f$ is Lipschitz continuous in $t$, and that the domain is bounded. I.e. that $T<\infty$ and $\Omega$ is bounded. Then all the previous assumptions we've made on $\Phi$ and $f$ will ensure that this condition holds.
	
	\item We of course have
	\begin{equation*}
		\begin{cases}
		\Psi(x,t) &= 0, \quad (x,t) \in \partial \Omega \times [0,T] \\
		\Psi(x,0) &= u_0(x), \quad x\in \Omega 
		\end{cases}
	\end{equation*}
	and so this condition isn't too hard to handle\footnote{I think...}. First off, we should require that $u_0$ vanish at the boundary, and that it is sufficiently smooth. To be more precise, $u_0$, $u'_0$ and $u''_0$ are all continuous and are zero at the boundary. The condition (\ref{exist_cond5}) holds trivially, because  $\Psi\big|_{\partial \Omega \times \{t=0\}} = 0$.
	
	\item In this condition we can do nothing more than assume sufficient smoothness of the boundary.
\end{enumerate}

It seems only appropriate that we summarise this somehow. A nice little corollary tailored for our purposes is just what we need now.

\begin{cor}
\label{cor:GPME_existence}
Assume the following conditions are satisfied:
	\begin{enumerate}[a)]
		\item 	$\Phi: \mathbb{R} \to \mathbb{R}$ is in $C^2(\mathbb{R})$, strictly increasing, and $\Phi$, $\Phi'$ and $\Phi''$ are all Lipschitz continuous.
		
		\item $f = f(x,t)$ is $C^1(Q_T)$, bounded and Lipschitz in both its variables.
		
		\item $u_0$ is in $C^2_c(\Omega)$; $u_0$ and its derivatives vanish near the boundary $\partial \Omega$.
		
		\item $\partial \Omega$ is $C^{2+\beta}$-smooth.
	
	\end{enumerate}
Then (\ref{HDP}) has a unique solution $u\in C^{2,1}(\bar{Q}_T)$ for $T<\infty$. Moreover, this solution has derivatives $u_{xt}$ in $L^2(Q_T)$.
\end{cor}

We see that when these assumptions are fulfilled, the homogeneous Dirichlet problem is well posed, since uniqueness and stability are ensured by $L^1$-contractivity.


%%%%%%%%%%%%%%%%%%%%%%%%%%%%%%%%%%%%%%%%%%%%%%%%%%%%%%%
%				WEAK SOLUTIONS
%%%%%%%%%%%%%%%%%%%%%%%%%%%%%%%%%%%%%%%%%%%%%%%%%%%%%%%
\newpage
\section{Weak solutions}
In the previous sections we saw what assumptions needs to be made for the homogeneous Dirichlet problem to be well posed. Allthough a very nice result, the assumptions on the data are far too strict. Especially degenerate problems, where $\Phi$ is only increasing and not stricly so, are not valid for this existence result. In addition the assumptions on smoothness are often violated in applications. This heuristic will now lead us into a discussion of weak solutions, a more general sense of solution. 

In this section we will aim to prove well-posedness of weak solutions, and what sort of assumptions we'll need to put on the data to ensure it, well-posedness that is. Again, this follows \citep[Ch. 5]{vazquez2007porous} quite closely. However, the proofs are worked out in greater detail than presented there. We will, unless stated otherwise, assume that $\Phi$ is an increasing function, and not only stricly increasing.

Let's just jump off with the definition of weak solutions that we will be using in this text.
\begin{mydef}[Weak solutions]
A locally integrable function $u$ defined in $Q_T$ is said to be a \textbf{weak solution} of (\ref{HDP}) if
\begin{enumerate}[i)]
	\item $u \in L^1(Q_T)$ and $\Phi(u) \in L^1(0,T : W_0^{1,1}(\Omega))$\footnote{$f:\Omega \to \mathbb{R}$ is an element of $W^{k,p}(\Omega)$ if all weak derivatives with $|\alpha| \leq k$ exists and are in $L^p(\Omega)$. $W_0^{k,p}$ is the closure of $C^\infty_c(\Omega)$ in $W^{k,p}(\Omega)$.};
	\item $u$ satisfies
		\begin{equation}
		\label{weak}
		\iint_{Q_T} \{\nabla\Phi(u)\cdot \nabla\eta - u\partial_t \eta \}dxdt = \int_\Omega u_0(x)\eta(x,0)dx + \iint_{Q_T}f\eta dxdt
		\end{equation}
for any $\eta \in C^1(\overline{Q}_T)$ that vanishes on $\partial \Omega \times [0,T)$ and for $t=T$.
\end{enumerate}
\end{mydef}

\setcounter{obs}{0}
\begin{obs}
With $u_0 \in L^1(\Omega)$ and $f\in L^1(Q_T)$, all the integrals in (\ref{weak}) make sense.
\end{obs}
\begin{obs}
Classical solutions are also weak solutions. This is fairly straight forward to see. Let $u$ be a smooth solutions and multiply (\ref{GPME}) with a test function $\eta$. Then, when integrating over $Q_T$
\begin{equation*}
\iint_{Q_T}\partial_t u \eta dxdt = \iint_{Q_T}\Delta\Phi(u) \eta + f\eta dxdt.
\end{equation*}
Using integration by parts in time on the left hand side and in space on the first term on the right hand side yields
\begin{equation*}
\int_\Omega u\eta dx \Bigg|_{t=0}^T - \iint_{Q_T}u\partial_t \eta dxdt = -\iint_{Q_T}\nabla \Phi(u) \cdot \nabla \eta dxdt + \iint_{Q_T}f\eta dxdt,
\end{equation*}
and now all that remains is rearranging the terms to get (\ref{weak}).
\end{obs}

%%%%%%%%%%%%%%%%%%%%%%%%%%%%%%%%%%%%%%%%%%%%%%%%%%%%%%%
%				UNIQUENESS
%%%%%%%%%%%%%%%%%%%%%%%%%%%%%%%%%%%%%%%%%%%%%%%%%%%%%%%
\subsection{Uniqueness}

\begin{theorem}[Uniqueness]
\label{thm:uniqueness}
	Assuming in addition that $u \in L^2(Q_T)$ and $\Phi(u) \in L^2(0,T : H_0^1(\Omega))$, (\ref{HDP}) has at most one weak solution.
\end{theorem}

\begin{proof}
This is a pretty neat proof using a smart choice of $\eta$. So let's assume $u_1$ and $u_2$ are weak solutions. Then, when putting $u_1$ and $u_2$ into (\ref{weak}) and subtracting one from the other, we get
\begin{equation}
\label{weak_diff}
\iint_{Q_T}\nabla(\Phi(u_2) - \Phi(u_1)) \cdot \nabla \eta - (u_2 - u_1)\partial_t \eta dxdt = 0.
\end{equation}
Now the inspired move\footnote{Of course, this isn't my proof so that sort of horn-tooting is in good taste.} is choosing
\begin{equation*}
\eta(x,t) = \begin{cases}
		\int_t^T(\Phi(u_2) - \Phi(u_1))ds, \quad &\text{ if } 0 < t < T \\
		0 \quad &\text{ if } t\geq T.
	\end{cases}
\end{equation*}
We then have
\begin{equation*}
\begin{cases}
	\partial_t \eta &= -(\Phi(u_2) - \Phi(u_1) ) \\
	\nabla \eta &= \int_t^T\nabla (\Phi(u_2) - \Phi(u_1))ds,
\end{cases}
\end{equation*}
where the time derivation is valid using the First Fundamentel Theorem of Calculus and that $\Phi(u_i) \in L^1(0,T : W_0^{1,1}(\Omega))$ by the definition of weak soultion. For the validity of the spatial derivative, confer appendix \ref{sec:derivative}.

When putting these derivatives into the above equality we end up with
\begin{equation}
\label{unique}
\begin{aligned}
	&\iint_{Q_T}(\Phi(u_2) - \Phi(u_1))(u_2 - u_1)dxdt \\ 
	&+ \iint_{Q_T}\nabla(\Phi(u_2(t)) - \Phi(u_1(t))) \cdot \left(\int_t^T 					\nabla(\Phi(u_2(s)) - \Phi(u_1(s))ds\right)dtdx = 0.
\end{aligned}
\end{equation}
Let's take these in order:
\begin{enumerate}[i)]
	\item In the first integral we notice that $u_2 > u_1 \Rightarrow \Phi(u_2) \geq \Phi(u_1)$ and vice versa, so this integral is nonnegative.
	
	\item For the second integral it is helpful to first only consider integration of the temporal dimensions, leaving us with
	\begin{align*}
	&\int_0^T\int_t^T \nabla(\Phi(u_2(t)) - \Phi(u_1(t)))\cdot \nabla(\Phi(u_2(s))-\Phi(u_1(s)))dsdt \\
	&= \int_0^T\int_0^s\nabla(\Phi(u_2(t)) - \Phi(u_1(t)))\cdot \nabla(\Phi(u_2(s))-\Phi(u_1(s)))dtds.
	\end{align*}
	Due to the symmetry about the line $s=t$, we have
	\begin{align*}
	&\int_0^T\int_0^s\nabla(\Phi(u_2(t)) - \Phi(u_1(t)))\cdot \nabla(\Phi(u_2(s))-\Phi(u_1(s)))dtds \\
	&= \frac{1}{2}\int_0^T \int_0^T \nabla(\Phi(u_2(t)) - \Phi(u_1(t)))\cdot \nabla(\Phi(u_2(s))-\Phi(u_1(s)))dtds \\
	&= \frac{1}{2}\left( \int_0^T \nabla(\Phi(u_2)-\Phi(u_1))dt\right)^2,
	\end{align*}
	which is also nonnegative, and therefore the second integral in (\ref{unique}) must also be nonnegative.
\end{enumerate}
From this we may conclude that both integral terms in (\ref{unique}) are zero, which in turn implies, by the first integral that $(\Phi(u_2)-\Phi(u_1))(u_2 - u_1) = 0$ a.e. in $Q_T$. Actually, we have that $\Phi(u_2) - \Phi(u_1) = 0$ a.e., because wherever $u_2 = u_1$ it is trivially true, and wherever $ u_2 \neq u_1$ it holds from the first integral being zero. Thus, $ \nabla (\Phi(u_2) - \Phi(u_1) ) = 0$ a.e. and putting this into (\ref{weak_diff}) we get that
\begin{equation*}
\iint_{Q_T}(u_2 - u_1)\partial_t \eta dx dt = 0.
\end{equation*}
This should hold for any test function $\eta$, and so the weak derivative of $u_2 - u_1$ with respect to time is zero. Because $u_2$ and $u_1$ have identical initial condition $u_0$ we conclude that $u_2 = u_1$.
\end{proof}

%%%%%%%%%%%%%%%%%%%%%%%%%%%%%%%%%%%%%%%%%%
%			EXISTENCE
%%%%%%%%%%%%%%%%%%%%%%%%%%%%%%%%%%%%%%%%%%
\subsection{Existence}
As the assumptions in theorem \ref{thm:uniqueness} may hint at, some additional assumptions on $u$ and $\Phi(u)$ needs to be put to ensure that the homogeneous Dirichlet problem is well posed in the weak setting. There are some alternatives, and of course, any further analysis will be informed what sort of additional condition we want a weak solution $u$ to fulfill. The one we will be using here is guided by the notion that in a physical system the energy should be controlled, and is therefore called a weak energy solution. 

Recall our definition of $\Psi$ in (\ref{phi_primitive}) as the primitive of $\Phi$,
and we define the function space $L_\Psi(\Omega) = \{ u\in L^1(\Omega): \Psi(u)\in L^1(\Omega)\}$.
With this in mind we can define energy solutions in the following way:
\begin{mydef}[Energy solution]
We say that $u$ is a \textbf{weak energy solution} if it's a weak solution, and $u_0 \in L_\Psi(\Omega)$, and the energy inequality
\begin{equation}
\label{energy_ineq}
\iint_{Q_T} |\nabla \Phi(u)|^2 dxdt + \int_\Omega \Psi(u(x,T))dx \leq \int_\Omega \Psi(u_0)dx + \iint_{Q_T}f\Phi(u)dxdt
\end{equation}
holds.
\end{mydef}
\setcounter{obs}{0}
\begin{obs}
If $f\in L^2(Q_T)$ we can do the same trick on the right hand side as with the energy identity in (\ref{spatial_derivative_bound}) to get that $\Phi(u) \in L^2(0,T: H_0^1(\Omega))$.
\end{obs}

Now we jump in with trying to prove existence of signed, weak energy solutions. Following closely the proof provided in \citep{vazquez2007porous}. We will first prove existence of weak energy solutions under the assumption that $\Phi$ is strictly increasing, and then extend the result to the case of $\Phi$ being nondecreasing. Another useful addition to the proof will be that we will also incorporate a $L^1$-contractivity result.

The proof will use an approximate problem which admits a unique solution from theorem \ref{thm:classical_existence}, and then employ the Rellich compactness argument (cf. appendix \ref{sec:compactness}) to get a limit which hopefully will be a weak energy solution.

In the following we will assume $u_0 \in L_\Psi(\Omega)$ and $f\in L^p(\Omega)$ with $p = \frac{2d}{d+2}$ if $d \geq 3$, or for some $p>1$ if $d=1,2$\footnote{This choice of value for $p$, I can only say is a mystery for me at this moment, but it seems to be related to the Sobolev conjugate. Let's just stay curious and excited for now, shall we?}.

The main theorem reads:
\begin{theorem}
\label{thm:weak_existence}
Assume $u_0 \in L_\Psi(\Omega)$ and $f \in L^p (\Omega)$. Then the homogeneous, general porous medium equation (\ref{HDP}) has a weak solution defined in an infinite time interval, $T=\infty$. We have $u \in L^\infty(0,T:L_\Psi(\Omega))$ and $\Phi(u) \in L^2(0,T:H_0^1(\Omega))$, and the energy inequality (\ref{energy_ineq}) holds.
\end{theorem}

We will not prove this in one big chunk, but rather split it off into smaller lemmas, and as already stated, some attention will also be paid to the $L^1$-contractivity.
\subsubsection{Preliminaries}
To set ourselves up in a position where we can make use of theorem \ref{thm:classical_existence}, or rather corollary \ref{cor:GPME_existence}, we need to approximate $\Phi$, $u_0$, and $f$.

We approximate $\Phi$ by the smooth approximations. To ensure suffient smoothness we define
\begin{equation*}
\Phi_n'(s) = \left( \Phi' + \frac{1}{n}\right)*\varphi_{\frac{1}{n}},
\end{equation*}
where $\varphi_{\frac{1}{n}}(s) = n\varphi\left( \frac{s}{n} \right)$ and $\varphi$ is the standard mollifier. Then, we can approximate $\Phi$ as
\begin{align*}
\Phi_n(u) &= \int_0^u \int_\mathbb{R}\left( \Phi'(s-y) + \frac{1}{n}\right)\varphi_{\frac{1}{n}}(y)dyds \\
&= \int_\mathbb{R}\varphi_{\frac{1}{n}}(y) \int_0^u \left( \Phi'(s-y) + \frac{1}{n}\right)dsdy \\
&= \int_\mathbb{R}\varphi_{\frac{1}{n}}(y) \left( \Phi(u-y) - \Phi(-y) + \frac{u}{n}\right)dy \\
&= \Phi * \varphi_{\frac{1}{n}}(u) - \Phi * \varphi_{\frac{1}{n}}(0) + \frac{u}{n}.
\end{align*}
Further discussion of $\Phi$ and its sequence of approximations has been relegated to appendix \ref{app:Phi} for sake of continuity.




We also let $u_{0,n}\in C^\infty_c(\Omega)$ be a smooth approximation of $u_0$, and converging in $L_\Psi(\Omega)$. We also ask that $|u_{0,n}| \leq $min$\{n,|u_0|\}$.

Apparently, also according to \citep{vazquez2007porous}, we also approximate $f$ by a sequnce of functions $f_n$ that converges to $f$ in $L^p(Q_T)$ for all\footnote{Not sure if \textit{sic}...} $p<\infty$, and is so that $|f_n| \leq |f|$ for all $n$.

\subsubsection{$u_0$, $f$ and $\partial_t f$ bounded}
Our first lemma then reads
\begin{lemma}
\label{lem:existence_part1}
Theorem \ref{thm:weak_existence} holds with the additional assumption that $u_0$, $f$, and $\partial_t f$ are bounded, $\partial \Omega \in C^{2+\alpha}$, and $\Phi$ is locally lipschitz continuous.
\end{lemma}

\begin{proof}

\begin{description}
	\item[Setting up an approximate problem:] We fix $T<\infty$ and consider the problem
		\begin{equation}
		\label{approximate_problem}
			\begin{cases}
			\partial_t u_n &= \Delta\Phi_n(u_n) + f_n, \quad \text{ in } Q_T \\
			u_n(x,0) &= u_{0,n}(x),	\quad \text{ on } \overline{\Omega} \\
			u_n(x,t) &= 0, \quad \text{ on }  \partial \Omega \times (0,T).
			\end{cases}
		\end{equation}			
	By virtue of theorem \ref{thm:classical_existence}, this approximate problem has a unique solution $u_n \in C^{2,1}(\overline{Q}_T)$. In addition we can from the maximum principle see that
	\begin{align*}
	||u_n||_{L^\infty(Q_T)} &\leq ||u_{0,n}||_{L^\infty(\Omega)} + T||f_n||_{L^\infty(Q_T)} \\
	&\leq ||u_0||_{L^\infty(\Omega)} + T||f||_{L^\infty(Q_T)} = C.
	\end{align*}
	A bound independent of $n$.
	
	\item[Control of the spatial derivative:] Now we define $w_n = \Phi_n(u_n)$, and do exactly as when deriving (\ref{spatial_derivative_bound}) to get 
	\begin{equation*}
	\int_\Omega \Psi_n(u_n(T))dx + \iint_{Q_T}|\nabla w_n|^2dxdt = \int_\Omega \Psi_n(u_{0,n})dx + \iint_{Q_T}f_n w_n dxdt.
	\end{equation*}
	Considering the second term on the right hand side we have
	\begin{equation*}
	\iint_{Q_T}f_n w_n dxdt \leq C\iint_{Q_T}f_n^2 dxdt + \frac{1}{2}\iint_{Q_T}|\nabla w_n|^2dxdt,
	\end{equation*}
	after exactly the same argument as we did with (\ref{using_Poincare}). After putting it back
	\begin{equation}
	\label{exist_proof_spatial}
	\int_\Omega \Psi_n(u_n(T))dx + \frac{1}{2}\iint_{Q_T}|\nabla w_n|^2 dxdt \leq \int_\Omega \Psi_n(u_{0,n})dx + C\iint_{Q_T}f_n^2 dxdt,
	\end{equation}
	and so we're getting close to a uniform bound on $\nabla w_n$ in $L^2(Q_T)$. We notice that
	\begin{equation*}
	\iint_{Q_T}f_n^2dxdt \leq \iint_{Q_T} f^2 dxdt,
	\end{equation*}
	by our assumption on the approximation.
	
	For the $\Psi$-term on the right hand side we start by recalling that we define
	\begin{equation*}
	\Psi_n(u) = \int_0^{u}\Phi_n(s)ds,
	\end{equation*}
	from which it is to see that $\Psi_n(u) \leq 0$ for every $u \in \mathbb{R}$. It is also very much worthwhile to see that because $\Phi_n$ is monotone for every $n$ we have the estimate
	\begin{equation*}
	|\Psi_n(u)| \leq |\Phi_n(u)u|.
	\end{equation*}
	This leads to
	\begin{equation*}
	\int_\Omega \Psi_n(u_{0,n})dx \leq \int_\Omega |\Phi_n(u_{0,n})u_{0,n}|dx	
	\end{equation*}
	Using now that $|u_{0,n}| \leq u_0$ and that $\Phi_n$ is monotone we can remove some of the dependence on $n$ as
	\begin{equation*}
	\int_\Omega \Psi_n(u_{0,n})dx \leq \int_\Omega |\Phi_n(u_{0})u_{0}|dx.
	\end{equation*}
	Now we do as in appendix \ref{app:Phi} to get a bound in $\Phi_n$ in terms of $\Phi$ and $u_0$. A crude estimate will be sufficient, and we have
	\begin{align*}
	|\Phi_n(u) - \Phi(u)| &= \left|\frac{u}{n} + \int_\mathbb{R}\left( \Phi(u-y) - \Phi(u) - \Phi(y)\right)\varphi_{\frac{1}{n}}(y)dy \right| \\
	&\leq \frac{|u|}{n} + \underset{y\in [-1/n, 1/n]}{\mathrm{sup}}|\Phi(u-y) - \Phi(u)| + \mathrm{max}\{|\Phi(-1/n)|, |\Phi(1/n)|\} \\
	&\leq |u| + \underset{y\in [-1, 1]}{\mathrm{sup}}|\Phi(u-y) - \Phi(u)| +\mathrm{max}\{|\Phi(-1)|, |\Phi(1)|\},
	\end{align*}
	where we have used that $\mathrm{supp}(\varphi_{\frac{1}{n}})\subset [-1/n, 1/n]$ as well as that $\Phi$ is monotone. This, when considering $u_0$, we have
	\begin{align*}
	|\Phi_n(u_0)| &\leq |\Phi(u_0)| + |u_0| + \underset{y\in [-1, 1]}{\mathrm{sup}}|\Phi(u_0-y) - \Phi(u_0)| +\mathrm{max}\{|\Phi(-1)|, |\Phi(1)|\} \\
	&= K.
	\end{align*}
	All terms are bounded since $u_0$ is bounded, and so we get the crude, but sufficient estimate
	\begin{equation*}
	\int_\Omega \Psi_n(u_{0,n})dx \leq K ||u_0||_{L^1(\Omega)},
	\end{equation*}
	which is finite from $u_0 \in L^1(\Omega)$. Most importantly, this shows that the left hand side of (\ref{exist_proof_spatial}) is bounded independently of $n$, which immediately implies that $\nabla w_n$ is a bounded sequence in $L^2(Q_T)$.
	
	\item[Control of the temporal derivative:] The derivation of (\ref{temporal_bound}) holds up in this case and we have
	\begin{equation*}
	\label{exist_proof_temporal}
	\iint_{Q_T} \Phi'_n(u_n)|\partial_t u_n|^2 dxdt \leq \iint_{Q_T} \left\{ -\partial_t f_n \Phi_n(u_n) + Cf_n\partial_tf_n \right\} dxdt.
	\end{equation*}
	Just to reiterate, $C$ is here the constant given by Poincaré inequality, and is as such only dependent on $\Omega$. 
	
	Now, since we have assumed $\Phi$ to be locally Lipschitz, $|\Phi_n'(u_n)| \leq K < \infty$ over the relevant values of $u_n$, we notice that $|\partial_t w_n|^2 \leq K|\Phi'_n(u_n) (\partial_t u_n)^2|$. We have assumed both $f$ and $\partial_t f$ to be bounded, so we can put a uniform bound on the right hand side, independent of $n$. We conclude that $\partial_t w_n$ is uniformly bounded in $L^2(Q_T)$.
	
	\item[Using a compactness argument:]From the two previous parts we have that $\{w_n\}_{n \in \mathbb{N}}$ is a bounded sequence in $H^1(Q_T)$, so by theorem \ref{thm:Rellich} it has convergent subsequence $\{w_{n_j}\}_{j \in \mathbb{N}}$ that converges to some $w$ in $L^2(Q_T)$. The energy estimate continues to hold in the limit as
	\begin{equation}
	\iint_{Q_T}|\nabla w|^2 dxdt \leq \int_\Omega \Psi(u_0)dx + \iint_{Q_T}fw dxdt.
	\end{equation}
	Notice that since $w \in L^2(Q_T)$, we can from this inequality conclude that $w \in H^1(Q_T)$.
	
	\item[Convergence of $u_{n_j}$:] Both $\Phi$ and its approximations $\Phi_n$ all admit a continuous inverse, and so
	\begin{equation*}
	u_{n_j} = \Phi_{n_j}^{-1}(w_{n_j}).
	\end{equation*}
	We now simply have by the triangle inequality
	\begin{align*}
	|u_{n_j} - \Phi^{-1}(w)| &= |\Phi_{n_j}^{-1}(w_{n_j}) - \Phi^{-1}(w)| \\
		&\leq |\Phi_{n_j}^{-1}(w_{n_j}) - \Phi^{-1}(w_{n_j})| + |\Phi^{-1}(w_{n_j}) - \Phi^{-1}(w)|.
	\end{align*}
	 We have that $w_{n_j}$ is a bounded sequence, seeing as $u_{n_j}$ is bounded. Hence we can use proposition \ref{prop:inverse_uniform} to make the first term in the above inequality arbitrarily small. The second term is controlled by the observation that since $w_{n_j}$ converges to $w$ in $L^2(Q_T)$, the convergence is also pointwise, and the continuity of $\Phi^{-1}$. That means that
	 \begin{equation*}
	 u_{n_j} \to u = \Phi^{-1}(w)
	 \end{equation*}
	 pointwise. We immediately see that $\Phi(u) = w$.
	 
	 More importantly, since we have the uniform bound $u_{n_j} \leq M$ in $Q_T$ we can use the Dominated Convergence Theorem to state that $u_{n_j} \to u$ in $L^1(Q_T)$.
	
	 
	\item[$u$ is a weak energy solution:] We already know that $u \in L^1(Q_T)$, and $\Phi(u)=w \in L^1(0,T: W^{1,1}(\Omega))$\footnote{Actually, this may not be as easy to see as I've stated here. Need to use yet another estimate to show this.}. So what's left is to show that
	\begin{equation}
	\label{claim_weak_solution}
	\iint_{Q_T}\left\{ \nabla\Phi(u) \cdot \nabla \eta - u\partial_t \eta - f\eta \right\}dxdt - \int_\Omega u_0\eta\big|_{t=0}dx = 0
	\end{equation}
	for any $\eta \in C^1(\overline{Q}_T)$ that vanish on $\partial \Omega \times [0,T)$ and for $t=T$. Seeing as every $u_n$ is a weak solution to the approximate (\ref{approximate_problem}) we can subtract it from the integral above to get
	\begin{align*}
	&\iint_{Q_T}\left\{ \nabla\Phi(u) \cdot \nabla \eta - u\partial_t \eta - f\eta \right\}dxdt - \int_\Omega u_0\eta\big|_{t=0}dx \\
	= &\iint_{Q_T}\left\{ \nabla(\Phi(u)-w_n) \cdot \nabla \eta - (u-u_n)\partial_t \eta - (f-f_n)\eta \right\}dxdt - \int_\Omega (u_0-u_{0,n})\eta\big|_{t=0}dx.
	\end{align*}
	
	All the terms are controlled by the convergences of $w_n$, $u_n$, $u_{0,n}$, as long as the limit is taken along the subsequence for which we know $w_n$ converges. Thus, (\ref{claim_weak_solution}) holds, and $u$ is a weak energy solution.
	
	We have from before that $u\in L^1(Q_T)$ as well as $\Phi(u) \in L^2(0,T: H^1(\Omega))$, so the solution is unique.
	
	
	\item[Considering the $L^1$-contractivity:] Let now $\hat{u}$ be the weak energy solution to some auxiliary problem with intial value $\hat{u}_0$ and loading $\hat{f}$. We also let $\hat{u}_n$ be a sequence of approximations similarly constructed as those for $u$. We go by way of first showing that
	\begin{equation}
	\label{weak_solutions_l1_start}
	\int_\Omega(u(t)-\hat{u}(t))_+dx \leq \int_\Omega (u_0 + \hat{u}_0)_+dx + \int_0^t \int_\Omega (f- \hat{f})_+ dxdt. 
	\end{equation}
	With that in mind it's worth noting that for $\int_\Omega (\cdot)_+dx$ the triangle inequality is valid, so
	\begin{align*}
	\int_\Omega (u(t) - \hat{u}(t))_+dx &\leq \int_\Omega(u(t) - u_n(t))_+dx + \int_\Omega (u_n(t) - \hat{u}_n(t))_+dx \\
	&+ \int_\Omega(\hat{u}_n - \hat{u})_+dx.
	\end{align*}
	The first and last term we can make arbitrarily small, and for the middle term we use the $L^1$-contractivity for smooth solutions. Then, for every $\epsilon > 0$
	\begin{equation*}
	\int_\Omega (u(t) - \hat{u}(t))_+dx < \epsilon + \int_\Omega(u_{0,n} + \hat{u}_{0,n})_+ dx + \int_0^t \int_\Omega (f_n - \hat{f}_n)_+dxdt,
	\end{equation*}
	for sufficiently large $n$. When using the triangle inequality again
	\begin{align*}
	\int_\Omega (u(t) - \hat{u}(t))_+dx &< \epsilon + \int_\Omega(u_{0,n} - u_0)_+dx + \int_\Omega(u_0 - \hat{u}_0)_+dx \\
	&+ \int_\Omega (\hat{u}_0 - \hat{u}_{0,n})_+dx + \int_0^t \int_\Omega(f_n - f)_+dxdt\\
	&+ \int_0^t(f-\hat{f})_+dxdt + \int_0^t\int_\Omega (\hat{f}-\hat{f}_n)_+ dxdt.
	\end{align*}	
	All the term involving $n$ in the integrand can now be made arbitrarily small, so for sufficiently large $n$
	\begin{equation*}
	\int_\Omega (u(t) - \hat{u}(t))_+dx < 2\epsilon +\int_\Omega(u_0 - \hat{u}_0)_+dx + \int_0^t \int_\Omega (f-\hat{f})dxdt.
	\end{equation*}					
	Since this should hold for every $\epsilon >0 $, we must have that (\ref{weak_solutions_l1_start}) is satisfied.
	
	From (\ref{weak_solutions_l1_start}), we do as for (\ref{l1_contractivity}), and get
	\begin{equation*}
	||u(t) - \hat{u}(t)||_{L^1(\Omega)} \leq ||u_0 - \hat{u}_0||_{L^1(\Omega)} + ||f-\hat{f}||_{L^1(Q_t)}
	\end{equation*}
\end{description}


\end{proof}


\subsubsection{$\Phi$ no longer locally Lipschitz}
We proceed by generalising the result of \ref{thm:weak_existence} to the case when $\Phi$ is no longer locally Lipschitz.

\begin{lemma}
Theorem \ref{thm:weak_existence} holds with the additional assumption that $u_0$, $f$ and $\partial_t f$ are bounded.
\end{lemma}

\begin{proof}
We begin this proof in the exact same manner as with the proof of lemma \ref{lem:existence_part1}, by setting up an approximate problem using $\Phi_n$, $u_{0,n}$ and $f_n$ as smooth approximations which yields a sequence of classical solutions $u_n$. We again define $w_n = \Phi_n(u_n)$.

Looking back at the proof for lemma \ref{lem:existence_part1} the only part where we actually needed $\Phi$ to be locally Lipschitz was in controlling the temporal derivative of $\Phi_n(u_n)$ in $L^2(Q_T)$. This means we can in this case no longer say that $w_n$ is a bounded sequence in $H^1(Q_T)$.

The strategy for showing that $u_n$ converges to a weak solution will be similar to that of lemma \ref{lem:existence_part1}, only now we will use the Rellich-Kondrachov Compactness Theorem on another sequence than $w_n$. I.e. we need a bounded sequence in $H^1(Q_T)$. To that end we define for $u \in \mathbb{R}$
\begin{equation}
Z_n(u) = \int_0^u \mathrm{min}\{1, \Phi_n'(s)\} ds,
\end{equation}
which is easily seen to be globally Lipschitz. In addition $Z_n'(u) \leq 1$ and $Z_n'(u) \leq \Phi_n(u)$.

We then go on to define
\begin{equation}
z_n(x,t) = Z_n(u_n(x,t)).
\end{equation}

See now that
\begin{align*}
\nabla z_n &= Z_n'(u_n)\nabla u_n \\
	&\leq \Phi_n'(u_n)\nabla u_n \\
	&= \nabla \Phi_n(u_n) \\
	&= \nabla w_n,
\end{align*}
which means that $|\nabla z_n|^2 \leq |\nabla w_n|^2$ in all of $Q_T$. As a trivial consequence of this
\begin{equation*}
\iint_{Q_T}|\nabla z_n|^2dxdt \leq \iint_{Q_T}|\nabla w_n|^2dxdt.
\end{equation*}

Seeing as estimate (\ref{exist_proof_spatial}), we can argue exactly as we did in the previous proof to say that $\nabla z_n$ is uniformly bounded in $L^2(Q_T)$.
Similarly we have the two inequalities
\begin{equation*}
|\partial_t z_n | \leq |\partial_t u_n|, \quad \text{ and} \quad |\partial_t z_n| \leq |\Phi_n'(u_n) \partial_t u_n|,
\end{equation*}
and as a consequence
\begin{equation*}
\iint_{Q_T}|\partial_t z_n|^2 dxdt \leq \iint_{Q_T} |\Phi_n'(u_n)|\cdot |\partial_t u_n|^2dxdt.
\end{equation*}
The estimate (\ref{exist_proof_temporal}) holds in this case as well, and by the assumptions on $f$ and its approximations we also have a uniform bound of $\partial_t z_n$ in $L^2(Q_T)$. Hence $z_n$ is a bounded sequence in $H^1(Q_T)$, and we can utilize the Rellich-Kondrachov Compactness Theorem (theorem \ref{Rellich}) again to conclude that there is a subsequence $\{z_{n_j}\}$ converging to some $z$ in $L^2(Q_T)$.

Now we obviously have $u_{n_j} = Z_{n_j}^{-1}(z_{n_j})$, and so
\begin{align*}
|u_{n_j} - Z^{-1}(z)| &= |Z_{n_j}^{-1}(z_{n_j}) - Z^{-1}(z)| \\
&\leq |Z_{n_j}^{-1}(z_{n_j}) - Z^{-1}(z_{n_j})| + |Z^{-1}(z_{n_j})- Z^{-1}(z)|.
\end{align*}
The sequence $(z_{n_j})$ converges pointwise to $z$ as well as being uniformly bounded 
since $|z_{n_j}| \leq |u_{n_j}| \leq M$. And so the first term on the right hand side in the above inequality is controlled by $Z_n^{-1}$ converging uniformly to $Z^{-1}$ on compact sets, as shown in proposition \ref{prop:Zinv_uniform_convergence}. The second term tends to zero since $z_{n_j}$ converges pointwise to $z$ and $Z^{-1}$ is continuous.

We define $u=Z^{-1}(z)$ as the pointwise limit of $u_{n_j}$, and note as in the previous proof that $u_{n_j}$ also converges to $u$ in $L^1(Q_T)$.
\end{proof}






%%%%%%%%%%%%%%%%%%%%%%%%%%%%%%%%%%%%%%%%%%%%%%%%%%%%%%%
%				BIBLIOGRAPHY
%%%%%%%%%%%%%%%%%%%%%%%%%%%%%%%%%%%%%%%%%%%%%%%%%%%%%%%
\newpage
\bibliographystyle{plainnat}
\bibliography{project_bibliography.bib}




%%%%%%%%%%%%%%%%%%%%%%%%%%%%%%%%%%%%%%%%%%%%%%%%%%%%%%%
%				APPENDICES
%%%%%%%%%%%%%%%%%%%%%%%%%%%%%%%%%%%%%%%%%%%%%%%%%%%%%%%
\newpage
\begin{appendix}
\section{Passing derivative inside the integral}
\label{sec:derivative}

We are now concern with whether
\begin{equation}
\label{diff_int}
	\nabla \int_t^T \varphi(x,s)ds = \int_t^T \nabla\varphi(x,s)ds
\end{equation}
holds or not for $\varphi \in L^2(0,T:H_0^1 (\Omega))$. 
For simplicity we now only consider one spatial dimension.

To begin with, assume that $\psi \in C^{\infty}_c(\Omega)$ and $\eta \in L^2(0,T: H_0^1(\Omega))\cap C^\infty_c(Q_T)$, then we have that
\begin{align*}
\int_{\Omega}\psi \partial_x\left(\int_t^T \eta(x,s) ds\right)dx &= \int_{\Omega}\psi \lim_{h \to 0}\left(\int_t^T \frac{\eta(x+h,s)-\eta(x,s)}{h} ds\right)dx\\
&= \int_{\Omega}\psi \lim_{h \to 0}\left(\int_t^T \eta_h(x,s) ds\right)dx.
\end{align*}

Assuming that $\partial_x \eta(x,s)$ is bounded, we can use the mean value theorem to state that $|\eta_h| \leq  ||\partial_x \eta||_\infty$. We also have that $\eta_h(x,s) \to \partial_x \eta(x,s)$ pointwise, and so we're in a position to use the Dominated Convergence theorem to conclude that
\begin{equation}
\label{results_diff_first}
\int_{\Omega}\psi \partial_x\left(\int_t^T \eta(x,s) ds\right)dx = \int_{\Omega}\psi \left(\int_t^T \partial_x\eta(x,s) ds\right)dx.
\end{equation}

To extend the result to $L^2(0,T:H_0^1 (\Omega))$ we use that $C_c^\infty$ is dense in this space. So let $\{\psi_n\},\{\varphi_n\}$ be two sequence converging to $\psi$ and $\varphi$ in $L^2(\Omega)$ and $L^2(0,T:H_0^1(\Omega))$ respectively. Considering the left hand side of (\ref{results_diff_first}), we get
\begin{align*}
&\left|\int_{\Omega}\psi \partial_x \left( \int_t^T\varphi ds \right)dx - \int_{\Omega}\psi_n \partial_x \left( \int_t^T \varphi_n ds \right) dx \right| \\
&\leq  \left| \int_{\Omega} \psi \partial_x \left(\int_t^T \varphi - \varphi_n ds\right)dx \right| + \left| \int_{\Omega}(\psi - \psi_n)\partial_x\left(\int_t^T \varphi_n ds\right) dx \right| \\
&\leq ||\psi||_{L^2(\Omega)}|(\varphi -\varphi_n)|_{L^2(H_0^1)} + ||\psi-\psi_n||_{L^2}| \varphi_n |_{L^2(H_0^1)} \\
&\leq ||\psi||_{L^2(\Omega)}||(\varphi -\varphi_n)||_{L^2(H_0^1)} + ||\psi-\psi_n||_{L^2}|| \varphi_n ||_{L^2(H_0^1)} \\
&\to 0,
\end{align*}
after using the triangle- and Cauchy-Schwarz' inequality as well as the assumptions on convergence.

An exactly similar approach can be done to the right hand side, and so passing the limit in (\ref{results_diff_first}) we get
\begin{equation}
\label{result_diff_final}
\int_\Omega \psi \partial_x\left(\int_t^T \varphi(x,s)ds\right)dx = \int_\Omega \psi \left( \int_t^T \partial_x \varphi(x,s)ds \right) dx.
\end{equation}

To get the more general (\ref{diff_int}), we simply apply this result for each individual spatial dimension while keeping the others fixed.



%%%%%%%%%%%%%%%%%%%%%%%%%%%%%%%%%%%%%%%%%%
%		COMPACTNESS
%%%%%%%%%%%%%%%%%%%%%%%%%%%%%%%%%%%%%%%%%%
\newpage
\section{Compactness argument}
\label{sec:compactness}
Here I will simply state the Rellich-Kondrachov Compactness theorem as it is presented in \citep{evans}, mostly for ease of reference. For more details, please cf. Chapter 5 ibid.

\begin{mydef}
Let $X$ and $Y$ be Banach spaces with $X \subset Y$. We say that $X$ is \textbf{compactly embedded} in $Y$, written
\begin{equation*}
X \subset \subset Y,
\end{equation*}
provided 
\begin{enumerate}[i)]
	\item $||u||_Y \leq ||u||_X$ for every $u \in X$, for some constant $C$, and

	\item each bounded sequence in $X$ is precompact in $Y$.
\end{enumerate}
\end{mydef}

With this definition in mind we have the following theorem:
\begin{theorem}[Rellich-Kondrachov Compactness Theorem]
\label{thm:Rellich}
Assume $U$ is an open, bounded subset of $\mathbb{R}^n$ and $\partial U$ is $C^1$.
Suppose $1\leq p < n$. Then
\begin{equation}
\label{Rellich}
W^{1,p}(U) \subset \subset L^q(U)
\end{equation}
for each $1\leq q < p^*$, where $p^* = \frac{np}{n-p}$ is the Sobolev conjugate of $p$.
\end{theorem}
\setcounter{rem}{0}
\begin{rem}
If $p=2$, the we have
\begin{equation*}
H^1(U) \subset \subset L^q(U)
\end{equation*}
for each $1\leq q < p^*$.
\end{rem}
\begin{rem}
Since $p^* > p$ the result is valid for $q=p$. This actually holds for all $1\leq p \leq \infty$, even when $p > n$.
\end{rem}
\begin{rem}
We also have 
\begin{equation*}
W^{1,p}_0(U) \subset \subset L^p(U)
\end{equation*}
even when the assumption that $\partial U$ is $C^1$ is lifted.
\end{rem}


%%%%%%%%%%%%%%%%%%%%%%%%%%%%%%%%%%%%%%%%%%%%%%%%%%%%%%%%%%%%%%
%			PROPERTIES OF THE APPROXIMATIONS:
%%%%%%%%%%%%%%%%%%%%%%%%%%%%%%%%%%%%%%%%%%%%%%%%%%%%%%%%%%%%%%
\newpage
\section{Properties of the approximate nonlinearities}
\subsection{Properties of $\Phi$ and its approximations}
\label{app:Phi}
Let's discuss some of the properties of the sequence of approximations to the nonlinearity $\Phi$. To reiterate we defined
\begin{equation*}
\Phi_n(u) = (\Phi * \varphi_{\frac{1}{n}})(u) - (\Phi * \varphi_{\frac{1}{n}})(0) + \frac{u}{n}.
\end{equation*}
It is obvious that $\Phi_n$ is infintely differentiable for each $n$, as well as $\Phi_n'(u) \geq \frac{1}{n}$. In addition the construction is so that $\Phi_n(0) = 0$.

The first property we'll consider is summarised in the following proposition.
\begin{proposition}
If $\Phi$ is continuous, $\Phi_n$ converges to $\Phi$ uniformly on compact sets.
\end{proposition}
\begin{proof}
This follows almost immediately from part iii) of theorem 7 of appendix C in \citep{evans}, which states that if $\Phi$ is continuous, then $\Phi * \varphi_{\frac{1}{n}}$ converges uniformly to $\Phi$ on compact sets. So let $K\subset \mathbb{R}$ be a compact set, and fix $\epsilon < 0$. Then there is an $N_1 \in \mathbb{N}$ so that
\begin{equation*}
\underset{x\in K}{\mathrm{sup}}|\Phi * \varphi_{\frac{1}{n}}(x) - \Phi(x)| < \frac{\epsilon}{3},
\end{equation*} 
whenever $n \geq N_1$. Similarly we have an $N_2 \in \mathbb{N}$, so that 
\begin{equation*}
|\Phi * \varphi_{\frac{1}{n}}(0) - \Phi(0)| < \frac{\epsilon}{3},
\end{equation*}
for every $n \geq N_2$. Lastly, since $K$ is compact, there is an $N_3 \in \mathbb{N}$ so that
\begin{equation*}
\underset{x\in K}{\mathrm{sup}}\frac{|x|}{n} < \frac{\epsilon}{3},
\end{equation*}
whenever $n \geq N_3$.

Now we set $N = \mathrm{max}\{N_1,N_2,N_3\}$, and see that
\begin{align*}
\underset{x\in K}{\mathrm{sup}}|\Phi_n(x) - \Phi(x)| &\leq  \underset{x\in K}{\mathrm{sup}}|\Phi * \varphi_{\frac{1}{n}}(x) - \Phi(x)| + |\Phi * \varphi_{\frac{1}{n}}(0) - \Phi(0)| + \underset{x\in K}{\mathrm{sup}}\frac{|x|}{n} \\
&< \frac{\epsilon}{3} + \frac{\epsilon}{3} + \frac{\epsilon}{3} \\
&= \epsilon,
\end{align*}
for every $n \geq N$, which proves the proposition.
\end{proof}

In the remainder of this section we'll assume that $\Phi$ is continuous and strictly increasing, and that $\Phi( \pm \infty) = \pm \infty$. Then it is obvious that both $\Phi$ is both one-to-one and onto, and so has an inverse $\Phi^{-1}$.

\begin{proposition}
$\Phi^{-1}$ is strictly increasing, and continuous.
\end{proposition}
\begin{proof}
Let's first prove that it's strictly increasing, which is easy. Take $y_1 < y_2$, and let $x_1$ and $x_2$ be so that $\Phi(x_i) = y_i$ for $i=1,2$. Then, because $\Phi$ is strictly increasing, we have $x_1 < x_2$. Which is the same as saying that
\begin{equation*}
y_1 < y_2 \quad \Rightarrow \quad \Phi^{-1}(y_1) < \Phi^{-1}(y_2),
\end{equation*}
i.e. $\Phi^{-1}$ is stricly increasing.

Moving along, let $y \in \mathbb{R}$ and $x$ be so that $\Phi(x) = y$. Take any $\epsilon > 0$, and consider the interval $(x-\epsilon, x + \epsilon)$. We have that $(\Phi^{-1}(x-\epsilon), \Phi^{-1}(x+\epsilon))$ is non-empty because $\Phi^{-1}$ is strictly increasing, and $y$ is an element. Take now $\delta = \mathrm{min}\{ |y-\Phi^{-1}(x\pm\epsilon)|\} >0$, then for any $\tilde{y} \in (y-\delta, y + \delta)$ we have
\begin{equation*}
\tilde{y} > y-\delta \Rightarrow \Phi^{-1}(\tilde{y}) > \Phi^{-1}(y-\delta) \geq x-\epsilon
\end{equation*}
and
\begin{equation*}
\tilde{y} < y+\delta \Rightarrow \Phi^{-1}(\tilde{y}) < \Phi^{-1}(y+\delta) \leq x+\epsilon,
\end{equation*}
by using that $\Phi^{-1}$ is strictly increasing, as well as $\Phi(x-\epsilon) \leq y-\delta$ and $\Phi(x+\epsilon) \geq y + \delta$. This shows that $\Phi^{-1}$ is continuous.
\end{proof}

This result is also valid for $\Phi_n$ for every $n \in \mathbb{N}$, beacause they are all continuous and stricly increasing. We then have the following helpful result.

\begin{proposition}
$\Phi_n^{-1}$ converges pointwise to $\Phi^{-1}$.
\end{proposition}

\begin{proof}
We need to consider $|\Phi_n^{-1}(y) - \Phi^{-1}(y)| = |x_n - x|$. Define $y_n = \Phi(x_n)$, then if we can show that $y_n \to y$ we can use the coninuity of $\Phi^{-1}$  to show that $x_n = \Phi^{-1}(y_n) \to \Phi^{-1}(y) = x$, and we would be done. It is now useful to see that
\begin{align*}
|y-y_n| &= |\Phi(x) - \Phi(x_n)| \\
	&= |\Phi_n(x_n) - \Phi(x_n)|
\end{align*}
and hence we are in a nice position to use that $\Phi_n$ converges uniformly on compact sets, if we can show that $x_n$ is a bounded sequence. To that end we let $x_1<x$ be so that
\begin{equation*}
\Phi(x_1) = \Phi(x) -1
\end{equation*}
and similarly $x_2 > x$ by
\begin{equation*}
\Phi(x_2) = \Phi(x) +1.
\end{equation*}
Since $\Phi_n$ converges uniformly on $[x_1, x_2]$, there is an $N \in \mathbb{N}$ so that
\begin{equation*}
|\Phi_n(x) - \Phi(x)| < 1
\end{equation*} 
for every $x \in [x_1, x_2]$ and $n \geq N$. So take any $n \geq N$ and see that
\begin{equation*}
\Phi_n(x_2) \geq \Phi(x_2) - 1 = \Phi(x),
\end{equation*}
which by the strict monotonicity of $\Phi_n$ means that $x_n \leq x_2$. A similar argument yields $x_n \geq x_1$, and so we have that for every $n \geq N$ that $x_n \in [x_1, x_2]$. This means that $x_n$ is a bounded sequence, and hence we can use the unform convergence of $\Phi_n$ on compact sets to conclude that $y_n$ converges to $y$. Finally we have that
\begin{equation*}
x_n = \Phi_n^{-1} (y) = \Phi^{-1}(y_n) \overset{y_n\to y}{\to} \Phi^{-1}(y) = x
\end{equation*}
\end{proof}

We can go even a step further with the last result of this section. It will require a bit more involved argument, but will operate on some familiar ideas from our latest proof.

\begin{proposition}
\label{prop:inverse_uniform}
$\Phi_n^{-1}$ converges to $\Phi^{-1}$ uniformly on compact sets.
\end{proposition}

\begin{proof}
Let $K \subset \mathbb{R}$ be a compact set, and fix $\epsilon > 0$. Our aim will be to establish that there is $N \in \mathbb{N}$ so that 
\begin{equation*}
|x-x_n| = |\Phi^{-1}(y) - \Phi_n^{-1}(y)| < \epsilon
\end{equation*}
for every $y \in K$ and $n \geq N$. 

Define $\tilde{K} = \Phi^{-1}(K)$ which is compact since $\Phi^{-1}$ is continuous. We will want to consider
\begin{equation*}
\Phi(x+\epsilon) - \Phi(x) > 0
\end{equation*}
for $x \in \tilde{K}$, so let $L$ be a Lipschitz constant for $\Phi$ over $\tilde{K}+[0,\epsilon]$. Our first claim is that
\begin{equation*}
\mathrm{inf}_{x \in \tilde{K}} \Phi(x+\epsilon) - \Phi(x) > 0.
\end{equation*}
It is obvious that the infimum is nonnegative, so we need to show that it is not zero. Assume to the contrary that for every $n \in \mathbb{N}$ there is an $x_n$ so that
\begin{equation*}
0 < \Phi(x_n + \epsilon) - \Phi(x_n) < \frac{1}{n}
\end{equation*}
which yields a sequence $(x_n)_n$ in $\tilde{K}$. Since $\tilde{K}$ is compact we have a subsequence $(x_{n_j})_j$ that converges to some $ \tilde{x} \in \tilde{K}$. But then
\begin{align*}
|\Phi(\tilde{x} + \epsilon) - \Phi(\tilde{x})| & \leq |\Phi(\tilde{x}+\epsilon) - \Phi(x_{n_j} + \epsilon)| + |\Phi(x_{n_j}+\epsilon) - \Phi(x_{n_j})| \\
&+ |\Phi(x_{n_j}) - \Phi(\tilde{x})| \\
&\leq 2L|\tilde{x}-x_{n_j}| + \frac{1}{n_j},
\end{align*}
Which can be made arbitrarily small. Hence, $\Phi(\tilde{x}+\epsilon) - \Phi(\tilde{x}) = 0$, which is in stark contrast to $\Phi$ being strictly increasing. This proves our claim and we can with some confidence say that there is a $\delta_+>0$ so that $\Phi(x+\epsilon) - \Phi(x) \geq \delta_+$ for every $x \in \tilde{K}$.

We can do quite a similar argument on
\begin{equation*}
\Phi(x) - \Phi(x-\epsilon) > 0,
\end{equation*}
to establish a $\delta_- > 0$ so that $\Phi(x) - \Phi(x-\epsilon) \geq \delta_-$ for every $x\in \tilde{K}$.

Let now $\delta = \mathrm{min}\{\delta_+, \delta_- \}>0$, and there is by the uniform convergence of $\Phi_n$ an $N \in \mathbb{N}$ so that
\begin{equation*}
|\Phi_n(x) - \Phi(x)| < \delta,
\end{equation*}
for every $x \in \tilde{K}+[\epsilon, \epsilon]$ and $n \geq N$. Take now any $y \in K$ and let $x$ and $x_n$ be so that
\begin{equation*}
\Phi(x) = \Phi_n(x_n) = y.
\end{equation*}
See now that
\begin{align*}
\Phi_n(x+\epsilon) &> \Phi(x+\epsilon) -\delta \\
&\geq    \Phi(x+\epsilon) -\delta_+ \\
&\geq \Phi(x+\epsilon) - (\Phi(x+\epsilon) - \Phi(x)) \\
&= \Phi(x).
\end{align*}
So we have $\Phi_n(x+\epsilon) < y$, which implies that $x_n < x+\epsilon$ since $\Phi_n$ is strictly increasing. Quite similarly we can establish that
\begin{align*}
\Phi_n(x-\epsilon) &< \Phi(x-\epsilon) + \delta \\
&\leq \Phi(x-\epsilon) + \delta_- \\
&\leq \Phi(x-\epsilon) + (\Phi(x) - \Phi(x-\epsilon)) \\
&= \Phi(x),
\end{align*}
which implies that $x_n > x-\epsilon$. In toto we have that whenever $n\geq N$
\begin{equation*}
|\Phi^{-1}(y) - \Phi_n^{-1}(y)| = |x - x_n| < \epsilon
\end{equation*}
for every $y \in K$, which is exactly what we wanted to prove.
\end{proof}

Looking back at this proof, the only things we needed were that $\Phi$ is strictly increasing and continuous, and is locally Lipschitz, and that $\Phi_n$ converges to $\Phi$ uniformly on compact sets. So we can generalise our result in the following lemma.

\begin{lemma}
\label{lem:Inverse_uniform_general}
Let $f: \mathbb{R} \to \mathbb{R}$ be a strictly monotone, continuous function, and let $\{f_n\}$ be a family of strictly monotone, continuous functions that converge to $f$ uniformly on compact sets. Then $f_n^{-1}$ converges to $f^{-1}$ uniformly on compact sets.
\end{lemma}

\begin{proof}
The proof is identical to that of proposition \ref{prop:inverse_uniform} with only minor alterations in notation.
\end{proof}

\subsection{Properties of $Z$ and its approximations}
In this section we assume $\Phi$ to be absolutely continuous, i.e.
\begin{equation*}
\Phi(u) = \int_0^u \Phi'(s)ds,
\end{equation*}
and $\Phi' \in L^1_{loc}(\mathbb{R})$. We also assume $\Phi$ to be strictly increasing and  $\Phi(\pm \infty) = \pm \infty$.

We define 
\begin{equation}
Z(u) = \int_0^u \mathrm{min}\{1, \Phi'(s)\} ds,
\end{equation}
and similarly
\begin{equation}
Z_n(u) = \int_0^u\mathrm{min}\{1, \Phi_n'(s)\}ds.
\end{equation}

As in the previous section we'll establish that $Z_n$ converges to $Z$ uniformly on compact sets, and continue on with the inverses and a similar convergence results for them. We'll start off with the following proposition:

\begin{proposition}
$Z_n$ converges pointwise to $Z$.
\end{proposition}
\begin{proof}
A useful identity to have in mind at this point is
\begin{equation*}
\mathrm{min}\{a,b\} = \frac{a+b}{2} - \frac{|a-b|}{2},
\end{equation*}
which makes
\begin{align*}
|\mathrm{min}\{1,a\} - \mathrm{min}\{1,b\}| &= \left| \frac{1+a}{2} -\frac{1+b}{2} + \frac{|1-b|}{2} - \frac{|1-a|}{2}\right| \\
&\leq \frac{1}{2}|a-b| + \frac{1}{2}\big| |1-b| - |1-a| \big| \\
&\leq \frac{1}{2}|a-b| + \frac{1}{2}|a-b| \\
&= |a-b|.
\end{align*}

For any $u \in \mathbb{R}$ we then have
\begin{align*}
|Z_n(u) - Z(u)| &\leq \int_0^{|u|} |\mathrm{min}\{1,\Phi'_n(s)\} - \mathrm{min}\{1,\Phi'(s)\}|ds \\
&\leq \int_0^{|u|} |\Phi'_n(s)-\Phi'(s)|ds,
\end{align*}
by using the above derived inequality. Following the definition of $\Phi_n$, its derivative is
\begin{equation*}
\Phi'_n(s) = \Phi' * \varphi_{\frac{1}{n}}(s) + \frac{1}{n},
\end{equation*}
and so
\begin{equation*}
|Z_n(u) - Z(u)| \leq \int_0^{|u|}|\Phi'*\varphi_{\frac{1}{n}}(s) - \Phi'(s)|ds + \frac{|u|}{n}.
\end{equation*}
We have of course complete control of the latter of these terms by making $n$ sufficiently large. For the first term we use part iv) of theorem 7 of appendix C in \citep{evans}, to state that $\Phi'*\varphi_{\frac{1}{n}}$ converges to $\Phi$ in $L^1_{loc}(\mathbb{R})$. With this we deduce that $Z_n$ converges pointwise to $Z$.
\end{proof}

Next we will prove that $Z_n$ converges to $Z$ uniformly on compact sets, but for this we need the Arzela-Ascoli theorem, presented below for ease of reference, (cf. e.g. \citep[p 718]{evans}). This theorem hinges on the concept of equicontinuity, for which a definition is presented directly below.
\begin{mydef}
A sequence $(f_n)$ of continuous, real-valued functions defined on an interval $[a,b]$ is said to be \textbf{equicontinuous} if for every $\epsilon >0$ there is a $\delta >0 $, depending only on $\epsilon$, so that for $x,y \in [a,b]$
\begin{equation*}
|x-y| < \delta \Rightarrow |f_n(x) - f_n(y)| < \epsilon
\end{equation*}
for every $n \in \mathbb{N}$.
\end{mydef}
We are now prepared for the Arzela-Ascoli theorem.
\begin{theorem}[Arzela-Ascoli]
\label{thm:Ascoli}
A bounded, equicontinuous sequence, $(f_n)$, of continuous, real-valued functions has a subsequence which converges uniformly.
\end{theorem}

Using this powerful theorem we can prove the following:
\begin{proposition}
\label{prop:Z_uniform_convergence}
$Z_n$ converges to $Z$ uniformly on compact sets.
\end{proposition}
\begin{proof}
The proof will be done by contradiction. Take $K \subset \mathbb{R}$ to be a compact set, and consider a closed interval of the form $[-M,M]$, with $M$ so large as to contain $K$. Obviously, $Z_n$ is a bounded sequence, with
\begin{equation*}
|Z_n(x)| \leq M.
\end{equation*}
In addition, the sequence is equicontinuous, since all $Z_n$'s are globally Lipschitz continuous with the same Lipschitz constant $1$.

We can from this conclude that $(Z_n)$ has a subsequence that converges uniformly on $[-M,M]$, and that the limit is $Z$, using theorem \ref{thm:Ascoli} and that $Z_n$ converges pointwise to $Z$. We can, with and identitical argument say that any subsequence of $(Z_n)$ also contains a subsequence converging uniformly to $Z$.

Suppose now that $(Z_n)$ does not converge to $Z$ uniformly on $[-M,M]$. Then there is an $\epsilon > 0$ so that for every $k \in \mathbb{N}$ there is an $n \geq k$ so that
\begin{equation*}
\underset{x\in[-M,M]}{\mathrm{sup}}|Z_n(x) - Z(x)| \geq \epsilon.
\end{equation*}
From this we can construct a subsequence $(Z_{n_k})$ so that
\begin{equation*}
\underset{x\in[-M,M]}{\mathrm{sup}}|Z_{n_k}(x) - Z(x)| \geq \epsilon,
\end{equation*}
for every $k$, but this is in clear contradiction with the fact that $(Z_{n_k})$ must have a subsequence converging uniformly to $Z$. Hence we can conclude that $(Z_n)$ converges uniformly to $Z$ on $[-M,M]$, and as a consequence also on $K$.
\end{proof}

We'll continue on with the last result of this section.
\begin{proposition}
\label{prop:Zinv_uniform_convergence}
With $\Phi$ strictly increasing and continuos, $Z$ is strictly increasing and continuous, and thus permits a strictly increasing and continuous inverse. Furthermore $Z_n^{-1}$ converges to $Z^{-1}$ uniformly on compact sets.
\end{proposition}
\begin{proof}
 We note that $Z_n' \geq \frac{1}{n}$, and is thus stricly increasing, permitting a continuous inverse $Z_n^{-1}$. 

That $Z$ is also strictly increasing follows from that $\Phi$ is strictly increasing. Indeed, suppose $t<u$ and see that
\begin{equation*}
Z(u) - Z(t) = \int_t^u \mathrm{min}\{1,\Phi'(s)\}ds \geq 0.
\end{equation*}
The only way this can be $0$ is if $\Phi' = 0$ a.e. everywhere on $(t,u)$, but if that were the case then $\Phi(u) = \Phi(t)$, in contradiction with $\Phi$ being strictly increasing. So $Z$ is strictly increasing, and has a continuous and strictly increasing inverse $Z^{-1}$.

We have from proposition \ref{prop:Z_uniform_convergence} that $Z_n$ converges uniformly to $Z$ on compact sets, and so we can simply use lemma \ref{lem:Inverse_uniform_general} to conclude that $Z_n^{-1}$ converges uniformly to $Z^{-1}$ on compact sets.
\end{proof}

\end{appendix}
\end{document}