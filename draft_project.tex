\documentclass[11pt, a4paper]{article}
\usepackage{mathptmx}
\usepackage{mathtools}
\usepackage{amssymb}
%\usepackage{amsfonts}
\usepackage{amsmath}
\usepackage{fullpage}
\usepackage{enumerate}
\usepackage{amsthm}
\usepackage{remreset}
\usepackage[utf8x]{inputenc}
\usepackage[toc,page]{appendix}
\usepackage[numbers]{natbib}

\usepackage{lmodern}
\newcommand{\HRule}{\rule{\linewidth}{0.5mm}}
\begin{document}
%%%%%%%%%%%%%%%%%%%%%%%%%%%%%%
%	TITLE PAGE:
%%%%%%%%%%%%%%%%%%%%%%%%%%%%%%
\pagestyle{empty}
\pagenumbering{roman}
\begin{titlepage}
\begin{center}

% Upper part of the page. The '~' is needed because \\
% only works if a paragraph has started.

\textsc{\LARGE Norwegians University of Science and Technology}\\[1.5cm]

\textsc{\Large TMA4500: Spesialisation project}\\[0.5cm]

% Title
\HRule \\[0.4cm]
{ \huge \bfseries The general porous medium equation \\[0.4cm] }

\HRule \\[1.5cm]

% Author and supervisor
\noindent
\begin{minipage}{0.4\textwidth}
\begin{flushleft} \large
\emph{Author:}\\
Trygve \textsc{Bærland}
\end{flushleft}
\end{minipage}%
\begin{minipage}{0.4\textwidth}
\begin{flushright} \large
\emph{Supervisor:} \\
Espen R. \textsc{Jakobsen}
\end{flushright}
\end{minipage}

\vfill

% Bottom of the page
{\large \today}

\end{center}
\end{titlepage}
%%%%%%%%%%%%%%%%%%%%%%%%%%%%%%%%%%%%%%%%%%%%%
%		END OF TITLE PAGE
%%%%%%%%%%%%%%%%%%%%%%%%%%%%%%%%%%%%%%%%%%%%%

\abstract{For now, heavily reliant on \citep{vazquez2007porous}.}

\newpage
\tableofcontents
\pagenumbering{arabic}
% My text categories:
\theoremstyle{plain}
\newtheorem{theorem}{Theorem}[section]
\newtheorem{proposition}{Proposition}[section]
\newtheorem{cor}{Corollary}[section]
\newtheorem{lemma}{Lemma}[section]

%Assumptions:
\newtheorem{assumption}{Assumption}[section]

\theoremstyle{definition}
\newtheorem{mydef}{Definition}[section]
\newtheorem{example}{Example}%[section]

%\theoremstyle{plain}
\newtheorem{obs}{Observation}
\newtheorem{rem}{Remark}




% Reset numbering for each section
\numberwithin{equation}{section}
%%%%%%%%%%%%%%%%%%%%%%%%%%%%%%%%%%%%%%%%%%%%%%%%%%%%%%%
%				INTRODUCTION
%%%%%%%%%%%%%%%%%%%%%%%%%%%%%%%%%%%%%%%%%%%%%%%%%%%%%%%
\newpage
\section{Introduction}

Let $\Omega \subset \mathbb{R}^d$ be an open, bounded, and simply connected domain. Then the filtration, or general porous medium equation is

\begin{equation}
\label{GPME}
	\partial_t u = \Delta \Phi(u) + f,
\end{equation}

 where $u, f: \Omega \times (0,\infty) \to \mathbb{R}$, and $\Phi: \mathbb{R} \to \mathbb{R}$.
 
The PDE (\ref{GPME}) encompasses some noteworthy special cases:
\begin{example}[PME]
When $\Phi(u) = u^m$ for $m>1$, (\ref{GPME}) is the Porous Medium Equation (PME), whose name is derived from its main application: Describing gas concentration flow in porous media, like rock. To allow for solutions with changing signs, one has $\Phi(u) = |u|^{m-1}u$. Notice that this choice of $\Phi$ is strictly increasing and locally Lipschitz continuous.
\end{example}

\begin{example}[FDE]
With the same choice of $\Phi(u) = |u|^{m-1}u$, only now with $0<m<1$ yields the so-called Fast Diffusion Equation (FDE). $\Phi$ is still strictly increasing, but no longer locally Lipschitz continuous.
\end{example}

\begin{example}[Stefan problem]
The Stefan problem can be posed with $\Phi(u) = (u-1)_+$, which is used to describe temperature distribution in a system with a phase transitions. E.g. melting an ice cube in a glass of water. Here $\Phi$ is no longer strictly increasing, which constitutes a degenerate problem.
\end{example}

\subsection{Basic sets and notation}
Before delving into the actual analysis of our problem, let's just assure ourselves that we're on the same page when it comes to notation of various sets and function spaces.

To begin with, we let $\Omega \subset \mathbb{R}^d$ be an open, bounded and simply connected set, with boundary $\partial \Omega$. This will denote our spatial domain, and so we let write $Q_T = \Omega \times (0,T]$ to mean the space-time cylinder up to a time $T$, which can be both finite or infinite, but throughout this text we will assume $T < \infty$. It will also be useful to define $Q_T^\tau = \Omega \times (\tau, T-\tau)$ for $0<\tau < \frac{T}{2}$. The last set pertaining to the space-time domain we'll need is the parabolic boundary $\Gamma_T = (\partial \Omega \times [0,T]) \cup (\Omega\times \{t=0\})$.


Most, if not all, of the notation concerning function spaces will follow the notation in \citep{evans} (see in particular Ch. 5). In the following let $U$ be an open subset of $\mathbb{R}^d$, then we will denote by $C(U)$ the space of bounded and continuous functions with norm
\begin{equation*}
||u||_{C(U)} = \underset{x \in U}{\mathrm{sup}}|u(x)|.
\end{equation*}
Further, with $0< \gamma \leq 1$ we define the Hölder seminorm as
\begin{equation*}
|u|_{C^{0,\gamma}(U)} = \underset{x,y \in U, x\neq y}{\mathrm{sup}}\frac{|u(x)-u(y)|}{|x-y|^\gamma}
\end{equation*}
which enables the Hölder space $C^{0,\gamma}(U)$, with norm
\begin{equation*}
||u||_{C^{0,\gamma}(U)} = ||u||_{C(U)} + |u|_{C^{0,\gamma}(U)}.
\end{equation*}

With $C^k(U)$ being the $k$ times continuously differentiable functions in $U$, we define the Hölder space $C^{k,\gamma}(U)$ to be all $u \in C^{k}(U)$ si that the norm
\begin{equation*}
||u||_{C^{k,\gamma}(U)} = \sum_{|\alpha| \leq k} ||D^\alpha u||_{C^(U)} + \sum_{|\alpha| = k} |D^\alpha u |_{C^{0,\gamma}},
\end{equation*}
where we have adopted the multiindex notation of \citep{evans}. In addition we say that the boundary $U$, $\partial U$, is $C^{k, \alpha}$ if for every point on $\partial U$ there is a neighbourhood over which $\partial U$ describes the graph of a $C^{k,\alpha}$ function in $d-1$ coordinates, \citep{gilbarg2001elliptic}.

Seeing as we're considering a time-dependent problem, different assumptions on the smoothness of a functions in space and time will emerge, and as such function spaces that reflect this are necessary. The spaces $C^k_r(U)$ are the $k$-times continuously differentiable in space and $r$ times continuously differentiable in time. Meaning that if $u \in C^k_r(U)$, then $\partial_t^m D^\alpha u$ exists and is bounded and continuous in $U$ for all $m \leq r$ and all multiindices $\alpha$ so that $|\alpha| \leq k$.

We will also need to consider the spaces of strongly summable functions. Let $X$ be a normed space and $1\leq p < \infty$, then $u \in L^p(0,T:X)$ if
\begin{equation*}
||u||_{L^p(0,T:X)} = \left( \int_0^T ||u(t)||_X^p dt \right)^\frac{1}{p}
\end{equation*} 
is finite. This of course has the natural extension to $p=\infty$ with norm
\begin{equation*}
||u||_{L^\infty(0,T:X)} = \mathrm{ess \; sup}_{0\leq t \leq T} ||u(t)||_X.
\end{equation*}

The last function space we'll pay particular attention to in this section is the Hölder space $H^{l, l/2}(Q_T)$ \citep{ladyzhenskaya}, which consists of all $u$ that are continuous as well as $\partial_t^rD^\alpha u$ are continuous for all $r$ and $\alpha$ so that $2r + |\alpha| <l$. The norm is given as
\begin{equation*}
	||u||_{H^{l,l/2}(Q_T)} = \langle u \rangle^{(l)}_{Q_T} + \sum_{j=0}^{[l]}\langle u \rangle^{(j)}_{Q_T},
\end{equation*}
with
\begin{align*}
\langle u \rangle^{(j)}_{Q_T} &= \sum_{2r+|\alpha|=j} ||\partial_t^r D^\alpha u||_{C(\bar{Q}_T)} \\
\langle u \rangle^{(l)}_{Q_T} &= \langle u \rangle^{(l)}_{x,Q_T} + \langle u \rangle^{(l/2)}_{t,Q_T} \\
\langle u \rangle^{(l)}_{x,Q_T} &= \sum_{2r+|\alpha|=[l]} |\partial_t^r D^{\alpha}u|_{C^{0,l-[l]}_x(\bar{Q}_T)} \\
 \langle u \rangle^{(l/2)}_{t,Q_T} &= \sum_{0 < l - 2r -|\alpha| < 2} |\partial_t^r D^\alpha u|_{C^{0,(l-2r-s)/2}_t(\bar{Q}_T)},
\end{align*}
with the Hölder semi-norm with respect to $x$ and $t$ respectively in the last two definitions being 
\begin{align*}
|u|_{C_x^{0,\rho}(Q_T)} &= \underset{\overset{(x,t),(y,t)\in Q_T}{x\neq y}}{\mathrm{sup}} \frac{|u(x,t) - u(y,t)|}{|x-y|^\rho} \\
|u|_{C_t^{0,\rho}(Q_T)} &= \underset{\overset{(x,t),(x,s)\in Q_T}{t\neq s}}{\mathrm{sup}}\frac{|u(x,t) - u(x,s)|}{|t-s|^\rho},
\end{align*}
for $0 < \rho \leq 1$.

Luckily this norm will not be of the most importance in what is to follow, but some acquaintance with it is useful.

 
 
\subsection{Problem statement}
We will consider on the following Dirichlet problem:
\begin{equation}
\label{HDP}
	\begin{cases}
		\partial_t u &= \Delta \Phi(u) + f, \quad \text{ in } Q_T \\
		u(x,0) &= u_0(x), \quad \text{ in } \Omega \\
		u(x,t) &= 0, \quad \text{ on } \partial\Omega \times [0,T].
	\end{cases}
\end{equation}
Here $Q_T = \Omega \times (0,T]$, where $T$ may very well be infinite. Furthermore we denote by $\Gamma_T = \overline{Q}_T\backslash Q_T$ the parabolic boundary.

%%%%%%%%%%%%%%%%%%%%%%%%%%%%%%%%%%%%%%%%%%%%%%%%%%
%		ASSUMPTIONS CAN POSSIBLY GO HERE???
%%%%%%%%%%%%%%%%%%%%%%%%%%%%%%%%%%%%%%%%%%%%%%%%%%


%%%%%%%%%%%%%%%%%%%%%%%%%%%%%%%%%%%%%%%%%%%%%%%%%%%%%%%
%				CLASSICAL SOLUTIONS
%%%%%%%%%%%%%%%%%%%%%%%%%%%%%%%%%%%%%%%%%%%%%%%%%%%%%%%
\section{Classical solutions}
\subsection{Preliminaries and assumptions}
The strictest sense in which we can have a solution of (\ref{HDP}) is that there is a $u \in C^2_1(Q_T)\cap C(\overline{Q}_T)$ for which $u\big|_{t=0} = u_0$,  $u\big|_{\partial\Omega \times [0,T]} = 0$, and the equality
\begin{equation*}
\partial_t u = \Delta \Phi(u) + f
\end{equation*}
holds everywhere in $Q_T$, and we call such a solution $u$ a classical solution.

In this section, when deriving various results pertaining to classical solutions, we will do so under various assumptions on $\Phi$, $u_0$ and $f$. These assumptions are listed immediately below.

\begin{assumption}[Regularity of $\Phi$]
\label{ass:classical_Phi_smooth}
$\Phi \in C^2(\mathbb{R})$ and strictly increasing, as well as $\Phi(0) = 0$.
\end{assumption}

\begin{assumption}[Uniform parabolicity]
\label{ass:classical_uniform_parabolicity}
There exists constants $ 0 < \nu < \mu < \infty$ so that
\begin{equation}
\label{uniform_parabGPME}
\nu \leq \Phi'(s) \leq \mu, \quad u \in \mathbb{R}
\end{equation}
holds for all $s \in \mathbb{R}$.
\end{assumption}

\begin{assumption}[Continuity of $f$]
\label{ass:classical_f}
$f$ is bounded and continuous in $\overline{Q}_T$.
\end{assumption}

\begin{assumption}[Continuity of $u_0$]
\label{ass:classical_u_0}
$u_0$ is bounded and continuous in $\overline{\Omega}$. In addition, 
\begin{equation}
u_0\big|_{\partial \Omega} = 0.
\end{equation}
\end{assumption}

Most of the above assumptions are naturally motivated by the regularity restrictions on a classical solution. For example if $f$ was discontinuous somewhere in $Q_T$ this would lead to a discontinuity of the time derivative of a solution.

The only one that may not be immediately clear is assumption \ref{ass:classical_uniform_parabolicity}, and to somewhat motivate this assumption it is useful to consider the more general parabolic Dirichlet problem 
\begin{equation}
\label{general_parabolic_problem}
	\begin{cases}
		\partial_t u + Lu &= f, \quad \text{in } Q_T \\
		u(x,0) &= u_0(x), \quad \text{in } \Omega \\
		u &= 0, \quad \text{on } \partial \Omega \times [0,T]
	\end{cases}
\end{equation}
where $L$ is a differential operator that can be written in the divergence form
\begin{equation}
\label{divergence_operator}
Lu = -\sum_{i=1}^d \partial_{x_i}(a^i(x,t,u,\nabla u)) + a(x,t,u,\nabla u),
\end{equation}
where the functions $a^i$ and $a$ are given and continuous with respect to all variables.

Many results regarding problems of the type (\ref{general_parabolic_problem}) assumes there are constants $0 < \nu < \mu < \infty $ so that
\begin{equation}
\label{uniform_parabolicity}
\nu |\xi|^2 \leq \sum_{i,j=1}^d \frac{\partial a^i(x,t,u,p)}{\partial p_j}\xi_i \xi_j \leq \mu |\xi|^2 
\end{equation}
for every $\xi \in \mathbb{R}^d$, all $(x,t) \in Q_T$ and all $u\in \mathbb{R}$. If this condition is satisfied we say that the operator $\partial_t + L$ is \textbf{uniformly parabolic}.

With the special case
\begin{equation*}
a^i(x,t,u,p) = \Phi'(u)p_i,
\end{equation*}
(\ref{general_parabolic_problem}) reduces to the homogeneous Dirichlet problem (\ref{HDP}),
and so (\ref{uniform_parabolicity}) translates to assumption \ref{ass:classical_uniform_parabolicity}.

In this framework the remainder of this section considering classical solutions will be devoted to deriving various a priori estimates for classical solutions, and ending with a corollary stating sufficient conditions for existence of a classical solution to (\ref{HDP}). As we will see, these conditions are quite strict, leading us naturally to want to consider weak solutions, which the next section is devoted to.
%%%%%%%%%%%%%%%%%%%%%%%%%%%%%%%%%%%%%%%%%%%%%%%%%%%%%
%			MAXIMUM PRINCIPLE:
%%%%%%%%%%%%%%%%%%%%%%%%%%%%%%%%%%%%%%%%%%%%%%%%%%%%%
\subsection{Maximum principle}
The first main estimate we'll deal with is a $L^\infty$ bound on a classical solution. To that end, the following strong maximum principle will be helpful (cf. \citep[p. 396]{evans}).

\begin{theorem}[Strong maximum principle]
\begin{enumerate}[i)]
	Suppose $\Omega$ is connected, and $u\in C^{2,1}(Q_T)\cap C(\overline{Q}_T)$, and $\partial_t + L$ is uniformly parabolic, then:
	\item If
	\begin{equation*}
		\partial_tu + Lu \leq 0, \quad \text{ in } Q_T, 
	\end{equation*}
	and $u$ attains its maximum over $\overline{Q}_T$ at point $(x_0,t_0)\in Q_T$, then
	$u$ is constant on $Q_{t_0}$.
	
	\item Analagously, if
	\begin{equation*}
		\partial_tu + Lu \geq 0, \quad \text{ in } Q_T, 
	\end{equation*}
	and $u$ attains its minimum over $\overline{Q}_T$ at a points  $(x_0,t_0) \in Q_T$, then $u$ is constant on $Q_{t_0}$.
\end{enumerate}
\end{theorem}

Using this theorem we can prove the comparison principle:
\begin{lemma}[Comparison principle]
\label{lemma:comparison}
Under assumptions \ref{ass:classical_Phi_smooth}-\ref{ass:classical_u_0}, let $u,v \in C^{2,1}(Q_T)\cap C(\overline{Q}_T)$ be two solutions of the General Porous Medium Equation (\ref{GPME}) in $Q_T$ with initial data $u_0$, $v_0$, loading $f$, $\tilde{f}$ respectively as well as prescribed Dirchlet data on the boundary. Further, assume that $v\big|_{\Gamma_T} \geq u\big|_{\Gamma_T}$, and  $\tilde{f} \geq f$ in $Q_T$. Then $u \leq v$ in $Q_T$.
\end{lemma}
\begin{proof}
The strategy will be to set $w = v-u$ and derive a initial- and boundary value problem that  $w$ solves. First off, we see that
\begin{align*}
\partial_t w &= \Delta(\Phi(v) - \Phi(u)) + (\tilde{f}-f) \\
&= \Delta ( \Phi(w+u) - \Phi(v-w) ) + (\tilde{f}-f)
\end{align*}
So, when treating $u$ and $v$ as parameters, we can formulate the problem
\begin{equation}
\begin{cases}
\partial_t w &= \Delta ( \Phi(w+u) - \Phi(v-w) ) + (\tilde{f}-f), \quad \text{ in } Q_T \\
w(x,0) &= v_0(x) - u_0(x), \quad \text{ in } \Omega \\
w(x,t) &= v(x,t) - u(x,t), \quad \text{ on } \partial \Omega \times [0,T].
\end{cases}
\end{equation}
Now we want to use the strong maximum principle, so we should find out what $a^i$ and $a$ are. We have that
\begin{align*}
\Delta ( \Phi(w+u) - \Phi(v-w) )&= \sum_{i=1}^d \partial_{x_i}\left(\Phi'(w+u)(\partial_{x_i}w + \partial_{x_i}u) - \Phi'(v-w)(\partial_{x_i}v-\partial_{x_i}w) \right),
\end{align*} 
and seeing as $u$ and $v$ should be considered parameters to this problem we can establish that
\begin{equation*}
\begin{cases}
a^i(x,t,w,p) &= (\Phi'(w+u) + \Phi'(v-w))p_i \\
a(x,t,w,p) &= \sum_{i=1}^d \partial_{x_i}\left( \Phi'(u+w)\partial_{x_i}u - \Phi'(v-w)\partial_{x_i}v\right).
\end{cases}
\end{equation*}
The $a^i$'s are continuous from the assumptions on $\Phi$, and in addition it satisfies (\ref{uniform_parabGPME}) inherited from $\Phi$. The $a$ is continuous since both $u$ and $v$ are in $C^{2,1}(Q_T)$ and $\Phi \in C^2(\mathbb{R})$.

We can thus use the maximum principle together with $\tilde{f}-f \geq 0$ to deduce that
\begin{equation*}
w \geq \mathrm{min}_{\overline{Q}_T}w = \mathrm{min}_{\Gamma_T}w.
\end{equation*}
By assumption $\mathrm{min}_{\Gamma_T}w \geq 0$, so $v \geq u$. We conclude the proof that seeing that if $u=v$ at some internal point $(x_0,t_0) \in Q_T$, then $u=v$ in $Q_{t_0}$.
\end{proof}


Let's now use this lemma to get an $L^\infty$-bound on classical solutions.
\begin{cor}
\label{cor:Linf_bound_classical}
Under the same assumptions as Lemma \ref{lemma:comparison},  let $u \in C^{2,1}(Q_T)\cap C(\overline{Q}_T)$ be a classical solution of (\ref{HDP}), then
	\begin{equation}
		\label{infinity_bound}
		||u||_{L^\infty(Q_T)} \leq ||u_0||_{L^\infty(\Omega)} + T||f||_{L^\infty(Q_T)}.
	\end{equation}
\end{cor}

\begin{proof}
We start by defnining $M_1 = ||u_0^+||_{L^\infty(\Omega)} $, and $N_1 = ||f^+||_{L^\infty(Q_T)}$, where $(\cdot)^+ = \mathrm{max} \{ 0,\cdot \}$. With $v(x,t) = M_1 + N_1t$, it solves
\begin{equation*}
\begin{cases}
	\partial_t v &= \Delta\Phi(v) +N_1 \quad \text{in } Q_T\\
	v(x,0) &= M_1 \quad \text{in } \Omega \\
	v(x,t) &= M_1 + N_1t \quad \text{on } \partial \Omega \times [0,T].
\end{cases}
\end{equation*}
Since $u_0 \leq M_1$ and $f\leq N_1$ we can use the comparison principle to conclude that
\begin{equation*}
u(x,t) \leq M_1 + N_1t
\end{equation*}
in $Q_T$.

Similarly we define $M_2 = -||u_0^-||_{L^\infty(\Omega)}$, and $N_2 = -||f^-||_{L^\infty(Q_T)}$. To avoid any possible confusion, here $(\cdot)^- = \mathrm{min}\{0, \cdot \}$. We use the same reasoning to get the lower bound
\begin{equation*}
u(x,t) \geq M_2 + N_2t.
\end{equation*}

Because
\begin{equation*}
||u_0||_{L^\infty(\Omega)} = \mathrm{max}\{ ||u_0^+||_{L^\infty(\Omega)}, ||u_0^-||_{L^\infty(\Omega)} \} = \mathrm{max}\{|M_1|,|M_2|\},
\end{equation*}
and similarly for $f$, our two bounds implies that
\begin{equation*}
|u(x,t)| \leq ||u_0||_{L^\infty(\Omega)} + t||f||_{L^\infty(Q_T)},
\end{equation*}
which  is what we wanted to prove.

\end{proof}


%%%%%%%%%%%%%%%%%%%%%%%%%%%%%%%%%%%%%%%%%%%%%%%%%%%%%%%
%				L^1-CONTRACTIVITY
%%%%%%%%%%%%%%%%%%%%%%%%%%%%%%%%%%%%%%%%%%%%%%%%%%%%%%%
\subsection{$L^1$-contractivity}
Moving right along, the next a priori estimate up for consideration is one regarding stability in $L^\infty(0,T: L^1(\Omega))$. To reach our goal we start off with the following lemma, from which the main result will be an almost immediate consequence.

\begin{lemma}
\label{lem:l1_stability_classical}
Under assumptions \ref{ass:classical_Phi_smooth}, \ref{ass:classical_f} and \ref{ass:classical_u_0}, let $u, \hat{u} \in C^2_1(Q_T)\cap C(\overline{Q}_T)$, be two classical solutions of the homogeneous Dirichlet problem (\ref{HDP}), with initial data $u_0$ and $\hat{u}_0$ as well as loading $f$ and $\hat{f}$, respectively. Then, for $\tau, t \in [0,T]$, the inequality 
\begin{equation}
\label{l1_contr_part1}
\int_{\Omega}(u(t)-\hat{u}(t))_+dx \leq \int_{\Omega}(u(\tau)-\hat{u}(\tau))_+dx + \int_\tau^t\int_{\Omega}(f-\hat{f})_+dxdt.
\end{equation}
is satisfied.
\end{lemma}

\begin{proof}
Let $p: \mathbb{R} \to \mathbb{R}$ be a $C^1$ function satisfying $p(s) = 0$ for $s \leq 0$, $p'(s) \geq 0$ for all $s\in \mathbb{R}$, and $0 \leq p(s) \leq 1$ for $s > 0$. Define now $w = \Phi(u) - \Phi(\hat{u})$, and multiply $p(w)$ with the difference of (\ref{GPME}) for $u$ and $\hat{u}$ to get
\begin{equation}
\label{l1_stab_step1}
\partial_t(u-\hat{u})p(w) = p(w)\Delta w + p(w)(f-\hat{f}).
\end{equation}
Integrating over $\Omega$, and using integration by parts on the first part on the right hand side yields
\begin{equation*}
\int_\Omega p(w)\Delta w dx = -\int_\Omega p'(w)|\nabla w|^2dx + \int_{\partial \Omega}p(w)\nabla w \cdot \hat{n} dS,
\end{equation*}
where $\hat{n}$ is the unit normal vector pointing out of $\Omega$. The boundary integral term vanishes since $w\big|_{\partial \Omega} = 0$ and $p(0)=0$. Thus integration over $\Omega$ in (\ref{l1_stab_step1}) yields 
\begin{align*}
\int_{\Omega}\partial_t(u-\hat{u})p(w)dx &= -\int_{\Omega}p'(w)|\nabla w|^2dx + \int_{\Omega}p(w)(f-\hat{f}) dx \\
	&\leq \int_{\Omega}p(w)(f-\hat{f}) dx \\
	&\leq \int_{\Omega}(f-\hat{f})_+ dx,
\end{align*}
where the last step follows from the fact that $0 \leq p(w) \leq 1$. Here $(\cdot)_+ = \mathrm{max}\{\cdot, 0\}$.
Now, let $p$ converge pointwise in an increasing fashion to the positive sign function $\mathrm{sign}_0^+$ defined
\begin{equation*}
\mathrm{sign}^+_0(s) = \begin{cases}
1&, \quad \text{for } s>0 \\
0&, \quad \text{for } s\leq 0.
\end{cases}
\end{equation*}

We need to ensure then that
\begin{equation}
\label{cool_use_of_MCT}
\int_\Omega \partial_t (u - \hat{u}) p(w) dx \underset{n \to \infty}{\to} \int_\Omega \partial_t (u - \hat{u})\mathrm{sign}^+_0(w)dx.
\end{equation}
It is obvious that $\partial_t (u - \hat{u})p(w)$ converges pointwise to $\partial_t (u-\hat{u})\mathrm{sign}_0^+(w)$, and we proceed by defining the two sets
\begin{equation*}
\begin{cases}
\Omega_+ &= \{ x \in \Omega: \partial_t (u - \hat{u}) \geq 0 \}, \text{ and} \\
\Omega_- &=  \{ x \in \Omega: \partial_t (u - \hat{u}) < 0\} 
\end{cases}
\end{equation*}
and split up the integral in following fashion,
\begin{equation}
\int_\Omega \partial_t (u - \hat{u}) p(w) dx = \int_{\Omega_+}\partial_t (u - \hat{u}) p(w)dx - \int_{\Omega_-} -\partial_t (u - \hat{u}) p(w)dx. 
\end{equation}

In both integrals the integrands are nonnegative and converge monotonically to $\pm \partial_t (u - \hat{u})\mathrm{sign}^+_0(w)$, and so use of the Monotone convergence theorem (cf. \citep[Thm. 5.6, p. 165]{weiss1999course}) to conclude that (\ref{cool_use_of_MCT}) is indeed valid.


Notice now that $\mathrm{sign}_0^+(u-\hat{u}) = \mathrm{sign}_0^+(\Phi(u) - \Phi(\hat{u}))$ because of the assumption that $\Phi$ is strictly increasing. In addition we have $\partial_t (u-\hat{u})_+ = \mathrm{sign}_0^+(u-\hat{u}) \partial_t (u - \hat{u})$, in the weak derivative sense. Thus
\begin{equation*}
\int_\Omega \partial_t(u-\hat{u})p(w) dx \to \int_\Omega \partial_t (u-\hat{u})_+ dx,
\end{equation*}
and integrating in time from $\tau$ to $t$ yields
\begin{equation*}
\int_\tau^t \int_{\Omega}\partial_t(u(t)-\hat{u}(t))_+dx \leq \int_\tau^t\int_{\Omega}(f-\hat{f})_+dxdt.
\end{equation*}

Using Fubini's theorem to interchange the integration on the left hand side then finally results in
\begin{equation}
\int_\Omega (u(t) - \hat{u})_+ dx \leq \int_\Omega (u(\tau) - \hat{u}(\tau))_+ dx + int_\tau^t \int_\Omega (f-\hat{f})_+ dxdt.
\end{equation}
\end{proof}


\begin{cor}[$L^1$-contractivity]
\label{cor:l1-contractivity_classical}
Under the same assumption as lemma \ref{lem:l1_stability_classical}
\begin{equation}
\label{l1_contractivity}
||u(t) - \hat{u}(t)||_{L^1(\Omega)} \leq ||u_0 - \hat{u}_0||_{L^1(\Omega)} + ||f-\hat{f}||_{L^1(Q_t)}
\end{equation}
holds for every $t\in [0,T]$.
\end{cor}

\begin{proof}
Setting $\tau = 0$ we have from the previous lemma that
\begin{equation*}
\int_\Omega  (u(t)-\hat{u}(t))_+dx \leq \int_\Omega (u_0 - \hat{u}_0)_+dx + \int_0^t \int_\Omega (f-\hat{f})_+ dxdt,
\end{equation*}
and by interchanging $u$ and $\hat{u}$ we also have 
\begin{equation*}
\int_\Omega  (\hat{u}(t)-u(t))_+dx \leq \int_\Omega (\hat{u}_0 - u_0)_+dx + \int_0^t \int_\Omega (\hat{f}-f)_+ dxdt.
\end{equation*}

Since $|a| = (a)_+ + (-a)_+$, the sum of these inequalities yields the desired result.
\end{proof}
An obvious consequence of lemma \ref{cor:l1-contractivity_classical} is uniqueness of classical solutions.
\begin{cor}
Under the same assumptions as Lemma \ref{lem:l1_stability_classical}, then the homogeneous Dirichlet problem (\ref{HDP}) has at most on solution in $C^2_1(Q_T)\cap C(\overline{Q}_T)$.
\end{cor}

\begin{proof}
Assume $u$ and $v$ are two classical solutions of (\ref{HDP}) with the same initial data $u_0$, then corollary \ref{cor:l1-contractivity_classical} implies that
\begin{equation*}
||u(t)-v(t)||_{L^1(\Omega)} \leq 0
\end{equation*}
for all $t \in [0,T]$, which means that $u=v$ almost everywhere. Since both solutions are continuous we conclude that $u=v$ everywhere in $\overline{Q}_T$.
\end{proof}

%%%%%%%%%%%%%%%%%%%%%%%%%%%%%%%%%%%%%%%%%%%%%%%
%			SPATIAL DER. BOUND
%%%%%%%%%%%%%%%%%%%%%%%%%%%%%%%%%%%%%%%%%%%%%%%
\subsection{Control of the spatial derivative}
This bound follows nicely the $L^1$-contractivity in that its derivation hinges on a clever choice of function to multiply (\ref{GPME}) with. Before proceeding, we define  $\Psi$ as the primitive of $\Phi$ with $\Psi(0) = 0$, i.e.
\begin{equation}
\label{phi_primitive}
	\Psi(s) = \int_0^s \Phi(u) ds,
\end{equation}

The bound on the spatial derivative is then summarised in the following lemma.

\begin{lemma}
\label{lem:Energy_equality_classical}
With the same assumptions as Lemma \ref{lem:l1_stability_classical} we have for a classical solution $u\in C^{2,1}(Q_T) \cap C(\overline{Q}_T)$ of (\ref{HDP}) that
\begin{equation}
\label{spatial_derivative_bound}
\iint_{Q_\tau} |\nabla \Phi(u)|^2dxdt + \int_\Omega \Psi(u(\tau))dx = \int_\Omega \Psi(u_0)dx + \iint_{Q_\tau} f\Phi(u)dxdt,
\end{equation}
for every $\tau \in [0,T]$.
\end{lemma}

\begin{proof}
We start off by multiplying (\ref{GPME}) with $\Phi(u)$ to get
\begin{equation*}
\Phi(u)\partial_t u = \Phi(u)\Delta \Phi(u) + \Phi(u)f,
\end{equation*}
and we notice that $\Phi(u)\partial_t u = \partial_t \Psi(u)$. That means when we integrate over $Q_t$ we have
\begin{equation*}
\iint_{Q_\tau} \partial_t \Psi(u)dxdt = \int_\Omega \Psi(u(\tau))dx - \int_\Omega \Psi(u_0)dx.
\end{equation*}
On the diffusive term we can use integration by parts, so
\begin{equation*}
\iint_{Q_\tau}\Phi(u) \Delta \Phi(u) dxdt = -\iint_{Q_\tau} |\nabla \Phi(u)|^2dxdt + \int_0^\tau \int_{\partial\Omega} \Phi(u)\nabla \Phi(u) \cdot \hat{n}dS,
\end{equation*}
where $\hat{n}$ is the outward unit normal. By the Dirichlet condition
\begin{equation*}
\iint_{Q_\tau}\Phi(u) \Delta \Phi(u) dxdt = -\iint_{Q_\tau} |\nabla \Phi(u)|^2dxdt,
\end{equation*}
and notice that this step would have been valid for homogeneous Neumann condition as well. In toto 
\begin{equation*}
\iint_{Q_\tau} |\nabla \Phi(u)|^2dxdt + \int_\Omega \Psi(u(\tau))dx = \int_\Omega \Psi(u_0)dx + \iint_{Q_\tau} f\Phi(u)dxdt.
\end{equation*}
\end{proof}
\setcounter{obs}{0}

We can go even further and deal with the last term on the right hand side of (\ref{spatial_derivative_bound}):

\begin{cor}
\label{cor:spatial_bound_w_poincare}
In addition to the assumptions under lemma \ref{lem:Energy_equality_classical}, assume $f,\Phi(u) \in L^2(Q_T)$, then for each $\tau \in [0,T]$
\begin{equation}
\label{spatial_derivative_bound_w_poaincare}
\frac{1}{2}\iint_{Q_\tau}|\nabla \Phi(u)|^2dxdt + \int_\Omega \Psi(u(\tau))dx \leq \int_\Omega \Psi(u_0)dx + C\iint_{Q_\tau}f^2dxdt.
\end{equation}
for some $C$ dependent only on $\Omega$ and the number of spatial dimensions $d$.
\end{cor}

\begin{proof}
for the term involving $f$ in (\ref{spatial_derivative_bound}) we use the inequality $ab \leq \frac{a^2}{4c} + cb^2$ for any reals $a$ and $b$, and any positive $c$, to get
\begin{equation*}
\iint_{Q_\tau}f\Phi(u) dxdt \leq \frac{1}{4c}\iint_{Q_\tau}f^2dxdt + c\iint_{Q_\tau}\Phi(u)^2 dxdt.
\end{equation*}

On the latter of the right hand terms we can for each $t \in [0,\tau]$ use the Poincaré inequality
\begin{equation*}
\int_\Omega \Phi(u)^2 dx \leq \tilde{C}\int_\Omega |\nabla \Phi(u) |^2 dx,
\end{equation*}
where $\tilde{C}>0$ depends only on $\Omega$ and $d$. We then have
\begin{equation*}
\iint_{Q_\tau} f\Phi(u)dxdt \leq \frac{1}{4c}\iint_{Q_\tau}f^2dxdt + c\tilde{C}\iint_{Q_\tau}|\nabla \Phi(u)|^2dxdt. 
\end{equation*}

We set now $c = \frac{1}{2\tilde{C}}$, and $C= \frac{\tilde{C}}{2}$, which results in
\begin{equation*}
\iint_{Q_\tau}f\Phi(u)dxdt \leq C\iint_{Q_\tau}f^2dxdt + \frac{1}{2}\iint_{Q_\tau}|\nabla \Phi(u)|^2dxdt.
\end{equation*}
Putting this back into (\ref{spatial_derivative_bound}) we get
\begin{equation}
\frac{1}{2}\iint_{Q_\tau}|\nabla \Phi(u)|^2dxdt + \int_\Omega \Psi(u(\tau))dx \leq \int_\Omega \Psi(u_0)dx + C\iint_{Q_\tau}f^2dxdt.
\end{equation}
\end{proof}


%%%%%%%%%%%%%%%%%%%%%%%%%%%%%%%%%%%%%%%%%%%%%%%%%%%%%%%%%%%%
%				TEMPORAL DER. BOUND
%%%%%%%%%%%%%%%%%%%%%%%%%%%%%%%%%%%%%%%%%%%%%%%%%%%%%%%%%%%%
\subsection{Control of the temporal derivative}
A similar, but perhaps a bit more involved argument can be used to make an estimate the temporal derivative.

\begin{lemma}
\label{lem:temporal_bound_classical}
Under the same assumptions as in Lemma \ref{lem:l1_stability_classical}, assume in addition that $\partial_t f$ is continuous. Let $u\in C^2_1(Q_T) \cap C(\overline{Q}_T)$ be a classical solution of (\ref{HDP}). Further, let $\zeta = \zeta(t)$ be a $C^1([0,T],[0,1])$ cutoff function so that $\zeta(0) = \zeta(T) = 0$. Then
\begin{equation}
\label{temporal_bound_classical}
\iint_{Q_T} \zeta \Phi'(u) |\partial_t u|^2 dxdt = \iint_{Q_T} \left\{ \frac{\partial_t \zeta}{2}|\nabla \Phi(u)|^2 - \partial_t(\zeta f) \Phi(u) \right\} dxdt.
\end{equation}
\end{lemma}

\begin{proof}

 If we multiply (\ref{GPME}) by $\zeta \partial_t\Phi(u)$ and integrate over $Q_T$ we get
	\begin{equation}
	\label{temporal_bound_start}
	\iint_{Q_T}\zeta \Phi'(u)|\partial_t u|^2 dxdt = \iint_{Q_T} \left\{ \zeta \partial_t \Phi(u) \Delta \Phi(u)  + \zeta \partial_t \Phi(u) f \right\} dxdt.
	\end{equation}
	Let's start by considering the first term on the right hand side. By the product rule for the divergence
	\begin{equation*}
	\int_0^T \zeta \int_\Omega \partial_t \Phi(u) \Delta \Phi(u) dxdt = \int_0^T \zeta \int_\Omega \left\{ \nabla \cdot (\partial_t \Phi(u) \nabla \Phi(u)) - (\partial_t \nabla \Phi(u)) \cdot \nabla \Phi(u) \right\} dx dt.
	\end{equation*}
	So after using the divergence theorem together with the homogeneous boundary data, as well as noticing that $(\partial_t \nabla \Phi(u)) \cdot \nabla \Phi(u) = \frac{1}{2}\partial_t |\nabla \Phi(u)|^2$, we see that
	\begin{equation*}
	\int_0^T \zeta \int_\Omega \partial_t \Phi(u) \Delta \Phi(u) dxdt = -\iint_{Q_T}\frac{\zeta}{2}\partial_t |\nabla \Phi(u)|^2 dxdt,
	\end{equation*}
	Integrating by parts in the temporal dimension, we end up with
	\begin{equation*}
	\int_0^T \zeta \int_\Omega \partial_t \Phi(u) \Delta \Phi(u) dxdt = \iint_{Q_T} \frac{\partial_t \zeta}{2} |\nabla \Phi(u)|^2 dxdt,
	\end{equation*}
	where the temporal end point integrals vanish since $\zeta(0) = \zeta(T) = 0$.
	A similar inetgration by parts in the temporal dimension of the last term on the right hand side in (\ref{temporal_bound_start}) yields
	\begin{equation*}
	\iint_{Q_T} \zeta f \partial_t \Phi(u) dxdt = -\iint_{Q_T} \partial_t (\zeta f) \Phi(u) dxdt
	\end{equation*}
	where again we've made use of $\zeta(0) = \zeta(T) = 0$.
	Putting all this back into (\ref{temporal_bound_start}) results in the desired equality 
	\begin{equation}
	\iint_{Q_T} \zeta \Phi'(u) |\partial_t u|^2 dxdt = \iint_{Q_T} \left\{ \frac{\partial_t \zeta}{2}|\nabla \Phi(u)|^2 - \partial_t(\zeta f) \Phi(u) \right\} dxdt.
	\end{equation}
\end{proof}


%%%%%%%%%%%%%%%%%%%%%%%%%%%%%%%%%%%%%%%%%%%%%%%%%
%				EXISTENCE:
%%%%%%%%%%%%%%%%%%%%%%%%%%%%%%%%%%%%%%%%%%%%%%%%%
\subsection{Existence}
Now we move on to tackle one of the central questions with regards to the well-posedness of (\ref{HDP}), viz. existence, and in particular what sort of assumptions we can put on $u_0$, $f$ and $\Phi$ to be guaranteed that a solution exists.

We will make good use of an admittedly quite harrowing theorem presented in \citep{ladyzhenskaya}. This theorem is quite general in its formulation and our job here will mainly be to translate all the conditions to the general porous medium equation, but before delving into this mammoth, let's set the stage, and introduce some notation used by Ladyzhenskaya et. al.

\subsubsection{Preliminaries}

Recalling the divergence form (\ref{divergence_operator}) at the beginning of this section, it will be useful to consider a general parabolic equation of the form
\begin{equation}
\label{lady_parabolic_eq}
	u_t - \sum_{i=1}^d\frac{d}{dx_i}a_i(x,t,u,\nabla u) + a(x,t,u,\nabla u) = 0
\end{equation}
in $Q_T$. In addition we formulate the initial value and boundary as
\begin{equation}
\label{lady_bc}
	u\big|_{\Gamma_T} = \psi\big|_{\Gamma_T}
\end{equation}
for some known function $\psi$.

Furthermore, we define
\begin{equation*}
A(x,t,u,\nabla u) = a(x,t,u,\nabla u) - \sum_{i=1}^d \left[ \frac{\partial a_i(x,t,u,\nabla u)}{\partial u}\partial_{x_i}u + \frac{\partial a_i(x,t,u,\nabla u)}{\partial x_i} \right]
\end{equation*}


\subsubsection{The big, general existence theorem}
We should now will at least partly prepared for the big existence theorem. It is presented immediately below, almost verbatim from \citep[ Theorem 6.1, p. 452]{ladyzhenskaya}.

\begin{theorem}[Existence of classical solutions]
\label{thm:classical_existence}
Suppose that the following conditions hold.
\begin{enumerate}[a)]
	\item For $(x,t) \in \overline{Q}_T$ 
	\begin{equation}
		\label{exist_cond1}
		\frac{\partial a_i(x,t,u,p)}{\partial p_j}\xi_i \xi_j \Bigg|_{p=0} \geq 0, \quad A(x,t,u,0)u \geq -b_1 u^2 - b_2
	\end{equation}
	 holds for arbitrary $u$ and $\xi = (\xi_1, \xi_2, \ldots, \xi_d)$.
	
	\item For $(x,t) \in \overline{Q}_T$, $|u| \leq M$ and arbitrary $p$ the functions $a_i(x,t,u,p)$ and $a(x,t,u,p)$ are continuous, the $a_i(x,t,u,p)$ are differentiable with respect to $x$, $u$ and $p$ and the following inequalities are satisfied
	\begin{equation}
	\label{exist_cond3}
		\begin{cases}
			\nu|\xi|^2 \leq \sum_{i,j=1}^d\frac{\partial a_i(x,t,u,p)}{\partial p_j}\xi_i \xi_j \leq \mu |\xi|^2 \\
			\sum_{i=1}^d \left(|a_i| + |\frac{\partial a_i}{\partial u}|\right)(1+|p|) + \sum_{i,j=1}^d |\frac{\partial a_i}{\partial x_j}| + |a| \leq \mu (1 + |p|)^2. 		
		\end{cases}
	\end{equation}
	where $\nu$ and $\mu$ are positive numbers.
	
	\item For $(x,t) \in \overline{Q}_T$, $|u| \leq M$ and $|p| \leq M_1$, the functions $a_i$, $a$, $\frac{\partial a_i}{\partial p_j}$, $\frac{\partial a_i}{\partial u}$, and $\frac{\partial a_i}{\partial x_i}$ are continuous functions satisfying a Hölder condition in $x$, $t$, $u$ and $p$ with exponents $\beta$, $\frac{\beta}{2}$, $\beta$ and $\beta$ respectively, for some $\beta \in (0,1]$.
	
	\item For $(x,t) \in \overline{Q}_T$, $|u| \leq M$ and $|p| \leq M_1$ the function $a(x,t,u,p)$ has the partial derivatives $\frac{\partial a}{\partial p_j}$ and $\frac{\partial a}{\partial u}$ and the functions $a_i$ and $a$ satisfy
	\begin{equation}
	\label{exist_cond4}
	\begin{split}
		&\Bigg| \frac{\partial a_i(x,t,u,p)}{\partial u}\Bigg|, \Bigg|\frac{a_i(x,t+h,u,p) - 		a_i(x,t,u,p)}{h}\Bigg|, \\
		&\Bigg|\frac{\partial a}{\partial p}\Bigg|, \Bigg|\frac{\partial a}{\partial u}\Bigg|, \Bigg|\frac{a(x,t+h,u,p) - a(x,t,u,p)}{h} \Bigg| \leq \varphi(x,t).
	\end{split}.
	\end{equation}
	Here $\varphi$ is required to satisfy
	\begin{equation*}
	\left( \int_0^T \left(\int_\Omega |\varphi(x,t)|^q dx\right)^\frac{r}{q} dt 			\right)^\frac{1}{r} < \infty,
	\end{equation*}
	i.e. $\varphi \in L^r(0,T: L^q(\Omega))$, where $q$ and $r$ satisfy
	\begin{equation*}
	\frac{1}{r} + \frac{d}{2q} = 1 - x_1
	\end{equation*}
	and
	\begin{equation*}
	\begin{cases}
		q\in \left[\frac{d}{2(1-x_1)},\infty\right], &r\in \left[\frac{1}{1-x_1},\infty \right], 0<x_1<1, \quad \text{ for } d \geq 2 \\
		q \in [1, \infty], &r \in \left[\frac{1}{1-x_1},\frac{2}{1-2x_1}\right], 0<x_1<\frac{1}{2}, \quad \text{ for } n=1
	\end{cases}
\end{equation*}
	
	\item $\psi(x,t) \in H^{2+\beta, 1+\beta/2}(\overline{Q}_T)$ and satisfy
	\begin{equation}
	\label{exist_cond5}
	\psi_t - \frac{d}{dx_i}a_i(x,t,\psi,\nabla\psi) + a(x,t,\psi,\nabla\psi) = 0.
	\end{equation}
	on $\partial \Omega \times \{t=0\}$.
	
	\item $\partial \Omega \in C^{2,\beta}$.
\end{enumerate}

Under these conditions there exists a unique solution of problem (\ref{lady_parabolic_eq})  with boundary condition (\ref{lady_bc}) from the class $H^{2+\beta, 1+\beta/2}(\overline{Q}_T)$. Moreover, this solution has derivatives $\partial_t \nabla u$ from $L^2(\overline{Q}_T)$.
\end{theorem}

\subsubsection{From general to specific: Existence for the GPME}
Now we'll consider theorem \ref{thm:classical_existence} in the setting of (\ref{HDP}) and under which conditions a solution is guaranteed to exist. The result is summarised in the following corollary.

\begin{cor}
\label{cor:GPME_existence}
Assume the following conditions are satisfied:
	\begin{enumerate}[a)]
		\item 	$\Phi: \mathbb{R} \to \mathbb{R}$ is in $C^2(\mathbb{R})$, satisfies (\ref{uniform_parabGPME}), and $\Phi$, $\Phi'$ and $\Phi''$ are all Lipschitz continuous.
		
		\item $f = f(x,t)$ is $C^1(Q_T)$, bounded and Lipschitz in both its variables.
		
		\item $u_0$ is in $C^2_c(\Omega)$; $u_0\in C^2(\Omega)$ and has compact support.
		
		\item $\partial \Omega$ is $C^{2+\beta}$-smooth.
	
	\end{enumerate}
Then (\ref{HDP}) has a unique solution $u\in C^{2}_1(Q_T)\cap C(\overline{Q}_T)$ for $T<\infty$. Moreover, this solution has derivatives $\partial_t \nabla u$ in $L^2(Q_T)$.
\end{cor}

\begin{proof}

The general porous medium equation (\ref{GPME}) we are concerned with can be written
\begin{equation*}
u_t - \nabla \cdot (\nabla \Phi(u) ) - f(x,t) = 0,
\end{equation*}
which is of the form (\ref{lady_parabolic_eq}) with
\begin{equation}
a_i(x,t,u,\nabla u) = \Phi'(u) u_{x_i}, \quad a(x,t,u,\nabla u) = -f(x,t).
\end{equation}


Let's go through each condition in theorem \ref{thm:classical_existence} and verify that each are met.

\begin{enumerate}[a)]
	\item We consider fullfilling (\ref{exist_cond1}). We have
	\begin{equation*}
		\frac{\partial a_i(x,t,u,p)}{\partial p_j} = \delta_{ij} \Phi'(u),
	\end{equation*}
	and so, since $\Phi'(u) \geq 0$ is assumed,
	\begin{equation*}
		\frac{\partial a_i(x,t,u,p)}{\partial p_j}\Bigg|_{p=0}\xi_i\xi_j = \Phi'(u)\xi_i^2 \geq 0.
	\end{equation*}
	
	In the general porous medium equation
	\begin{equation*}
		A(x,t,u,p) = -f(x,t) - \sum_{i=1}^d \Phi''(u)p_i,
	\end{equation*}
	which means $A(x,t,u,0) = -f(x,t)$. Thus, according to (\ref{exist_cond1}) we need constants $b_1$ and $b_2$ so that
	\begin{equation*}
		f(x,t)u \leq b_1 u^2 + b_2.
	\end{equation*}
	We are able to choose $b_1$ and $b_2$ since $f$ is continuous and bounded in $\overline{Q}_T$.
	
	\item We need $a_i(x,t,u,p)$ and $a(x,t,u,p)$ to be continuous. This translates to requiring that $\Phi'(u)$ and $f(x,t)$ are continuous. In addition $a_i(x,t,u,p)$ needs to be differentiable with respect to $x$, $t$, $u$ and $p$, which in our case means that $\Phi''(u)$ needs to be continuous.
	
	Considering the first inequality in (\ref{exist_cond3}) we have
	\begin{equation*}
		\sum_{i,j=1}^d \frac{\partial a_i(x,t,u,p)}{\partial p_j} \xi_i \xi_j = \sum_{i=1}^d \Phi'(u)\xi_i^2 =\Phi'(u)\xi^2,
	\end{equation*}
	which is satisfied because of (\ref{uniform_parabGPME}).
	
	After establishing that
	\begin{equation*}
		\begin{cases}
			a_i(x,t,u,p) &= \Phi'(u)p_i \\
			\frac{\partial a_i(x,t,u,p)}{\partial u} &= \Phi''(u)p_i \\
			\frac{\partial a_i(x,t,u,p)}{\partial x_j} &= 0,
		\end{cases}	
	\end{equation*}
	we put it into the left hand side of the second inequality of (\ref{exist_cond3}) as
	\begin{align*}
	&\sum_{i=1}^d\left( |a_i| + \left| \frac{\partial a_i}{\partial u}\right| \right)(1+|p|) + \sum_{i,j}^d\left| \frac{\partial a_i}{\partial x_j} \right| + |a| \\
	&= \sum_{i=1}^d\left( |\Phi'(u)| + |\Phi''(u)| \right)|p_i|(1+|p|) + |f(x,t)| \\
	&= (|\Phi'(u)| + |\Phi''(u)|)(1+|p|)\sum_{i=1}^d|p_i| + |f(x,t)|.
	\end{align*}
	By the equivalence of $l_p$-norms in finite dimensions $\sum_i |p_i| \leq C|p|$. Furthermore, we assume $\Phi''(u)$ to be continuous for $|u|\leq M$, then
	\begin{equation*}
	(|\Phi'(u)| + |\Phi''(u)|)(1+|p|)\sum_{i=1}^d|p_i| + |f(x,t)| \leq K(|p| + |p|^2) + |f(x,t)|,
	\end{equation*}
	where $K = C\cdot \text{sup}_{|u|\leq M}(|\Phi'(u)| + |\Phi''(u)|)$. Now we are almost done. All that remains is noticing that
	\begin{align*}
	K(|p| + |p|^2) + |f(x,t)| &\leq K(|p| + |p|^2) + ||f||_{\infty}(1+|p|) \\
	&\leq \text{max}\{ K, ||f||_\infty \} (1+|p|)^2.
	\end{align*}
	So we can set $\mu = \text{max}\{K, ||f||_\infty, \text{sup}_{|u|\leq M}\Phi'(u) \}$, and both inequalities in (\ref{exist_cond3}) are satisfied.
	
	\item To reiterate, we need the functions
	\begin{equation*}
		\begin{cases}
			a_i(x,t,u,p) &= \Phi'(u)p_i \\
			a(x,t,u,p) &= -f(x,t) \\
			\frac{\partial a_i(x,t,u,p)}{\partial p_j} &= \delta_{ij}\Phi'(u) \\
			\frac{\partial a_i(x,t,u,p)}{\partial u} &= \Phi''(u)p_i \\
			\frac{\partial a_i(x,t,u,p)}{\partial x_j} &= 0
		\end{cases}
	\end{equation*}
	to be Hölder continuous in $x$, $t$, $u$ and $p$ with exponents $\beta$, $\frac{\beta}{2}$, $\beta$ and $\beta$ respectively. This is satisfied for any $\beta \in (0,1]$ since $\Phi$, $\Phi'$, $\Phi''$ and $f$ are all assumed to be Lipschitz continuous.
	With $\beta = 1$, the requirement of theorem \ref{thm:classical_existence} is that $f$ is Hölder continuous with exponent $1/2$ in time, i.e.
	\begin{equation*}
	|f(x,t) - f(x,s)| \leq C|t-s|^\frac{1}{2}
	\end{equation*}
	for some constant $C$. Note that for bounded domains, Lipschitz continuity implies Hölder continuity with lower exponent. Therefore, this is satisfied with $f$ Lipschitz in time.
	
	\item This condition holds from $\frac{\partial a_i}{\partial u}$, $\frac{\partial a}{\partial t}$ (the rest of the expressions in the condition are zero), and since we're considering $Q_T$, with $T$ finite, constants are integrable.
	
	\item We of course have
	\begin{equation*}
		\begin{cases}
		\Psi(x,t) &= 0, \quad (x,t) \in \partial \Omega \times [0,T] \\
		\Psi(x,0) &= u_0(x), \quad x\in \Omega 
		\end{cases}
	\end{equation*}
	and so this condition isn't too hard to handle. First off, we should require that $u_0$ vanish at the boundary, and that it is sufficiently smooth. To be more precise, $u_0$, $\nabla u_0$ and $\Delta u_0$ are all continuous and are zero at the boundary. The condition (\ref{exist_cond5}) holds trivially, because  $\Psi\big|_{\partial \Omega \times \{t=0\}} = 0$.
	
	\item This condition holds trivially.
\end{enumerate}
All conditions of theorem (\ref{thm:classical_existence}) are met, which yields the conclusion that (\ref{HDP}) with the additional assumption of this corollary has a unique solution $u \in C^2_1(Q_T)\cap C(\overline{Q}_T)$.
\end{proof}

\setcounter{rem}{0}
\begin{rem}
We end this section on classical solutions with the remark that the assumptions in corollary \ref{cor:GPME_existence} is a subset of the assumptions for lemma \ref{cor:l1-contractivity_classical}. Therefore, two solutions $u$ and $\hat{u}$ aquired from corollary \ref{cor:GPME_existence} will also satisfy (\ref{l1_contractivity}), i.e. $L^1$-stability. Hence, The homogeneous Dirichlet problem (\ref{HDP}) with the assumptions in corollary \ref{cor:GPME_existence} is well-posed.
\end{rem}


%%%%%%%%%%%%%%%%%%%%%%%%%%%%%%%%%%%%%%%%%%%%%%%%%%%%%%%
%				WEAK SOLUTIONS
%%%%%%%%%%%%%%%%%%%%%%%%%%%%%%%%%%%%%%%%%%%%%%%%%%%%%%%
\newpage
\section{Weak solutions}
In the previous sections we saw what assumptions needs to be made for the homogeneous Dirichlet problem to be well posed. Allthough a very nice result, the assumptions on the data are far too strict. Especially degenerate problems, where $\Phi$ is only increasing and not stricly so, are not valid for this existence result. In addition the assumptions on smoothness are often violated in applications. This heuristic will now lead us into a discussion of weak solutions, a more general sense of solution. 

In this section we will aim to prove well-posedness of weak solutions, and what sort of assumptions we'll need to put on the data to ensure it, well-posedness that is. Again, this follows \citep[Ch. 5]{vazquez2007porous} quite closely. However, the proofs are worked out in much greater detail than presented there. We will, unless stated otherwise, assume that $\Phi$ is a strictly increasing function.

Let's just jump off with the definition of weak solutions that we will be using in this text.
\begin{mydef}[Weak solutions]
A locally integrable function $u$ defined in $Q_T$ is said to be a \textbf{weak solution} of (\ref{HDP}) if
\begin{enumerate}[i)]
	\item $u \in L^1(Q_T)$ and $\Phi(u) \in L^1(0,T : W_0^{1,1}(\Omega))$;
	\item $u$ satisfies
		\begin{equation}
		\label{weak}
		\iint_{Q_T} \{\nabla\Phi(u)\cdot \nabla\eta - u\partial_t \eta \}dxdt = \int_\Omega u_0(x)\eta(x,0)dx + \iint_{Q_T}f\eta dxdt
		\end{equation}
for any $\eta \in C^1(\overline{Q}_T)$ that vanishes on $\partial \Omega \times [0,T)$ and for $t=T$.
\end{enumerate}
\end{mydef}

\setcounter{obs}{0}
\begin{obs}
With $u_0 \in L^1(\Omega)$ and $f\in L^1(Q_T)$, all the integrals in (\ref{weak}) make sense.
\end{obs}
\begin{obs}
Classical solutions are also weak solutions. This is fairly straight forward to see. Let $u$ be a smooth solutions and multiply (\ref{GPME}) with a test function $\eta$. Then, when integrating over $Q_T$
\begin{equation*}
\iint_{Q_T}\partial_t u \eta dxdt = \iint_{Q_T}\Delta\Phi(u) \eta + f\eta dxdt.
\end{equation*}
Using integration by parts in time on the left hand side and in space on the first term on the right hand side yields
\begin{equation*}
\int_\Omega u\eta dx \Bigg|_{t=0}^T - \iint_{Q_T}u\partial_t \eta dxdt = -\iint_{Q_T}\nabla \Phi(u) \cdot \nabla \eta dxdt + \iint_{Q_T}f\eta dxdt,
\end{equation*}
and now all that remains is rearranging the terms to get (\ref{weak}).
\end{obs}

%%%%%%%%%%%%%%%%%%%%%%%%%%%%%%%%%%%%%%%%%%%%%%%%%%%%%%%
%				UNIQUENESS
%%%%%%%%%%%%%%%%%%%%%%%%%%%%%%%%%%%%%%%%%%%%%%%%%%%%%%%
\subsection{Uniqueness}
The first result we'll prove for weak solutions is uniqueness under some additional assumptions. The proof can be found in \citep{vazquez2007porous}, but here worked out in greater detail. The result is also somewhat improved upon by here admitting $\Phi$ to be nondecreasing, rather than strictly increasing.
\begin{theorem}[Uniqueness]
\label{thm:uniqueness}
	Assuming in addition that $u \in L^2(Q_T)$ and $\Phi(u) \in L^2(0,T : H_0^1(\Omega))$, (\ref{HDP}) has at most one weak solution.
\end{theorem}

\begin{proof}
This is a pretty neat proof using a smart choice of $\eta$. So let's assume $u_1$ and $u_2$ are weak solutions. Then, when putting $u_1$ and $u_2$ into (\ref{weak}) and subtracting one from the other, we get
\begin{equation}
\label{weak_diff}
\iint_{Q_T}\nabla(\Phi(u_2) - \Phi(u_1)) \cdot \nabla \eta - (u_2 - u_1)\partial_t \eta dxdt = 0.
\end{equation}
Now the inspired move is choosing
\begin{equation*}
\eta(x,t) = \begin{cases}
		\int_t^T(\Phi(u_2) - \Phi(u_1))ds, \quad &\text{ if } 0 < t < T \\
		0 \quad &\text{ if } t\geq T.
	\end{cases}
\end{equation*}
We then have
\begin{equation*}
\begin{cases}
	\partial_t \eta &= -(\Phi(u_2) - \Phi(u_1) ) \\
	\nabla \eta &= \int_t^T\nabla (\Phi(u_2) - \Phi(u_1))ds,
\end{cases}
\end{equation*}
where the temporal differentiation is valid using the First Fundamentel Theorem of Calculus and that $\Phi(u_i) \in L^1(0,T : W_0^{1,1}(\Omega))$ by the definition of weak soultion. For the validity of the spatial derivative, confer appendix \ref{sec:derivative}.

When putting these derivatives into the above equality we end up with
\begin{equation}
\label{unique}
\begin{aligned}
	&\iint_{Q_T}(\Phi(u_2) - \Phi(u_1))(u_2 - u_1)dxdt \\ 
	&+ \iint_{Q_T}\nabla(\Phi(u_2(t)) - \Phi(u_1(t))) \cdot \left(\int_t^T 					\nabla(\Phi(u_2(s)) - \Phi(u_1(s))ds\right)dtdx = 0.
\end{aligned}
\end{equation}
Let's take these in order:
\begin{enumerate}[i)]
	\item In the first integral we notice that $u_2 \geq u_1 \Rightarrow \Phi(u_2) \geq \Phi(u_1)$ and vice versa, so this integral is nonnegative.
	
	\item For the second integral it is helpful to first only consider integration of the temporal dimensions, leaving us with
	\begin{align*}
	&\int_0^T\int_t^T \nabla(\Phi(u_2(t)) - \Phi(u_1(t)))\cdot \nabla(\Phi(u_2(s))-\Phi(u_1(s)))dsdt \\
	&= \int_0^T\int_0^s\nabla(\Phi(u_2(t)) - \Phi(u_1(t)))\cdot \nabla(\Phi(u_2(s))-\Phi(u_1(s)))dtds.
	\end{align*}
	Due to the symmetry about the line $s=t$, we have
	\begin{align*}
	&\int_0^T\int_0^s\nabla(\Phi(u_2(t)) - \Phi(u_1(t)))\cdot \nabla(\Phi(u_2(s))-\Phi(u_1(s)))dtds \\
	&= \frac{1}{2}\int_0^T \int_0^T \nabla(\Phi(u_2(t)) - \Phi(u_1(t)))\cdot \nabla(\Phi(u_2(s))-\Phi(u_1(s)))dtds \\
	&= \frac{1}{2}\left( \int_0^T \nabla(\Phi(u_2)-\Phi(u_1))dt\right)^2,
	\end{align*}
	which is also nonnegative, and therefore the second integral in (\ref{unique}) must also be nonnegative.
\end{enumerate}
From this we may conclude that both integral terms in (\ref{unique}) are zero, which in turn implies, by the first integral that $(\Phi(u_2)-\Phi(u_1))(u_2 - u_1) = 0$ a.e. in $Q_T$. Actually, we have that $\Phi(u_2) - \Phi(u_1) = 0$ a.e., because wherever $u_2 = u_1$ it is trivially true, and wherever $ u_2 \neq u_1$ it holds from the first integral being zero. Thus, $ \nabla (\Phi(u_2) - \Phi(u_1) ) = 0$ a.e. and putting this into (\ref{weak_diff}) we get that
\begin{equation*}
\iint_{Q_T}(u_2 - u_1)\partial_t \eta dx dt = 0.
\end{equation*}
This should hold for any test function $\eta$, and so the weak derivative of $u_2 - u_1$ with respect to time is zero. Because $u_2$ and $u_1$ have identical initial condition $u_0$ we conclude that $u_2 = u_1$ almost everywhere.
\end{proof}

%%%%%%%%%%%%%%%%%%%%%%%%%%%%%%%%%%%%%%%%%%
%			EXISTENCE
%%%%%%%%%%%%%%%%%%%%%%%%%%%%%%%%%%%%%%%%%%
\subsection{Existence}
As the assumptions in theorem \ref{thm:uniqueness} may hint at, some additional assumptions on $u$ and $\Phi(u)$ needs to be put to ensure that the homogeneous Dirichlet problem is well posed in the weak setting. There are some alternatives, and of course, any further analysis will be informed by what sort of additional condition we want a weak solution $u$ to fulfill. The one we will be using here is guided by the notion that in a physical system the energy should be controlled, and is therefore called a weak energy solution. 

Recall our definition of $\Psi$ in (\ref{phi_primitive}) as the primitive of $\Phi$,
and we define the function space $L_\Psi(\Omega) = \{ u\in L^1(\Omega): \Psi(u)\in L^1(\Omega)\}$. This is with the norm
\begin{equation}
||u||_{L_\Psi (\Omega)} = ||u||_{L^1(\Omega)} + ||\Psi(u)||_{L^1(\Omega)}.
\end{equation}
With this in mind we can define energy solutions in the following way:
\begin{mydef}[Energy solution]
We say that $u$ is a \textbf{weak energy solution} if it's a weak solution, $u_0 \in L_\Psi(\Omega)$, and the energy inequality
\begin{equation}
\label{energy_ineq}
\iint_{Q_T} |\nabla \Phi(u)|^2 dxdt + \int_\Omega \Psi(u(x,T))dx \leq \int_\Omega \Psi(u_0)dx + \iint_{Q_T}f\Phi(u)dxdt
\end{equation}
holds.
\end{mydef}
\setcounter{obs}{0}
\begin{obs}
If $f\in L^2(Q_T)$ we can do the same trick on the right hand side as we did in deriving (\ref{spatial_derivative_bound_w_poincare}), to get that $\Phi(u) \in L^2(0,T: H^1_0(\Omega))$.
\end{obs}

We will prove the existence of weak energy solutions for the homogeneous Dirichlet problem (\ref{HDP}) under the following assumptions:
\begin{assumption}
\label{ass:Phi_general}
$\Phi$ is absolutely continuous and strictly increasing. $\Phi(0) = 0$ and $\Phi(\pm \infty) = \pm \infty$.
\end{assumption}

\begin{assumption}
\label{ass:f}
$f \in L^2(Q_T)$.
\end{assumption}

\begin{assumption}
\label{ass:u_0}
$u_0 \in L_{\Psi}(\Omega)$.
\end{assumption}

On our way to proving existence under these assumptions, we will also need some additional assumptions that will subsequently be lifted. These are:
\begin{assumption}
\label{ass:f_ft_u_0_bounded}
$f$, $\partial_t f$ are bounded in $Q_T$, and $u_0$ is bounded in $\Omega$.
\end{assumption}

\begin{assumption}
\label{ass:Phi_lipschitz}
$\Phi$ is locally Lipschitz.
\end{assumption}

\begin{assumption}
\label{ass:smooth_boundary}
$\partial \Omega$ is $C^{2,\alpha}$ for some $\alpha \in (0,1)$.
\end{assumption}


Our main existence theorem reads:
\begin{theorem}
\label{thm:weak_existence}
Under assumptions \ref{ass:Phi_general}-\ref{ass:u_0} the homogeneous Dirichlet problem (\ref{HDP}) has a weak solution, $u$, defined in any finite time interval. In addition $u \in L^\infty(0,T:L_\Psi(\Omega))$ and $\Phi(u) \in L^2(0,T:H_0^1(\Omega))$, and the energy inequality (\ref{energy_ineq}) holds.
\end{theorem}

The proof will use an approximate problem which admits a unique solution from theorem \ref{thm:classical_existence}, and then employ the Rellich compactness argument (cf. appendix \ref{sec:compactness}) to get a limit which hopefully will be a weak energy solution.

We will not prove this in one big chunk, but rather split it off into smaller lemmas, and as already stated, some attention will also be paid to the $L^1$-contractivity.
\subsubsection{Preliminaries}
\label{sec:approximations}
To set ourselves up in a position where we can make use of theorem \ref{thm:classical_existence}, or rather corollary \ref{cor:GPME_existence}, we need to approximate $\Phi$, $u_0$, and $f$.

We approximate $\Phi$ by the smooth approximations. To ensure suffient smoothness we define
\begin{equation*}
\Phi_n'(s) = \left( \Phi' + \frac{1}{n}\right)*\varphi_{\frac{1}{n}},
\end{equation*}
where $\varphi_{\frac{1}{n}}(s) = n\varphi(ns)$ and $\varphi$ is the standard mollifier. Then, we can approximate $\Phi$ as
\begin{align*}
\Phi_n(u) &= \int_0^u \int_\mathbb{R}\left( \Phi'(s-y) + \frac{1}{n}\right)\varphi_{\frac{1}{n}}(y)dyds \\
&= \int_\mathbb{R}\varphi_{\frac{1}{n}}(y) \int_0^u \left( \Phi'(s-y) + \frac{1}{n}\right)dsdy \\
&= \int_\mathbb{R}\varphi_{\frac{1}{n}}(y) \left( \Phi(u-y) - \Phi(-y) + \frac{u}{n}\right)dy \\
&= \Phi * \varphi_{\frac{1}{n}}(u) - \Phi * \varphi_{\frac{1}{n}}(0) + \frac{u}{n}.
\end{align*}
Further discussion of $\Phi$ and its sequence of approximations has been relegated to appendix \ref{app:Phi} for sake of continuity.


We also let $u_{0,n}\in C^\infty_c(\Omega)$ be a smooth approximation of $u_0$, and converging in $L_\Psi(\Omega)$.
This construction can be done as follows: First, define $\tilde{u}_n = \mathrm{max}\{ \mathrm{min}\{u_0, n\}, -n\}$, so $u_0$ is truncated at $\pm n$. Then let $(V_n)$ be a sequence of simply connected, compact subsets so that $K_n \subset K_{n+1}$ for all $n$ as well as $\cup_{n\in \mathbb{N}}K_n = \Omega$, and we require $\mathrm{dist}(K_n,\partial \Omega) \geq \frac{2}{n}$. Then we set $u_{0,n}= (\chi_{K_n}\tilde{u}_n)*\varphi_{\frac{1}{n}}$, where $\chi_{K_n}$ is the characteristic function for $K_n$ and $\varphi_{\frac{1}{n}}$ is the scaled standard mollifier in $\mathbb{R}^d$. The requirement on the distance between $V_n$ and $\partial \Omega$ we see is to ensure that $u_{0,n}$ and its derivatives vanish at the boundary. See that this way of approximating $u_0$ ensures that $||u_{0,n}||_{L^1(\Omega)} \leq ||u_0||_{L^1(\Omega)}$ as well as $||u_{0,n}||_{L^\infty(\Omega)} \leq ||u_0||_{L^\infty(\Omega)}$ for all $n$ by Young's inequality for convolutions.

We define the approximations of $f$ in a similar manner: Let $\tilde{f}_n = \mathrm{max}\{ \mathrm{min}\{f,n \}, -n\}$, and then define $f_n = \tilde{f}_n * \varphi_{\frac{1}{n}}$.
Notice that by Young's inequality $||f_n||_{L^2(Q_T)} \leq ||\tilde{f}_n||_{L^2(Q_T)} \leq ||f||_{L^2(Q_T)}$.

With these approximations at hand we can define an approximate problem to (\ref{HDP}), by
 \begin{equation}
	\label{approximate_problem}
	\begin{cases}
	\partial_t u_n &= \Delta\Phi_n(u_n) + f_n, \quad \text{ in } Q_T \\
	u_n(x,0) &= u_{0,n}(x),	\quad \text{ on } \overline{\Omega} \\
	u_n(x,t) &= 0, \quad \text{ on }  \partial \Omega \times (0,T).
	\end{cases}
\end{equation}


The main strategy in proving Theorem \ref{thm:weak_existence} is to show the right type of convergence for the solutions of (\ref{approximate_problem}). 
%%%%%%%%%%%%%%%%%%%%%%%%%%%%%%%%%%%%%%%%%%%%%%%%%%%%%%%%%%%%%%%%
%		PROOF OF EXISTENCE PART I:							  
%%%%%%%%%%%%%%%%%%%%%%%%%%%%%%%%%%%%%%%%%%%%%%%%%%%%%%%%%%%%%%%%
\subsubsection{With all assumptions in place}
The goal of this section will be to prove existence of a weak energy solution under all the assumptions \ref{ass:Phi_general}-\ref{ass:smooth_boundary}, but before doing that some preliminary facts should be established.

\begin{lemma}
\label{prop:classical_sols_seq}
With assumptions \ref{ass:Phi_general}-\ref{ass:u_0}, \ref{ass:f_ft_u_0_bounded} and \ref{ass:smooth_boundary} as well as the approximations introduced in section \ref{sec:approximations}, we can identify a sequence $(u_n)$ of classical solutions of (\ref{approximate_problem}). Moreover, there is a constant $M>0$, so that $|u_n(x,t)| \leq M$ in all of $Q_T$ and for all $n$.
\end{lemma}

\begin{proof}
With the provided approximations, the approximate problem 
will for each $n$ permit a classical solution by corollary \ref{cor:GPME_existence}. Then we simply define the sequence $(u_n) \subset C^2_1(Q_T)\cap C(\overline{Q}_T)$ as $u_n$ being for each $n$ the unique classical solution of (\ref{approximate_problem}).

In addition we can from the corollary   \ref{cor:Linf_bound_classical} see that
	\begin{align*}
	||u_n||_{L^\infty(Q_T)} &\leq ||u_{0,n}||_{L^\infty(\Omega)} + T||f_n||_{L^\infty(Q_T)} \\
	&\leq ||u_0||_{L^\infty(\Omega)} + T||f||_{L^\infty(Q_T)} = M.
	\end{align*}
	A bound independent of $n$.  	
\end{proof}

Define now for each $n$ $w_n = \Phi_n(u_n)$. It's the sequence $(w_n)$ we will use the Rellich-Kondrachov compactness argument, which motivates the following two lemmas.

\begin{lemma}
\label{prop:spatial_boundedness}
With the same assumptions as in Lemma \ref{prop:classical_sols_seq},
the sequences $(w_n)$ and $(\nabla w_n)$ are uniformly bounded in $L^2(Q_T)$. In other words $(w_n)$ is uniformly bounded $L^2(0,T: H^1_0(\Omega))$. Moreover, there is some $K >0$, depending only on $M$ from Lemma \ref{prop:classical_sols_seq} and $\Phi$, so that
\begin{equation}
\label{gradwn_bound_uniform}
||\nabla w_n||_{L^2(Q_T)} \leq K||u_0||_{L^1(\Omega)} + ||f||_{L^2(Q_T)}.
\end{equation}
for all $n$.
\end{lemma}
\begin{proof}
Since $u_n$ are classical solutions, we use (\ref{spatial_derivative_bound_w_poincare}) to get that
	\begin{equation}
	\label{exist_proof_spatial}
	\frac{1}{2}\iint_{Q_T}|\nabla w_n|^2 dxdt + \int_\Omega \Psi_n(u_n(T))dx  \leq \int_\Omega \Psi_n(u_{0,n})dx + C\iint_{Q_T}f_n^2 dxdt
	\end{equation}
	holds for each $n$.


	All that remains is to bound the right hand side (\ref{exist_proof_spatial}) independent of $n$, and we're done.To that end we notice that
	\begin{equation*}
	\iint_{Q_T}f_n^2dxdt \leq \iint_{Q_T} f^2 dxdt,
	\end{equation*}
	by our assumption on the approximation.
	
	For the $\Psi$-term on the right hand side we start by recalling that we define
	\begin{equation*}
	\Psi_n(u) = \int_0^{u}\Phi_n(s)ds,
	\end{equation*}
	from which it is easy to see that $\Psi_n(u) \geq 0$ for every $u \in \mathbb{R}$. It is also very much worthwhile to see that because $\Phi_n$ is monotone for every $n$ we have the estimate
	\begin{equation*}
	|\Psi_n(u)| \leq |\Phi_n(u)u|.
	\end{equation*}
	This leads to
	\begin{equation*}
	\int_\Omega \Psi_n(u_{0,n})dx \leq \int_\Omega |\Phi_n(u_{0,n})u_{0,n}|dx	
	\end{equation*}

	
	Now we derive an estimate on the $\Phi_n$-part in terms of $\Phi$ and $u_0$. A crude estimate will be sufficient, and we have
	\begin{align*}
	|\Phi_n(u) - \Phi(u)| &= \left|\frac{u}{n} + \int_\mathbb{R}\left( \Phi(u-y) - \Phi(u) - \Phi(y)\right)\varphi_{\frac{1}{n}}(y)dy \right| \\
	&\leq \frac{|u|}{n} + \underset{y\in [-1/n, 1/n]}{\mathrm{sup}}|\Phi(u-y) - \Phi(u)| + \mathrm{max}\{|\Phi(-1/n)|, |\Phi(1/n)|\} \\
	&\leq |u| + \underset{y\in [-1, 1]}{\mathrm{sup}}|\Phi(u-y) - \Phi(u)| +\mathrm{max}\{|\Phi(-1)|, |\Phi(1)|\},
	\end{align*}
	where we have used that $\mathrm{supp}(\varphi_{\frac{1}{n}})\subset [-1/n, 1/n]$ as well as that $\Phi$ is monotone. This, when considering $u_{0,n}$, we have
	\begin{align*}
	|\Phi_n(u_{0,n})| &\leq |\Phi(u_{0,n})| + |u_{0,n}| + \underset{y\in [-1, 1]}{\mathrm{sup}}|\Phi(u_{0,n}-y) - \Phi(u_{0,n})| +\mathrm{max}\{|\Phi(-1)|, |\Phi(1)|\} \\
	& \leq \underset{|s| \leq ||u_0||_\infty}{\mathrm{sup}}|\Phi(s)| + ||u_0||_{\infty}\\
	& + \underset{\underset{y\in[-1/n,1/n]}{|s| \leq ||u_0||_{\infty}}}{\mathrm{sup}}|\Phi(s-y) - \Phi(s)| + \mathrm{max}\{|\Phi(-1)|, |\Phi(1)|\} \\
	&=: K < \infty.
	\end{align*}
	In the last step we also made repeated use of $|u_{0,n}| \leq ||u_0||_\infty$.
	All terms are bounded since $u_0$ is bounded. When also using that $||u_{0,n}||_{L^1(\Omega)} \leq ||u_0||_{L^1(\Omega)}$ we finally get the estimate
	\begin{align*}
	\int_\Omega \Psi_n(u_{0,n})dx &\leq K ||u_{0,n}||_{L^1(\Omega)} \\
	&\leq K ||u_0||_{L^1(\Omega)}
	\end{align*}
	which is finite from $u_0 \in L^1(\Omega)$. Most importantly, this shows that the left hand side of (\ref{exist_proof_spatial}) is bounded independently of $n$, which immediately implies that $\nabla w_n$ is a bounded sequence in $L^2(Q_T)$. Notice also that from the Poincaré inequality,
	\begin{equation*}
	\iint_{Q_T}w_n^2 dxdt \leq C\iint_{Q_T} |\nabla w_n|^2dxdt,	
	\end{equation*}
	$w_n$ is also uniformly bounded in $L^2(Q_T)$.
\end{proof}

\begin{lemma}
\label{prop:temporal_bound}
With the same assumptions as in Lemma \ref{prop:classical_sols_seq}, the sequence $(\Phi'_n(u_n) |\partial_t u_n|^2)$ is uniformly bounded in $L^1(Q_T^\tau)$ for all $0 < \tau < \frac{T}{2}$.
\end{lemma}

\begin{proof}
From Lemma \ref{lem:temporal_bound_classical} we have for each $n$ and every cutoff function $\zeta\in C^1([0,T], [0,1])$
	\begin{equation}
	\iint_{Q_T} \zeta \Phi'_n(u_n) |\partial_t u_n|^2 dxdt = \iint_{Q_T} \left\{ \frac{\partial_t \zeta}{2}|\nabla w_n|^2 - \partial_t(\zeta f_n) w_n \right\} dxdt.
	\end{equation}
	
	Take now $0<\tau < \frac{T}{2}$, and take $\zeta(t)=1$ for $t\in [\tau, T-\tau]$. We define $Q_T^\tau = \Omega \times (\tau, T - \tau)$, then
	\begin{align*}
	\iint_{Q_T^\tau}\Phi'_n(u_n)|\partial_t u_n|^2 dxdt & \leq \iint_{Q_T}\zeta\Phi'_n(u_n)|\partial_t u_n|^2 dxdt \\
	&= \iint_{Q_T} \left\{ \frac{\partial_t \zeta}{2}|\nabla w_n|^2 - \partial_t(\zeta f_n) w_n \right\} dxdt.
	\end{align*}
	From the uniform boundedness of $w_n$ and $\nabla w_n$ in $L^2(Q_T)$ as well as the boundedness of $\partial_t f$ enables us to deduce from the above inequality that $\Phi'_n(u_n)|\partial_t u_n|^2$ is uniformly bounded in $L^1(Q_T^\tau)$.
\end{proof}

with these facts in place, we are in a position where we can start deriving some limit points of the various sequences, and as such get a candidate for teh weak solution.

\begin{lemma}
\label{prop:weak_convergence_w_n}
With the same assumptions as in proposition \ref{prop:classical_sols_seq},
$w_n$ contains a subsequence $(w_{n_j})$ converging almost everywhere to some $w \in L^2(Q_T)$. The convergence happens also weakly in $L^2(Q_T)$.
\end{lemma}

\begin{proof}
From Lemma \ref{prop:spatial_boundedness} $w_n$ and $\nabla w_n$ are uniformly bounded in $L^2(Q_T)$.

Notice that we have from Lemma \ref{prop:classical_sols_seq} that $||u_n||_{L^\infty(Q_T)} \leq M$ for all $n$. By assumption $\Phi$ is locally Lipschitz, and so there is a constant $L$ so that $\Phi'(s) \leq L$ for all $s \in [-M, M]$, and from the definition of $\Phi_n$ we then have taht $\Phi'_n(s) \leq L+1$ for all $s\in [-M,M]$ and all $n$. Hence $|\partial_t w_n|^2 \leq  (L+1)\Phi'_n(u_n)|\partial_t u_n|^2$, and by Lemma \ref{prop:temporal_bound} we have that $\partial_t w_n$ is uniformly bounded in $L^2(Q_T^\tau)$ for $\tau > 0$.As a consequence $w_n$ is uniformly bounded in $H^1(Q_T^\tau)$, for all $\tau > 0$.

 The fact that we were only able to bound $\partial_t w_n$ in $L^2(Q_T^\tau)$ causes some problems in using the Rellich-Kondrachov theorem to get a converging subsequence of $w_n$. We'll remedy this by the following diagonal argument:
	Let $(\tau_m)$ be a positive, decreasing sequence converging to $0$.
	\begin{enumerate}[i)]
	\item Consider $Q_T^{\tau_1}$. Then $w_n$ is bounded sequence in $H^1(Q_T^{\tau_1})$, and we use theorem \ref{Rellich} to get a subsequence $(w_{n_j}^1)$ that converges strongly to some $\tilde{w}_1$ in $L^2(Q_T^{\tau_1})$.
	
	\item $w_{n_j}^1$ is uniformly bounded in $H^1(Q_T^{\tau_2})$, and we use theorem \ref{Rellich} again to get a subsequence $(w_{n_j}^2)$ that converges to $\tilde{w}_2$. We have that $\tilde{w}_2 = \tilde{w}_1$ in $L^2(Q_T^{\tau_1})$.
	
	\item We proceed in this manner and for each $m \in \mathbb{R}$. $w_{n_j}^{m-1}$ is uniformly bounded in $H^1(Q_T^{\tau_m})$, so we use theorem \ref{Rellich} to get another subsequence $(w_{n_j}^m)$ that converges to $\tilde{w}_m$ in $L^2(Q_T^{\tau_m})$.
	
	\item We define the subsequence $(w_{n_j})$ by $w_{n_j} = w_{n_j}^j$, which will converge to some $w$ in $L^2(Q_T^\tau)$ for every $\tau > 0$, and hence in $L^2_{loc}(Q_T)$.

\end{enumerate}	 
	To ensure that we get a subsequence that converges almost everywhere to $w$, we need to make use of an additional diagonal argument on $w_{n_j}$. To that end, let $A_n$ be a sequence of compact subsets of $Q_T$, so that $A_k \subset A_{k+1}$ for every $k$, and $\cup_{k \in \mathbb{N}}A_k = Q_T$. We proceed as follows:
	\begin{enumerate}[i)]
		\item There is a subsequence $w_{n_j}^1$ that converges pointwise almost everwhere to $w$ in $A_1$. Denote by $N_1$, the null set for which pointwise convergence isn't guaranteed.
		\item Take a subsequence $w_{n_j}^1$ that converges almost everywhere to $w$ in $A_2$. Call this subsequence $w_{n_j}^2$, and $N_2$ the null set for which convergence doesn't hold.
		
		\item Proceed in this manner for each $k$ to get subsequences $w_{n_j}^k$ that converge everywhere in $A_k \setminus N_k$ to $w$.

		\item Finally define the subsequence $w_{n_j} = w_{n_j}^j$.	
	\end{enumerate}
	This final subsequence will converge almost everywhere in $Q_T$. To see this, consider any $(x,t) \in Q_T \setminus \left( \cup_{k \in \mathbb{N}}N_k \right)$. Then there is some $k$ so that $(x,t) \in A_k\setminus N_k$ which implies that $w_{n_j}(x,t) \to w(x,t)$. As a consequence the set where $w_{n_j}$ doesn't converge must be a subset of $\cup_{k\in \mathbb{N}} N_k$, and by the countable subadditivity of the Lebesgue measure, this has measure $0$.

	To determine that $w_{n_j}$ converges weakly to $w$ in $L^2(Q_T)$ we first note that every subsequence of $w_{n_j}$ is bounded in $L^2(Q_T)$ and by Banach-Alaoglu's theorem and that $L^2(Q_T)$ is a Hilbert deduce that the subsequence has a subsequence that converges weakly to some $v$. If we can prove that $v=w$ in $L^2(Q_T)$, then $w_{n_j}$ converges weakly to $w$, since every subsequence has a weakly convergent subsequence with the same limit.
	
	So assume that $v\neq w$, then there is an $\epsilon > 0$ and a subset $E\subset Q_T$ with positive measure where $|v-w| \geq \epsilon$. We can without loss of generality assume that $v > w$ in $E$.
	
	By the pointwise convergence of $w_{n_j}$ to $w$ and Egorov's theorem \citep[p. 120]{weiss1999course}	
	there is a measurable set $B\subset E$ with positive measure over which $w_{n_j}$ converges uniformly to $w$. Therefore, for any $\delta > 0$ and sufficiently large $n$,
	\begin{equation*}
	\iint_B |w_{n_j}-w|dxdt \leq \delta |B|,
	\end{equation*}
	where $|B|>0$ is the measure of $B$. We conclude that
	\begin{equation*}
	\iint_B w_{n_j}dxdt \to \iint_B wdxdt.
	\end{equation*}
	
	However, since $B$ is measurable the characteristic function $\chi_B \in L^2(Q_T)$, and by the weak convergence of $w_{n_j}$ to $v$ we have
	\begin{equation*}
	\iint_B w_{n_j}dxdt \to \iint_B vdxdt \geq \iint_B wdxdt + \epsilon|B|,
	\end{equation*}
	which is a contradiction. We conclude that $v=w$, and since this is true for any weakly convergent subsequence of $w_{n_j}$, this implies that $w_{n_j}$ converges weakly to $w$ in $L^2(Q_T)$.
\end{proof}

We can get a similar convergence result for $\nabla w_{n_j}$:
\begin{lemma}
\label{prop:gradw_weak_convergence}
With the same assumptions as in proposition \ref{prop:classical_sols_seq},
$\nabla w_{n_j}$ converges weakly to $\nabla w$ in $L^2(Q_T)$.
\end{lemma}
\begin{proof}
First off we note that $\nabla w_{n_j}$ converges to $\nabla w$ in the distributional sense. Indeed, the distributional gradient of $w_{n_j}$ is defined by
	\begin{equation}
	\iint_{Q_T}\nabla w_{n_j}\cdot \psi dxdt = -\iint_{Q_T} w_{n_j} \nabla \cdot \psi dxdt ,
	\end{equation}
	which is to hold for every $\psi \in C^\infty_c(Q_T)$. Since $w_{n_j}$ converges weakly to $w$
	\begin{align*}
	\iint_{Q_T}\nabla w_{n_j} \cdot \psi dxdt &= -\iint_{Q_T} w_{n_j} \nabla \cdot \psi dxdt \\
	&\to -\iint_{Q_T} w \nabla \cdot \psi dxdt \\
	&= \iint_{Q_T} \nabla w \cdot \psi dxdt,
	\end{align*}
	where the last inequality is by definition of the distributional derivative. From this we deduce that $\nabla w_{n_j}$ converges to $\nabla w$ in the distributional sense.
	
	Notice now that since $\nabla w_{n_j}$ is a bounded sequence in $L^2(Q_T)$ we can use Banach-Alaoglu's to say that every subsequence of $\nabla w_{n_j}$ has a weakly converging subsequence converging to some limit $v$. Since weak convergence implies convergence in the distributional sense and distributional limits are unique, we have that $v=\nabla w$. So every subsequence og $\nabla w_{n_j}$ has a subsequence that converges weakly to $\nabla w$, which implies that $\nabla w_{n_j}$ converges weakly to $\nabla w$ in $L^2(Q_T)$.
	
		
	 Lastly, note that with $w, \nabla w \in L^2(Q_T)$, $w \in L^2(0,T: H^1(\Omega))$.
\end{proof}

With this list of propositions we are in a position to prove our first existence Lemma, which reads
\begin{lemma}
\label{lem:existence_part1}
Theorem \ref{thm:weak_existence} holds with the additional assumptions \ref{ass:f_ft_u_0_bounded} through \ref{ass:smooth_boundary}.
\end{lemma}

\begin{proof}
With Lemma \ref{prop:classical_sols_seq} we get a sequence of classical solutions $(u_n)$, uniformly bounded by some $M$. Lemmas \ref{prop:weak_convergence_w_n} and \ref{prop:gradw_weak_convergence} provides a subsequence of $w_n = \Phi_n(u_n)$, $(w_{n_j})$, that converges pointwise almost everywhere to some $w$. In addition $w_{n_j}$ and $\nabla w_{n_j}$ converges weakly in $L^2(Q_T)$ to $w$ and $\nabla w$ respectively. What remains to establish, is a candidate solution $u$, and show that this is a weak solution that satisfies (\ref{energy_ineq}).

\begin{description}
	\item[Convergence of $u_{n_j}$:] Both $\Phi$ and its approximations $\Phi_n$ all admit a continuous inverse (cf. appendix \ref{app:Phi}), and so
	\begin{equation*}
	u_{n_j} = \Phi_{n_j}^{-1}(w_{n_j}).
	\end{equation*}
	We now simply have by the triangle inequality
	\begin{align*}
	|u_{n_j} - \Phi^{-1}(w)| &= |\Phi_{n_j}^{-1}(w_{n_j}) - \Phi^{-1}(w)| \\
		&\leq |\Phi_{n_j}^{-1}(w_{n_j}) - \Phi^{-1}(w_{n_j})| + |\Phi^{-1}(w_{n_j}) - \Phi^{-1}(w)|.
	\end{align*}
	 We have that $w_{n_j}$ is a bounded sequence, seeing as $u_{n_j}$ is bounded. Hence we can use proposition \ref{prop:inverse_uniform} to make the first term in the above inequality arbitrarily small. The second term is controlled by the almost everywhere convergence of $w_{n_j}$ to $w$ and the continuity of $\Phi^{-1}$. That means that
	 \begin{equation*}
	 u_{n_j} \to u = \Phi^{-1}(w)
	 \end{equation*}
	 almost everywhere. We immediately see that $\Phi(u) = w$.
	 
	 More importantly, since we have the uniform bound $u_{n_j} \leq M$ in $Q_T$ we can use the Dominated Convergence Theorem to state that $u_{n_j} \to u$ in $L^1(Q_T)$.
	 It is also easy to see that
	 \begin{equation}
	 ||u||_{L^2(Q_T)} \leq M||u||_{L^1(Q_T)},
	 \end{equation}
	 which means that $u \in L^2(Q_T)$.
	
	\item[$u \in L^\infty(0,T: L_\Psi(\Omega)$:]
	For any $t \in [0,T]$ we have by Fatou's Lemma \citep[p. 162]{weiss1999course} that 
	\begin{equation*}
	||u(t)||_{L^1(\Omega)} \leq \liminf_{n \to \infty} ||u_n(t)||_{L^1(\Omega)}.
	\end{equation*}
	Here we use Corollary \ref{cor:l1-contractivity_classical} with $\hat{u}=0$ as the solution to zero loading and initial data to get that
	\begin{align}
	||u(t)||_{L^1(\Omega)} &\leq \liminf_{n \to \infty} \left( ||u_{0,n}||_{L^1(\Omega)} + ||f_n||_{L^1(Q_T)}\right) \nonumber \\
	&= ||u_0||_{L^1(\Omega)} + ||f||_{L^1(Q_T)},	
	\end{align}
	implying that $u \in L^\infty(0,T: L^1(\Omega))$.
	
	We do similarly for $\Psi(u(t))$, but first we need to show that $\Psi_n(u_n)$ converges almost everwhere to $\Psi(u)$. We use the triangle inequality as
	\begin{equation}
	|\Psi_n(u_n) - \Psi(u)| \leq |\Psi_n(u_n) - \Psi(u_n)| + |\Psi(u_n) + \Psi(u)|.
	\end{equation}
	Because $|u_n| \leq M$ we use that $\Psi_n$ converges uniformly on compact sets (see proposition \ref{prop:Psi_uniform_convergence}) to control the first term. The second term is controlled by the almost everywhere convergence of $u_n$ to $u$ and the continuity of $\Psi$.
	
	Then, by Fatou's Lemma
	\begin{equation*}
	\int_\Omega \Psi(u(t))dx \leq \liminf_{n \to \infty} \int_\Omega \Psi_n(u_n(t))dx,
	\end{equation*}
	since $\Psi_n$ is nonnegative. Using Corollary \ref{cor:spatial_bound_w_poincare} we get
	\begin{align*}
	\int_\Omega \Psi(u(t))dx &\leq \liminf_{n \to \infty} \left( \int_\Omega \Psi_n(u_{0,n})dx + C\iint_{Q_T}f_n^2 dxdt \right)\nonumber \\
	&= \int_\Omega \Psi(u_0)dx + \iint_{Q_T} f^2dxdt, 
	\end{align*}
	and so we have that $u \in L^\infty(0,T: L_\Psi(\Omega))$.


	\item[$u$ is a weak solution:] We already know that $u \in L^1(Q_T)$, and $\Phi(u)=w \in L^1(0,T: W^{1,1}_0(\Omega))\subset L^2(0,T: H^1_0(\Omega))$ since $\Omega$ is bounded. So what's left is to show that
	\begin{equation}
	\label{claim_weak_solution}
	\iint_{Q_T}\left\{ \nabla\Phi(u) \cdot \nabla \eta - u\partial_t \eta - f\eta \right\}dxdt - \int_\Omega u_0\eta\big|_{t=0}dx = 0
	\end{equation}
	for any $\eta \in C^1(\overline{Q}_T)$ that vanish on $\partial \Omega \times [0,T)$ and for $t=T$. Seeing as every $u_n$ is a weak solution to the approximate (\ref{approximate_problem}) we can subtract it from the integral above to get
	\begin{align*}
	&\iint_{Q_T}\left\{ \nabla\Phi(u) \cdot \nabla \eta - u\partial_t \eta - f\eta \right\}dxdt - \int_\Omega u_0\eta\big|_{t=0}dx \\
	= &\iint_{Q_T}\left\{ \nabla(\Phi(u)-w_n) \cdot \nabla \eta - (u-u_n)\partial_t \eta - (f-f_n)\eta \right\}dxdt - \int_\Omega (u_0-u_{0,n})\eta\big|_{t=0}dx.
	\end{align*}
	
	All the terms are controlled by the convergences of $w_n$, $u_n$, $u_{0,n}$, as long as the limit is taken along the subsequence for which we know $w_n$ converges. Thus, (\ref{claim_weak_solution}) holds, and $u$ is a weak solution.
	
	
	We have from before that $u\in L^2(Q_T)$ as well as $\Phi(u) \in L^2(0,T: H^1(\Omega))$, so the solution is unique. In addition it is worth noticing that any accumulation point of the sequence of classical solutions is a weak solution, and so the uniqueness in theorem \ref{thm:uniqueness} precludes any other accumulation point than $u$. I.e. the whole sequence $(u_n)$ converges to $u$.


	\item[Fulfillment of the energy inequality:]
	By Fatou's lemma \citep[p. 162]{weiss1999course} and that $\Psi_n(u_n)$ are all nonnegative
	\begin{equation}
	\int_\Omega \Psi(u(T)) dx \leq \liminf_{n \to \infty} \int_\Omega \Psi_n(u_n(T))dx.
	\end{equation}
	
	In addition, since $|\cdot|^2$ is convex and nonnegative, the functional $I[u]:= ||\nabla u||^2_{L^2(Q_T)}$ is weakly lower semicontinuous (cf. \citep[Theorem 1, p. 468]{evans}). Then, since $w_n$ converges weakly to $w$ we have that
	\begin{equation}
	\iint_{Q_T}|\nabla w|^2 dxdt \leq \liminf_{n \to \infty} \iint_{Q_T} |\nabla w_n|^2 dxdt.
	\end{equation}
	
	Putting these two together and using a property of the limit inferior yields
	\begin{align*}
	\iint_{Q_T}|\nabla w|^2dxdt + \int_\Omega \Psi(u(T))dx &\leq \liminf_{n \to \infty} \iint_{Q_T} |\nabla w_n|^2 dxdt + \liminf_{n \to \infty} \int_\Omega \Psi_n(u_n(T))dx \\
	&\leq \liminf_{n \to \infty} \left[ \iint_{Q_T} |\nabla w_n|^2 dxdt + \int_\Omega \Psi_n(u_n(T))dx \right].
	\end{align*}
	
	Recall now that by Lemma \ref{lem:Energy_equality_classical} 
	\begin{equation*}
	\iint_{Q_T} |\nabla w_n|^2 dxdt + \int_\Omega \Psi_n(u_n(T))dxdt = \int_\Omega \Psi(u_{0,n})dx + \iint_{Q_T}f_n w_n dxdt,
	\end{equation*}
	for each $n$, and so
	\begin{align*}
	\iint_{Q_T}|\nabla w|^2dxdt + \int_\Omega \Psi(u(T))dx &\leq \liminf_{n \to \infty} \left[\int_\Omega \Psi(u_{0,n})dx + \iint_{Q_T}f_n w_n dxdt \right] \\
	&= \int_\Omega \Psi(u_0)dx + \iint_{Q_T} f w dxdt,
	\end{align*}
	where in the last step we've made use of that $u_{0,n}$ converges to $u_0$ in $L_{\Psi}(\Omega)$, $w_n$ converges weakly to $w$ and the convergence of $f_n$ to $f$, both in $L^2(Q_T)$. This shows that (\ref{energy_ineq}) is satisfied, making $u$ a weak energy solution.
\end{description}

\end{proof}


%%%%%%%%%%%%%%%%%%%%%%%%%%%%%%%%%%%%%%%%%%%%%%%%%%%%%%%%%%%
%		PROOF OF EXISTENCE PART II:
%%%%%%%%%%%%%%%%%%%%%%%%%%%%%%%%%%%%%%%%%%%%%%%%%%%%%%%%%%%

\subsubsection{$\Phi$ no longer locally Lipschitz}
We proceed by generalising Lemma \ref{lem:existence_part1} to the case when $\Phi$ is no longer locally Lipschitz, i.e. lifting assumption \ref{ass:Phi_lipschitz}. This is the case for example with $\Phi(u) = |u|^{m-1}u$ with $0< m < 1$, i.e. the fast diffusion equation.
Looking back at the previous section, we see that the only place where we actually made use of $\Phi$ being locally Lipschitz, was in getting a uniform bound on $\partial_t w_n$ in $L^2(Q_T^\tau)$ by way of Lemma \ref{prop:temporal_bound}. Because of this we'll take a small detour to find a sequence that enjoys the same uniform bounds as $w_n$ did in the previous section.

To that end we define
\begin{equation}
Z_n(u) = \int_0^u \mathrm{min}\{1, \Phi_n'(s)\} ds,
\end{equation}
which is easily seen to be globally Lipschitz. In addition $Z_n'(u) \leq 1$ and $Z_n'(u) \leq \Phi_n(u)$.

We then go on to define
\begin{equation}
z_n(x,t) = Z_n(u_n(x,t)).
\end{equation}

The following Lemma will establish that the sequence $(z_n)$ has the desired boundedness property.

\begin{lemma}
\label{prop:z_n_bounded_H1}
With assumptions \ref{ass:Phi_general}-\ref{ass:u_0}, \ref{ass:f_ft_u_0_bounded} and \ref{ass:smooth_boundary}, the sequence $(z_n)$ is uniformly bounded in $H^1(Q_T^\tau)$ for $0<\tau < \frac{T}{2}$.
\end{lemma} 

\begin{proof}
See that
\begin{align*}
\nabla z_n &= Z_n'(u_n)\nabla u_n \\
	&\leq \Phi_n'(u_n)\nabla u_n \\
	&= \nabla \Phi_n(u_n) \\
	&= \nabla w_n,
\end{align*}
which means that $|\nabla z_n|^2 \leq |\nabla w_n|^2$ in all of $Q_T$. As a trivial consequence of this
\begin{equation*}
\iint_{Q_T}|\nabla z_n|^2dxdt \leq \iint_{Q_T}|\nabla w_n|^2dxdt.
\end{equation*}
Using Lemma \ref{prop:spatial_boundedness} we conclude that both $z_n$ and $\nabla z_n$ are uniformly bounded in $L^2(Q_T)$.

Similarly we have the two inequalities
\begin{equation*}
|\partial_t z_n | \leq |\partial_t u_n|, \quad \text{ and} \quad |\partial_t z_n| \leq |\Phi_n'(u_n) \partial_t u_n|,
\end{equation*}
and as a consequence
\begin{equation*}
\iint_{Q_T}|\partial_t z_n|^2 dxdt \leq \iint_{Q_T} |\Phi_n'(u_n)|\cdot |\partial_t u_n|^2dxdt.
\end{equation*}
With Lemma \ref{prop:temporal_bound} $\partial_t z_n$ is uniformly bounded in $L^2(Q_T^\tau)$, and we conclude that $z_n$ is uniformly bounded in $H^1(Q_T^\tau)$.
\end{proof}

We are also able to establish some convergence results for $z_n$, summarised in lemmas immediately below.

\begin{lemma}
\label{prop:z_n_weak_convergence}
Under the same assumptions as Lemma \ref{prop:z_n_bounded_H1}, $z_n$ contains a subsequence $(z_{n_j})$ that converges almost everywhere to some $z$. The convergence happens also weakly in $L^2(Q_T)$. 
\end{lemma}

\begin{proof}
This proof is almost identical to the proofs of lemmas \ref{prop:weak_convergence_w_n} and \ref{prop:gradw_weak_convergence}: Using Lemma \ref{prop:z_n_bounded_H1} and Rellich-Kondrachov we can construct the same type of diagonal argument to extract a subsequence $(z_{n_j})$ that converges almost everywhere to $z$. Since $z_n$ is a bounded sequence in $L^2(Q_T)$, a reflexive Hilbert space, every subsequence contains a weakly convergent subsequence. The same use of Egorov's theorem as in Lemma \ref{prop:weak_convergence_w_n} establishes that $z_{n_j}$ converges weakly to $z$ in $L^2(Q_T)$.
\end{proof}


This enables us to prove existence in a slightly more general setting.
\begin{lemma}
\label{lem:weak_existence_part2}
Theorem \ref{thm:weak_existence} holds with the additional assumptions \ref{ass:f_ft_u_0_bounded} and \ref{ass:smooth_boundary}.
\end{lemma}

\begin{proof}
We use Lemma \ref{prop:classical_sols_seq} to get a sequence $(u_n)$ of classical soltutions, and we can use this sequence to define $(z_n)$. Further, Lemma \ref{prop:z_n_weak_convergence} provides us with a subsequence $(z_{n_j})$ that converges almost everywhere to $z$.
Now we obviously have $u_{n_j} = Z_{n_j}^{-1}(z_{n_j})$, and so
\begin{align*}
|u_{n_j} - Z^{-1}(z)| &= |Z_{n_j}^{-1}(z_{n_j}) - Z^{-1}(z)| \\
&\leq |Z_{n_j}^{-1}(z_{n_j}) - Z^{-1}(z_{n_j})| + |Z^{-1}(z_{n_j})- Z^{-1}(z)|.
\end{align*}
The sequence $(z_{n_j})$ converges pointwise to $z$ as well as being uniformly bounded 
since $|z_{n_j}| \leq |u_{n_j}| \leq M$. And so the first term on the right hand side in the above inequality is controlled by $Z_n^{-1}$ converging uniformly to $Z^{-1}$ on compact sets, as shown in Lemma \ref{prop:Zinv_uniform_convergence}. The second term tends to zero since $z_{n_j}$ converges pointwise to $z$ and $Z^{-1}$ is continuous.

We define $u=Z^{-1}(z)$ as the pointwise limit of $u_{n_j}$, and note as in the previous proof that $u_{n_j}$ also converges to $u$ in $L^1(Q_T)$.

The rest of the proof is almost identical to that for Lemma \ref{lem:existence_part1}. First, $w_{n_j}$ converges pointwise almost everywhere to some $w = \Phi(u)$, which follows from the fact that $\Phi_n$ converges uniformly to $\Phi$ on compact sets, and the continuity of $\Phi$. Since $w_{n}$ is bounded in $Q_T$ for all $n$ with uniform bound, we can use the dominated convergence theorem to deduce that  $w \in L^2(Q_T)$.

Since $\nabla w_{n_j}$ is a bounded sequence in $L^2(Q_T)$, by Lemma \ref{prop:spatial_boundedness}, we can argue as in Lemma \ref{prop:gradw_weak_convergence} to deduce that $\nabla w_{n_j}$ converges weakly to $\nabla w$:

It follows that $\nabla w_{n_j}$ converges to $\nabla w$ in the distributional sense, and since $w_{n_j}$ is bounded in $L^2(Q_T)$ we can deduce, by way of Banach-Alaoglu's and the uniqueness of distributional limits, that $\nabla w_{n_j}$ converges weakly to $\nabla w$ in $L^2(Q_T)$.


Lastly, establishing that $u$ is a weak solution and satisfies inequality (\ref{energy_ineq}) follows the exact same argument as in the proof of Lemma \ref{lem:existence_part1}.
\end{proof}

We will continue on with lifting more of the assumptions, in particular those on $f$ and $u_0$, but first an intermediate result is needed. The result is simply the $L^1$-stability presented in Lemma \ref{lem:l1_stability_classical} only now for weak solutions.

\begin{lemma}
\label{prop:l1_stability_weak_part1}
Assume the same as in Lemma \ref{lem:weak_existence_part2}, and let $u$ and $\hat{u}$ be two weak energy solutions with nonlinearity $\Phi$. They have initial data $u_0$ and $\hat{u}_0$, as well as loading $f$ and $\hat{f}$ respectively. Then the inequality
\begin{equation}
||u(t) - \hat{u}(t)||_{L^1(\Omega)} \leq ||u_0 - \hat{u}_0||_{L^1(\Omega)} + \int_0^t||f-\hat{f}||_{L^1(\Omega)}
\end{equation}
for almost every $t\in [0,T]$.
\end{lemma}

\begin{proof}
Let $\Phi_n$ be the approximations of $\Phi$, $u_{0,n}$ and $\hat{u}_{0,n}$ be approximations of $u_0$ and $\hat{u}_{0,n}$ to the initial data, converging in $L^1$. Furthermore, let $f_n$ and $\hat{f}_n$ be approximations of $f$ and $\hat{f}$ converging in $L^2(Q_T)$, and so also in $L^1(Q_T)$ since $Q_T$ is bounded. All the above approximations are as already presented in the foregoing proofs. Hence we have by Lemma \ref{prop:classical_sols_seq} sequences of classical solutions $u_n$ and $\hat{u}_n$. With the same argument as was done in Lemma \ref{lem:weak_existence_part2}, these sequences converge to $u$ and $\hat{u}$ in $L^1(Q_T)$.

We now use the triangle inequality for
\begin{equation*}
||u(t) - \hat{u}(t)||_{L^1(\Omega)} \leq ||u(t) - u_n(t)||_{L^1(\Omega)} + ||u_n(t) - \hat{u}_n(t)||_{L^1(\Omega)} + ||\hat{u}_n(t) - \hat{u}(t)||_{L^1(\Omega)}.
\end{equation*}
The first and last term on the right hand side are easily controlled by the convergence of $u_n$ and $\hat{u}_n$. As for the middle term we use Corollary \ref{cor:l1-contractivity_classical}, and so for any $\epsilon > 0$,
\begin{equation*}
||u(t) - \hat{u}(t)||_{L^1(\Omega)} \leq \epsilon + ||u_{0,n} - \hat{u}_{0,n}||_{L^1(\Omega)} + ||f_n - \hat{f}_n||_{L^1(Q_t)}
\end{equation*}
for sufficiently large $n$.
Using the triangle inequality another time yields
\begin{align*}
||u(t) - \hat{u}(t)||_{L^1(\Omega)} &\leq \epsilon + ||u_{0,n} - u_0||_{L^1(\Omega)} + ||u_0 - \hat{u}_0||_{L^1(\Omega)} + ||\hat{u}_0 -\hat{u}_{0,n}||_{L^1(\Omega)} \\
&+ ||f_n - f||_{L^1(Q_t)} + ||f - \hat{f}||_{L^1(Q_t)} + ||\hat{f}-\hat{f}_n||_{L^1(Q_t)}.
\end{align*}
Lastly, we have from the various convergences that for sufficiently large $n$  
\begin{equation*}
||u(t) - \hat{u}(t)||_{L^1(\Omega)} \leq 2\epsilon + ||u_0 - \hat{u}_0||_{L^1(\Omega)} + ||f - \hat{f}||_{L^1(Q_t)}, 
\end{equation*}
and since $\epsilon$ was arbitrary this concludes the proof.
\end{proof}

To conclude this section, is a Lemma on an estimate for $\partial_t z$ in $L^2(Q_T)$, which will prove useful in the coming section.

\begin{lemma}
\label{prop:z_t_estimate}
Under the same assumptions as Lemma \ref{prop:z_n_bounded_H1}, $\partial_t z_n$ converges weakly to $\partial_t z$ in $L^2(Q_T^\tau)$, for $0 < \tau < \frac{T}{2}$. Moreover, the esimate
\begin{equation}
\label{z_t_estimate}
\iint_{Q_T^\tau} |\partial_t z|^2 dxdt \leq \frac{||\partial_t \zeta||_\infty}{2}\left( K||u_0||_{L^1(\Omega)} + ||f||_{L^2(Q_T)}\right) + \iint_{Q_T}|\partial_t (f \zeta ) w| dxdt
\end{equation}
for $\tau >0$, where $\zeta=\zeta(t)$ is a smooth cutoff function with $\zeta(t)=1$ for $\tau \leq t \leq T-\tau$.

\end{lemma}
\begin{proof}
That $\partial_t z_n$ converges weakly to $\partial_t z$ in follows by almost the exact same approach we used in proving Lemma \ref{prop:gradw_weak_convergence}: First by noting that since $z_n$ converges weakly to $z$ in $L^2(Q_T)$, $\partial_t z_n$ must converge to $\partial_t z$ in the distributional sense.

For every $\tau > 0$, $\partial_t z_n$ is a uniformly bounded sequence in $L^2(Q_T^\tau)$, and so we can use the same argument as in Lemma \ref{prop:gradw_weak_convergence} to deduce that $\partial_t z_n$ converges weakly to $\partial_t z$ in $L^2(Q_T^\tau)$ for every $0 < \tau < \frac{T}{2}$.

From norms being weakly lower semicontinuous
\begin{equation}
\iint_{Q_T^\tau} |\partial_t z|^2 dxdt \leq \liminf_{n \to \infty} \iint_{Q_T} |\partial_t z_n|^2 dxdt,
\end{equation}
and by Lemma \ref{prop:temporal_bound}, we get that for any smooth cutoff function $\zeta$ with the above stated properties
\begin{align*}
\iint_{Q_T^\tau} |\partial_t z|^2 dxdt &\leq \liminf_{n \to \infty} \iint_{Q_T} \left\{\frac{\partial_t \zeta}{2}|\nabla w_n|^2 - \partial_t (\zeta f_n)\right\} dxdt \\
&\leq \liminf_{n\to \infty}\iint_{Q_T} \left\{ \frac{|\partial_t \zeta|}{2}|\nabla w_n|^2dxdt + |\partial_t (f_n \zeta) |w_n| \right\}dxdt.
\end{align*}
A property of the limit inferior is that $\liminf(a_n + b_n) \leq \liminf a_n + \limsup b_n$ for any two real sequences $(a_n)$ and $(b_n)$. This is relevant in this setting because then
\begin{equation}
\label{z_t_L2_almost_there}
\iint_{Q_T^\tau} |\partial_t z|^2 dxdt\leq \liminf_{n\to \infty} \iint_{Q_T}\frac{|\partial_t \zeta|}{2}|\nabla w_n|^2dxdt + \limsup_{n \to \infty} \iint_{Q_T}|\partial_t (f_n \zeta) w_n| dxdt.
\end{equation}
On the first term on the right hand side, we use the estimate (\ref{gradwn_bound_uniform}) provided by Lemma \ref{prop:spatial_boundedness} to get
\begin{equation*}
\liminf_{n\to \infty} \iint_{Q_T}\frac{|\partial_t \zeta|}{2}|\nabla w_n|^2dxdt \leq \frac{||\partial_t \zeta||_\infty}{2}\left( K||u_0||_{L^1(\Omega)} + ||f||_{L^2(Q_T)}\right).
\end{equation*}

For the second term, we only need to note that $\partial_t (f_n \zeta)$ converges pointwise to $\partial_t (f \zeta)$ and $w_n$ converges almost everywhere to $w$ Both are bounded sequences, and since we're on a bounded set, this permits us to use the dominated convergence theorem as
\begin{equation*}
\limsup_{n \to \infty} \iint_{Q_T}|\partial_t (f_n \zeta) w_n| dxdt = \iint_{Q_T}|\partial_t (f \zeta ) w| dxdt.
\end{equation*}

Putting these back into (\ref{z_t_L2_almost_there}), we get
\begin{equation}
\iint_{Q_T^\tau} |\partial_t z|^2 dxdt \leq \frac{||\partial_t \zeta||_\infty}{2}\left( K||u_0||_{L^1(\Omega)} + ||f||_{L^2(Q_T)}\right) +  \iint_{Q_T}|\partial_t (f \zeta ) w| dxdt.
\end{equation}
\end{proof}

%%%%%%%%%%%%%%%%%%%%%%%%%%%%%%%%%%%%%%%%%%%%%%%%%%%%%%%%%%%%%%%%%%
%			PROOF OF EXISTENCE PART III:
%%%%%%%%%%%%%%%%%%%%%%%%%%%%%%%%%%%%%%%%%%%%%%%%%%%%%%%%%%%%%%%%%%


\subsubsection{No longer $C^{2,\alpha}$ boundary}
Following suit from the previous section, we'll now aim to prove Lemma \ref{lem:weak_existence_part2} without the assumption that $\partial \Omega$ is $C^{2,\alpha}$, of course, this is usually the case in applications, where the domain can have corners and such, making the boundary not sufficiently smooth to satisfy the assumptions in theorem \ref{thm:classical_existence} to provide a classical solution. The strategy for proving existence now will be to get a sequence of weak solution on smaller subdomains that will eventually converge to all of $Q_T$.

To that end, we will in this section alter the approximations on $u_0$ and $f$ that were provided in section \ref{sec:approximations}. Take a sequence of simply connected compact sets $K_n \subset \Omega$ so that $K_n \subset K_{n+1}$ and $\cup_{n \in \mathbb{N}}K_n = \Omega$. Quite similarly to how we did when approximating $u_0$ in section \ref{sec:approximations}, but now with the additional asssumption that $\partial K_n$ is $C^{2,\alpha}$ smooth for each $n$. We'll also denote by $Q_n$ the time-space cylinder $K_n \times (0,T)$. 

We go on then to define $u_{0,n} = \chi_{K_n} u_0$ and $f_n = \chi_{K_n} f$. See then that $u_{0,n} \to u_0$ in $L^1(\Omega)$ and $f_n \to f$ in $L^2(Q_T)$. The feasibility of these approximations can be found in \citep[Sec. 5.3]{evans}. In this notational setting we'll continue with our generalisation.

\begin{lemma}
\label{prop:sequence_weak_solution_nonsmoothboundary}
With assumptions \ref{ass:Phi_general}-\ref{ass:f_ft_u_0_bounded}, we can identify a sequence $u_n$ in $L^1(Q_T)$ that are weak energy solutions on $Q_n \subset Q_T$. 
\end{lemma}
\begin{proof}
To start, we set up for each $n$ the approximate problem
\begin{equation}
\begin{cases}
\partial_t u_n &= \Delta \Phi(u_n) + f_n, \quad \text{in } Q_n \\
u_n(x,0) &= u_{0,n}, \quad \text{in } K_n \\
u_n &= 0, \quad \text{on } \partial K_n \times (0,T), 
\end{cases}
\end{equation}
which has by Lemma \ref{lem:weak_existence_part2} a weak energy solution. We'll denote by $u_n$ the weak solution extended by $0$ to be defined in all of $Q_T$.
\end{proof}

Similar to the previous section, we will for this section define the sequences $w_n = \Phi(u_n)$ and
\begin{equation}
z_n(x,t) = Z(u(x,t)) = \int_0^{u_n(x,t)} \mathrm{min}\{1, \Phi'(s)\} ds,
\end{equation}
and we will in a little bit see that we can find a subsequence of $(z_n)$ that converges almost everwhere. But for this we'll need to establish some boundedness properties.

\begin{lemma}
\label{prop:w_n_nonsmooth_spatial_boundedness}
With the same assumptions as Lemma \ref{prop:sequence_weak_solution_nonsmoothboundary}, the sequence $w_n$ is bounded in $L^2(0,T: H_0^1(\Omega))$.
\end{lemma}
\begin{proof}
Notice that for each $n$, $w_n= |\nabla w_n|=0$ in $Q_T\setminus Q_n$ by the construction done in Lemma \ref{prop:sequence_weak_solution_nonsmoothboundary}, and so

\begin{equation*}
\iint_{Q_T} |\nabla w_n|^2dxdt + \int_\Omega \Psi(u_n(T)) = \iint_{Q_n}|\nabla w_n|^2dxdt + \int_{K_n} \Psi(u_n(T))dx.
\end{equation*}
For each $n$, $u_n$ is a weak energy solution in $Q_n$ and as satisfies \ref{energy_ineq}, giving
\begin{align}
\label{step_spatial_bound_z_nonsmooth_boundary}
\iint_{Q_T} |\nabla w_n|^2dxdt + \int_\Omega \Psi(u_n(T)) &\leq \int_{K_n}\Psi(u_{0,n})dx + \iint_{Q_n}f_n w_n dxdt \nonumber \\
&= \int_\Omega \Psi(u_{0,n})dx + \iint_{Q_T} f_n w_n dxdt \nonumber\\
& \leq \int_\Omega \Psi(u_0)dx + \iint_{Q_T} f_n w_n dxdt,
\end{align}
where we've also made use of that $|u_{0,n}| \leq |u_0|$, as well as that $w_n$ and $u_{0,n}$ vanish outside of $Q_n$ and $K_n$ respectively.
$w_n$ is in $L^2(0,T:H_0^1(\Omega))$ for each $n$, so use of the inequality $ab \leq \frac{a^2}{4c} + cb^2$ on the last term on the right hand side yields
\begin{equation*}
\iint_{Q_T} f_n w_n dxdt \leq \frac{1}{4c}\iint_{Q_T}f_n^2dxdt + c\iint_{Q_T} w_n^2 dxdt.
\end{equation*}

We use a Poincaré inequality, set $c$ as was done in Corollary \ref{cor:spatial_bound_w_poincare} and notice that $f_n^2 \leq f^2$ in $Q_T$ to arrive at
\begin{equation*}
\iint_{Q_T} f_n w_n dxdt \leq C\iint_{Q_T} f^2dxdt + \frac{1}{2}\iint_{Q_T} |\nabla w_n|^2 dxdt,
\end{equation*}
where $C >0$ is a constant independent of $n$. When putting this back into (\ref{step_spatial_bound_z_nonsmooth_boundary}), the result is
\begin{equation}
\frac{1}{2}\iint_{Q_T} |\nabla w_n|^2 dxdt + \int_\Omega \Psi(u_n(T))dx \leq \int_\Omega \Psi(u_0)dx + C\iint_{Q_T} f^2dxdt,
\end{equation}
implying that $\nabla w_n$ is a bounded sequence in $L^2(Q_T)$. As a consequence of the Poincaré inequality
\begin{equation*}
\iint_{Q_T} w_n^2 dxdt \leq \tilde{C} \iint_{Q_T} |\nabla w_n|^2 dxdt,
\end{equation*}
making also $w_n$ a bounded sequence in $L^2(Q_T)$.

\end{proof}

\begin{lemma}
\label{prop:z_n_bounded_in_H1}
With the same assumptions as Lemma \ref{prop:sequence_weak_solution_nonsmoothboundary}, the sequence $z_n$ is bounded in $H^1(Q_T^\tau)$ for every $0 < \tau < \frac{T}{2}$.
\end{lemma}

\begin{proof}
That $z_n$ is bounded in $L^2(0,T: H_0^1(\Omega))$ follows from the observation that $|\nabla z_n| \leq |\nabla w_n|$ almost everywhere and Lemma \ref{prop:w_n_nonsmooth_spatial_boundedness}. We have for every $n$ that $z_n = 0$ in $Q_T\setminus Q_n$, and as a consequence $\partial_t z_n = 0$ also. Therefore
\begin{equation*}
\iint_{Q_T^\tau} |\partial_t z_n|^2 dxdt = \iint_{Q_n^\tau} |\partial_t z_n|^2 dxdt.
\end{equation*} 

By Lemma \ref{prop:z_t_estimate} we can use the estimate (\ref{z_t_estimate}) as
\begin{equation}
\label{z_n_bounded_in_H1_start}
\iint_{Q_T^\tau} |\partial_t z_n|^2 dxdt \leq \frac{||\partial_t \zeta||_\infty}{2}\left( K||u_{0,n}||_{L^1(K_n)} + ||f_n||_{L^2(Q_n)}\right) + \iint_{Q_n}|\partial_t (f_n \zeta ) w_n| dxdt,
\end{equation}
where $\zeta = \zeta(t)$ is a smooth cutoff function from Lemma \ref{prop:z_t_estimate}, which has a bounded derivative.

First off, we see that $||u_{0,n}||_{L^1(K_n)} = ||u_{0,n}||_{L^1(\Omega)}$ and $||f_n||_{L^2(Q_n)} \leq ||f||_{L^2(Q_T)}$. For the second part on the right hand side of (\ref{z_n_bounded_in_H1_start}) we have that $w_n=0$ in $Q_T\setminus Q_n$, and so integration can be extended to all of $Q_T$. In addition $|\partial_t (f_n \zeta)| \leq ||\partial_t f||_{\infty} + ||f||_\infty ||\partial_t \zeta||_\infty$, which are all bounded, independent of $n$. Hence the right hand side of (\ref{z_n_bounded_in_H1_start}) is bounded independent of $n$. This, along with $z_n$ being bounded in $L^2(0,T: H^1_0(\Omega)$, proves the assertion.
\end{proof}

With that last Lemma we are able to prove existence of weak energy solutions

\begin{lemma}
\label{lem:weak_existence_part3}
Theorem \ref{thm:weak_existence} holds with the additional assumption \ref{ass:f_ft_u_0_bounded}.
\end{lemma}

\begin{proof}
We start by constructing a sequence of weak solutions $(u_n)$ in $Q_n$ defined in $Q_T$ as done in Lemma \ref{prop:sequence_weak_solution_nonsmoothboundary}. With $z_n = Z(u_n)$ and $w_n = \Phi(u_n)$, $(z_n)$ is bounded in $H^1(Q_T^\tau)$ for every $0< \tau < \frac{T}{2}$ by Lemma \ref{prop:z_n_bounded_in_H1}. A diagonal argument, as was done in the proof of Lemma \ref{prop:weak_convergence_w_n} to extract a subsequence of $(z_n)$ that converges pointwise almost everywhere to some $z$. With a pointwise limit for a subsequence $z_{n_j}$, the rest of the proof is identical to the proof of Lemma \ref{lem:weak_existence_part2}.  
\end{proof}

Before we move on to remove the assumption regarding the boundedness of $u_0$ and $f$ we need to ensure the $L^1$-contractivity result of Lemma \ref{prop:l1_stability_weak_part1} still holds after passing to limit as we did in the proof of Lemma \ref{lem:weak_existence_part3}, it will become useful in the coming section.

\begin{lemma}
\label{prop:l1_stability_weak_part2}
With assumptions \ref{ass:Phi_general}-\ref{ass:f_ft_u_0_bounded},  let $u$ and $\hat{u}$ be two weak energy solutions with nonlinearity $\Phi$. They have initial data $u_0$ and $\hat{u}_0$, as well as loading $f$ and $\hat{f}$ respectively. Then the inequality
\begin{equation}
||u(t) - \hat{u}(t)||_{L^1(\Omega)} \leq ||u_0 - \hat{u}_0||_{L^1(\Omega)} + \int_0^t||f-\hat{f}||_{L^1(\Omega)}
\end{equation}
holds for almost every $t\in [0,T]$.
\end{lemma}

\begin{proof}
The proof follows closely that of Lemma \ref{prop:l1_stability_weak_part1}. Indeed, we start by constructing sequences of weak solutions $u_n$ and $\hat{u}_n$ by proposition \ref{prop:sequence_weak_solution_nonsmoothboundary} that converge to $u$ and $\hat{u}$ in $L^1(Q_T)$. Fix $\epsilon > 0$, then with the triangle inequality
\begin{align*}
||u(t) - \hat{u}(t)||_{L^1(\Omega)} &\leq ||u(t) - u_n(t)||_{L^1(\Omega)} + ||u_n(t) - \hat{u}_n(t)||_{L^1(\Omega)} + ||\hat{u}_n(t) - \hat{u}(t)||_{L^1(\Omega)} \\
&< \epsilon + ||u_n(t) - \hat{u}_n(t)||_{L^1(\Omega)},
\end{align*}
for sufficiently large $n$. Since both $u_n$ and $\hat{u}_n$ vanish on $Q_T\setminus Q_n$ the integration can be restricted to $K_n$ instead of $\Omega$, leaving us in a position to make use of Lemma \ref{prop:l1_stability_weak_part1}, resulting in
\begin{align*}
||u(t) - \hat{u}(t)||_{L^1(\Omega)} &< \epsilon + ||u_{0,n} - \hat{u}_{0,n}||_{L^1(\Omega)} + ||f_n - \hat{f}_n||_{L^1(Q_T)} \\
&\leq \epsilon + ||u_{0,n} - u_0||_{L^1(\Omega)} + ||u_0 - \hat{u}_0||_{L^1(\Omega)} + ||\hat{u}_0 - \hat{u}_{0,n}||_{L^1(\Omega)} \\
&+ ||f_n - f||_{L^1(Q_T)} + ||f - \hat{f}||_{L^1(Q_T)} + ||\hat{f} - \hat{f}_n||_{L^1(Q_T)}.
\end{align*}
All terms involving $n$ we can control by the various convergences. Thus, when passing to the limit $n \to \infty$ we end up with
\begin{equation}
||u(t) - \hat{u}(t)||_{L^1(\Omega)} \leq ||u_0 - \hat{u}_0||_{L^1(\Omega)} + ||f - \hat{f}||_{L^1(Q_T)}.
\end{equation}

\end{proof}


\subsubsection{Removing the assumptions on $f$ and $u_0$}
We've finally arrived at the last section, where we'll prove theorem \ref{thm:weak_existence} without the assumption \ref{ass:f_ft_u_0_bounded}, which have proved so very useful for us up to this point. Let's just get to it then.

\begin{proof}[Proof of theorem \ref{thm:weak_existence}]
As opposed to the previous weak existence proofs, the strategy here will be to use the $L^1$-contractivity to establish convergence, rather than a compactness argument.
\begin{description}
\item[Finding a candidate for a solution:]
Let $(u_{0,n})$ be sequence of bounded functions converging to $u_0$ in $L_{\Psi}(\Omega)$, and $(f_n)$ a sequence of bounded functions converging to $f$ in $L^2(Q_T)$, as well as $\partial_t f_n$ being bounded in $Q_T$ for each $n$. It will also be useful to assume $|u_{0,n}| \leq |u_0|$.
A way, to do this is first truncate $u_0$ and $f$ below and above by $-n$ and $n$, and then for the approximations of $f$, mollify, as was done in section \ref{sec:approximations}. The mollification is there to ensure that also $\partial_t f_n$ is bounded for each $n$.

Using Lemma \ref{lem:weak_existence_part3} we get a sequence of weak energy solutions, $(u_n)$, and by Lemma \ref{prop:l1_stability_weak_part2}
\begin{equation*}
||u_n(t) - u_m(t)||_{L^1(\Omega)} \leq ||u_{0,n} - u_{0,m}||_{L^1(\Omega)} + ||f_n - f_m||_{L^1(Q_T)}
\end{equation*}
for almost every $t \in [0,T]$, and so $(u_n)$ is a Cauchy sequence in $L^\infty(0,T: L^1(\Omega))$, guaranteeing a limit $u \in L^\infty(0,T: L^1(\Omega))$. The convergence happens also in $L^1(Q_T)$, since
\begin{equation*}
||u - u_n||_{L^1(Q_T)} \leq T\underset{t \in [0,T]}{\mathrm{sup}}||u(t) - u_n(t)||_{L^1(\Omega)}.
\end{equation*}
From here on we may also assume that $u_n$ converges almost everywhere to $u$, because if not we can take a subsequence that does.

\item[$\Phi(u)$ is in $L^2(Q_T)$:]
This part of the proof will closely resemble the part of the proof for Lemma \ref{prop:weak_convergence_w_n}. See that since $\Phi$ is continuous, $\Phi(u_n)$ converges almost everywhere to $\Phi(u)$. With the energy inequality
\begin{equation*}
\iint_{Q_T} |\nabla w_n|^2dxdt + \int_{\Omega} \Psi(u_n(T))dxdt \leq \int_\Omega \Psi(u_{0,n})dx + \iint_{Q_T}f_n \Phi(u_n)dxdt
\end{equation*}
holding for each $n$ we can do the same trick as when deriving (\ref{spatial_derivative_bound_w_poincare}) for classical solutions to arrive at the estimate
\begin{equation}
\frac{1}{2}\iint_{Q_T}|\nabla \Phi(u_n)|^2 dxdt + \int_\Omega \Psi(u_n(T))dx \leq \int_\Omega \Psi(u_{0,n})dx + C\iint_{Q_T}f_n^2dxdt,
\end{equation}
where $C$ is a constant independent of $n$.
With the assumptions we've put on $f_n$ and $u_{0,n}$ we can conclude that $||f_n||_{L^2(Q_T)} \leq ||f||_{L^2(Q_T)}$ and $||\Psi(u_{0,n})||_{L^1(\Omega)} \leq ||\Psi(u_0)||_{L^1(\Omega)}$. This last inequality comes from the fact that $\Psi(s) \leq \Psi(r)$ whenever $|s|\leq |r|$ and the have equal sign, which is the case with $u_{0,n}$ and $u_0$. This implies that $\nabla \Phi(u_n)$ is uniformly bounded in $L^2(Q_T)$, and as an extension $\Phi(u_n)$ is also bounded in $L^2(Q_T)$ from the Poincaré inequality.

By Banach-Alaoglu every subsequence of $\Phi(u_n)$ has a weakly convergent subsequence, and by the pointwise convergence we can, after arguing as in the proof in Lemma \ref{prop:weak_convergence_w_n}, conclude that the weak limit is always $\Phi(u)$. This implies that $\Phi(u_n)$ converges weakly to $\Phi(u)$.

\item[$\nabla \Phi(u_n)$ converges weakly to $\nabla \Phi(u)$:]
The proof of this assertion is done, as in the proof of proposition \ref{prop:gradw_weak_convergence}, by noticing from the weak convergence of $\Phi(u_n)$ we can deduce that $\nabla \Phi(u_n)$ converges to $\nabla \Phi(u)$ in the distributional sense. In addition $\nabla \Phi(u_n)$ is a bounded sequence in $L^2(Q_T)$, and so every subsequence has a weakly convergent subsequence. Arguing as before, the limit point is always $\nabla \Phi(u)$, and so $\nabla \Phi(u_n)$ converges weakly to $\nabla \Phi(u)$.


\item[$u$ is a weak solution:]
With the various convergences shown above or given by assumption, showing that $u$ is a weak solution and satisfies the energy inequality (\ref{energy_ineq}) is done identically as in the proof of lemma \ref{lem:existence_part1}. The same is true for showing that $u$ satisfies the energy inequality (\ref{energy_ineq}).
\end{description}
\end{proof}


For the sake of completeness, we'll end this project by ensuring that the $L^1$-contractivity still holds.

\begin{theorem}
\label{lem:l1_stability_weak_final}
With assumptions \ref{ass:Phi_general}-\ref{ass:u_0},  let $u$ and $\hat{u}$ be two weak energy solutions with nonlinearity $\Phi$. They have initial data $u_0$ and $\hat{u}_0$, as well as loading $f$ and $\hat{f}$ respectively. Then the inequality
\begin{equation}
||u(t) - \hat{u}(t)||_{L^1(\Omega)} \leq ||u_0 - \hat{u}_0||_{L^1(\Omega)} + \int_0^t||f-\hat{f}||_{L^1(\Omega)}
\end{equation}
holds for almost every $t\in [0,T]$.
\end{theorem}

\begin{proof}
The proof is identical to that of Lemma \ref{prop:l1_stability_weak_part2}, only now the use of Lemma \ref{prop:l1_stability_weak_part1} is exchanged with use of Lemma \ref{prop:l1_stability_weak_part2}.
\end{proof}





%%%%%%%%%%%%%%%%%%%%%%%%%%%%%%%%%%%%%%%%%%%%%%%%%%%%%%%
%				BIBLIOGRAPHY
%%%%%%%%%%%%%%%%%%%%%%%%%%%%%%%%%%%%%%%%%%%%%%%%%%%%%%%
\newpage
\bibliographystyle{plainnat}
\bibliography{project_bibliography.bib}




%%%%%%%%%%%%%%%%%%%%%%%%%%%%%%%%%%%%%%%%%%%%%%%%%%%%%%%
%				APPENDICES
%%%%%%%%%%%%%%%%%%%%%%%%%%%%%%%%%%%%%%%%%%%%%%%%%%%%%%%
\newpage
\begin{appendix}
\section{Passing derivative inside the integral}
\label{sec:derivative}
In this section we prove the assertion made in the proof of theorem \ref{thm:uniqueness} regarding passing the derivative inside an integral.

\begin{lemma}
Let $\varphi \in L^2(0,T: H^1(\Omega))$, then 
\begin{equation}
\label{diff_int}
	\nabla \int_t^T \varphi(x,s)ds = \int_t^T \nabla\varphi(x,s)ds
\end{equation}
weakly in $H^1_0(\Omega)$, meaning 
\begin{equation*}
\int_\Omega \psi \nabla \left(\int_t^T \varphi ds \right) dx = \int_\Omega \psi \left(\int_t^T \nabla \varphi ds \right) dx
\end{equation*}
for all $\psi \in H^1_0(\Omega)$.
\end{lemma}

\begin{proof}

For simplicity we now only consider one spatial dimension.

To begin with, assume that $\psi \in C^{\infty}_c(\Omega)$ and $\eta \in L^2(0,T: H_0^1(\Omega))\cap C^\infty_c(Q_T)$, then we have that
\begin{align*}
\int_{\Omega}\psi \partial_x\left(\int_t^T \eta(x,s) ds\right)dx &= \int_{\Omega}\psi \lim_{h \to 0}\left(\int_t^T \frac{\eta(x+h,s)-\eta(x,s)}{h} ds\right)dx\\
&= \int_{\Omega}\psi \lim_{h \to 0}\left(\int_t^T \eta_h(x,s) ds\right)dx.
\end{align*}

Assuming that $\partial_x \eta(x,s)$ is bounded, we can use the mean value theorem to state that $|\eta_h| \leq  ||\partial_x \eta||_\infty$. We also have that $\eta_h(x,s) \to \partial_x \eta(x,s)$ pointwise, and so we're in a position to use the Dominated Convergence theorem to conclude that
\begin{equation}
\label{results_diff_first}
\int_{\Omega}\psi \partial_x\left(\int_t^T \eta(x,s) ds\right)dx = \int_{\Omega}\psi \left(\int_t^T \partial_x\eta(x,s) ds\right)dx.
\end{equation}

To extend the result to $L^2(0,T:H_0^1 (\Omega))$ we use that $C_c^\infty$ is dense in this space. So let $\{\psi_n\},\{\varphi_n\}$ be two sequence converging to $\psi$ and $\varphi$ in $H^1_0(\Omega)$ and $L^2(0,T:H_0^1(\Omega))$ respectively. Considering the left hand side of (\ref{results_diff_first}), we get
\begin{align*}
&\left|\int_{\Omega}\psi \partial_x \left( \int_t^T\varphi ds \right)dx - \int_{\Omega}\psi_n \partial_x \left( \int_t^T \varphi_n ds \right) dx \right| \\
&\leq \underbrace{\left| \int_{\Omega} \psi \partial_x \left(\int_t^T \varphi - \varphi_n ds\right)dx \right| }_{I} + \underbrace{\left| \int_{\Omega}(\psi - \psi_n)\partial_x\left(\int_t^T \varphi_n ds\right) dx \right|}_{II}
\end{align*}
after adding and subtracting $\int_\Omega \psi \partial_x \left(\int_t^T \varphi_n ds\right)dx$.

Let's go over these two integrals one at a time.
\begin{enumerate}[I:]

\item After using integration by parts and Cauchy-Schwarz, the first integrals is estimated as
\begin{align*}
\left| \int_{\Omega} \psi \partial_x \left(\int_t^T \varphi - \varphi_n ds\right)dx \right| &= \left| \int_\Omega (\partial_x \psi) \int_t^T\varphi - \varphi_n ds dx \right| \\
&\leq ||\psi||_{H^1(\Omega)}\big|\big| \int_t^T \varphi - \varphi_n ds \big|\big|_{L^2(\Omega)}.
\end{align*}
For the latter of this factors we notice that since norms are convex we can use Jensen's inequality \citep[Theorem 2, p. 705]{evans} to get
\begin{align*}
\big|\big| \int_t^T \varphi - \varphi_n ds \big|\big|_{L^2(\Omega)} &\leq \int_t^T ||\varphi - \varphi_n||_{L^2(\Omega)}ds \\
&\leq ||\varphi - \varphi_n||_{L^2(Q_T)} \\
&\leq ||\varphi - \varphi_n||_{L^2(0,T: H^1(\Omega))},
\end{align*}
which we control by the convergence of $\varphi_n$ to $\varphi$.

\item We do exactly the same steps as we did immediately above for I, but now convergence is controlled by the convergence of $\psi_n$ to $\psi$ in $H^1(\Omega)$.

\end{enumerate}
From the convergence of $I$ and $II$ we deduce that
\begin{equation}
\label{diff_results_LHS}
\int_{\Omega}\psi \partial_x \left( \int_t^T\varphi ds \right)dx = \lim_{n\to \infty}\int_{\Omega}\psi_n \partial_x \left( \int_t^T \varphi_n ds \right) dx.
\end{equation}
%%%%%%%%%%%%%%%%%%%%%%%%%
The right hand side can be handled identically, save for a detail.
We do the same decomposition as
\begin{align*}
&\left| \int_\Omega \psi \left( \int_t^T \partial_x \varphi ds \right)dx - \int_\Omega \psi_n \left( \int_t^T \partial_x \varphi_n ds \right)dx \right| \\
&\leq \underbrace{ \left| \psi \left( \int_t^T \partial_x (\varphi - \varphi_n)ds \right) dx\right|}_{I} + \underbrace{ \left| \int_\Omega (\psi - \psi_n) \left( \int_t^T \partial_x \varphi_n ds \right) dx \right|}_{II}.
\end{align*}

Again, let's take these term by term:

\begin{enumerate}[I:]

\item Using Cauchy-Schwarz immediately gives
\begin{align*}
\left| \psi \left( \int_t^T \partial_x (\varphi - \varphi_n)ds \right) dx\right| & \leq ||\psi||_{L^2(\Omega)} \big| \big| \int_t^T \partial_x(\varphi - \varphi_n) ds \big| \big|_{L^2(\Omega)} \\
& \leq ||\psi||_{H^1(\Omega)}||\varphi - \varphi_n||_{L^2(0,T:H^1(\Omega))},
\end{align*}
where Jensen's inequality has been put to use yet another time. This is controlled by the convergence of $\varphi_n$ to $\varphi$ in $L^2(0,T: H^1(\Omega))$.

\item The same steps as immediately above, only now convergence is controlled by the convergence of $\psi_n$ to $\psi$. 
\end{enumerate}

Hence we deduce that
\begin{equation}
\label{diff_results_RHS}
\int_\Omega \psi \left( \int_t^T \partial_x \varphi ds \right)dx = \lim_{n \to \infty} \int_\Omega \psi_n \left( \int_t^T \partial_x \varphi_n ds \right)dx
\end{equation}


With (\ref{diff_results_RHS}) and (\ref{diff_results_LHS}) we get
\begin{align*}
\int_{\Omega}\psi \partial_x \left( \int_t^T\varphi ds \right)dx &= \lim_{n\to \infty}\int_{\Omega}\psi_n \partial_x \left( \int_t^T \varphi_n ds \right) dx \\
&= \lim_{n \to \infty} \int_\Omega \psi_n \left( \int_t^T \partial_x \varphi_n ds \right)dx \\
&= \int_\Omega \psi \left( \int_t^T \partial_x \varphi ds \right)dx,
\end{align*}
which is the final result for differentiation in one spatial dimension. To get the more general (\ref{diff_int}), we simply apply this result for each individual spatial dimension while keeping the others fixed.

\end{proof}


%%%%%%%%%%%%%%%%%%%%%%%%%%%%%%%%%%%%%%%%%%
%		COMPACTNESS
%%%%%%%%%%%%%%%%%%%%%%%%%%%%%%%%%%%%%%%%%%
\newpage
\section{Compactness argument}
\label{sec:compactness}
Here I will simply state the Rellich-Kondrachov Compactness theorem as it is presented in \citep{evans}, mostly for ease of reference. For more details, please cf. Chapter 5 ibid.

\begin{mydef}
Let $X$ and $Y$ be Banach spaces with $X \subset Y$. We say that $X$ is \textbf{compactly embedded} in $Y$, written
\begin{equation*}
X \subset \subset Y,
\end{equation*}
provided 
\begin{enumerate}[i)]
	\item $||u||_Y \leq ||u||_X$ for every $u \in X$, for some constant $C$, and

	\item each bounded sequence in $X$ is precompact in $Y$.
\end{enumerate}
\end{mydef}

With this definition in mind we have the following theorem:
\begin{theorem}[Rellich-Kondrachov Compactness Theorem]
\label{thm:Rellich}
Assume $U$ is an open, bounded subset of $\mathbb{R}^n$ and $\partial U$ is $C^1$.
Suppose $1\leq p < n$. Then
\begin{equation}
\label{Rellich}
W^{1,p}(U) \subset \subset L^q(U)
\end{equation}
for each $1\leq q < p^*$, where $p^* = \frac{np}{n-p}$ is the Sobolev conjugate of $p$.
\end{theorem}
\setcounter{rem}{0}
\begin{rem}
If $p=2$, the we have
\begin{equation*}
H^1(U) \subset \subset L^q(U)
\end{equation*}
for each $1\leq q < p^*$.
\end{rem}
\begin{rem}
Since $p^* > p$ the result is valid for $q=p$. This actually holds for all $1\leq p \leq \infty$, even when $p > n$.
\end{rem}
\begin{rem}
We also have 
\begin{equation*}
W^{1,p}_0(U) \subset \subset L^p(U)
\end{equation*}
even when the assumption that $\partial U$ is $C^1$ is lifted.
\end{rem}


%%%%%%%%%%%%%%%%%%%%%%%%%%%%%%%%%%%%%%%%%%%%%%%%%%%%%%%%%%%%%%
%			PROPERTIES OF THE APPROXIMATIONS:
%%%%%%%%%%%%%%%%%%%%%%%%%%%%%%%%%%%%%%%%%%%%%%%%%%%%%%%%%%%%%%
\newpage
\section{Properties of the approximate nonlinearities}
\subsection{Properties of $\Phi$ and its approximations}
\label{app:Phi}
Let's discuss some of the properties of the sequence of approximations to the nonlinearity $\Phi$. To reiterate we defined
\begin{equation*}
\Phi_n(u) = (\Phi * \varphi_{\frac{1}{n}})(u) - (\Phi * \varphi_{\frac{1}{n}})(0) + \frac{u}{n}.
\end{equation*}
It is obvious that $\Phi_n$ is infintely differentiable for each $n$, as well as $\Phi_n'(u) \geq \frac{1}{n}$. In addition the construction is so that $\Phi_n(0) = 0$.

The first property we'll consider is summarised in the following proposition.
\begin{proposition}
\label{prop:Phi_uniform_convergence}
If $\Phi$ is continuous, $\Phi_n$ converges to $\Phi$ uniformly on compact sets.
\end{proposition}
\begin{proof}
This follows almost immediately from part iii) of theorem 7 of appendix C in \citep{evans}, which states that if $\Phi$ is continuous, then $\Phi * \varphi_{\frac{1}{n}}$ converges uniformly to $\Phi$ on compact sets. So let $K\subset \mathbb{R}$ be a compact set, and fix $\epsilon < 0$. Then there is an $N_1 \in \mathbb{N}$ so that
\begin{equation*}
\underset{x\in K}{\mathrm{sup}}|\Phi * \varphi_{\frac{1}{n}}(x) - \Phi(x)| < \frac{\epsilon}{3},
\end{equation*} 
whenever $n \geq N_1$. Similarly we have an $N_2 \in \mathbb{N}$, so that 
\begin{equation*}
|\Phi * \varphi_{\frac{1}{n}}(0) - \Phi(0)| < \frac{\epsilon}{3},
\end{equation*}
for every $n \geq N_2$. Lastly, since $K$ is compact, there is an $N_3 \in \mathbb{N}$ so that
\begin{equation*}
\underset{x\in K}{\mathrm{sup}}\frac{|x|}{n} < \frac{\epsilon}{3},
\end{equation*}
whenever $n \geq N_3$.

Now we set $N = \mathrm{max}\{N_1,N_2,N_3\}$, and see that
\begin{align*}
\underset{x\in K}{\mathrm{sup}}|\Phi_n(x) - \Phi(x)| &\leq  \underset{x\in K}{\mathrm{sup}}|\Phi * \varphi_{\frac{1}{n}}(x) - \Phi(x)| + |\Phi * \varphi_{\frac{1}{n}}(0) - \Phi(0)| + \underset{x\in K}{\mathrm{sup}}\frac{|x|}{n} \\
&< \frac{\epsilon}{3} + \frac{\epsilon}{3} + \frac{\epsilon}{3} \\
&= \epsilon,
\end{align*}
for every $n \geq N$, which proves the proposition.
\end{proof}

Another useful fact about our approximations is the following:

\begin{proposition}
\label{prop:Psi_uniform_convergence}
$\Psi_n$ converges uniformly to $\Psi$ on compact sets.
\end{proposition}
\begin{proof}
Let $K$ be a compact set in $\mathbb{R}$, and take $R$ to be so that $|u| \leq R$ for all $u \in K$. Fix $\epsilon > 0$, and see that for any $u\in K$
\begin{align*}
|\Psi_n(u) - \Psi(u)| &\leq \left|\int_0^u \Phi_n(s) - \Phi(s)ds \right| \\
&\leq \int_0^u |\Phi_n(s) - \Phi(s)|ds.
\end{align*}
By proposition \ref{prop:Phi_uniform_convergence} we can take find an $N \in \mathbb{N}$ so that
\begin{equation*}
|\Phi_n(s) - \Phi(s)| < \frac{\epsilon}{R}
\end{equation*}
for all $n \geq N$ and $s \in [-R, R]$. Thus,
\begin{align*}
|\Psi_n(u) - \Psi(u)| &\leq \int_0^u |\Phi_n(s) - \Phi(s)|ds \\
&< \frac{\epsilon}{R}|u| \leq \epsilon.
\end{align*}
With $\epsilon$ being arbitrary, this proves the assertion.
\end{proof}

In the remainder of this section we'll assume that $\Phi$ is continuous and strictly increasing, and that $\Phi( \pm \infty) = \pm \infty$. Then it is obvious that both $\Phi$ is both one-to-one and onto, and so has an inverse $\Phi^{-1}$.

\begin{proposition}
$\Phi^{-1}$ is strictly increasing, and continuous.
\end{proposition}
\begin{proof}
Let's first prove that it's strictly increasing, which is easy. Take $y_1 < y_2$, and let $x_1$ and $x_2$ be so that $\Phi(x_i) = y_i$ for $i=1,2$. Then, because $\Phi$ is strictly increasing, we have $x_1 < x_2$. Which is the same as saying that
\begin{equation*}
y_1 < y_2 \quad \Rightarrow \quad \Phi^{-1}(y_1) < \Phi^{-1}(y_2),
\end{equation*}
i.e. $\Phi^{-1}$ is stricly increasing.

Moving along, let $y \in \mathbb{R}$ and $x$ be so that $\Phi(x) = y$. Take any $\epsilon > 0$, and consider the interval $(x-\epsilon, x + \epsilon)$. We have that $(\Phi^{-1}(x-\epsilon), \Phi^{-1}(x+\epsilon))$ is non-empty because $\Phi^{-1}$ is strictly increasing, and $y$ is an element. Take now $\delta = \mathrm{min}\{ |y-\Phi^{-1}(x\pm\epsilon)|\} >0$, then for any $\tilde{y} \in (y-\delta, y + \delta)$ we have
\begin{equation*}
\tilde{y} > y-\delta \Rightarrow \Phi^{-1}(\tilde{y}) > \Phi^{-1}(y-\delta) \geq x-\epsilon
\end{equation*}
and
\begin{equation*}
\tilde{y} < y+\delta \Rightarrow \Phi^{-1}(\tilde{y}) < \Phi^{-1}(y+\delta) \leq x+\epsilon,
\end{equation*}
by using that $\Phi^{-1}$ is strictly increasing, as well as $\Phi(x-\epsilon) \leq y-\delta$ and $\Phi(x+\epsilon) \geq y + \delta$. This shows that $\Phi^{-1}$ is continuous.
\end{proof}

This result is also valid for $\Phi_n$ for every $n \in \mathbb{N}$, beacause they are all continuous and stricly increasing. We then have the following helpful result.

\begin{proposition}
$\Phi_n^{-1}$ converges pointwise to $\Phi^{-1}$.
\end{proposition}

\begin{proof}
We need to consider $|\Phi_n^{-1}(y) - \Phi^{-1}(y)| = |x_n - x|$. Define $y_n = \Phi(x_n)$, then if we can show that $y_n \to y$ we can use the coninuity of $\Phi^{-1}$  to show that $x_n = \Phi^{-1}(y_n) \to \Phi^{-1}(y) = x$, and we would be done. It is now useful to see that
\begin{align*}
|y-y_n| &= |\Phi(x) - \Phi(x_n)| \\
	&= |\Phi_n(x_n) - \Phi(x_n)|
\end{align*}
and hence we are in a nice position to use that $\Phi_n$ converges uniformly on compact sets, if we can show that $x_n$ is a bounded sequence. To that end we let $x_1<x$ be so that
\begin{equation*}
\Phi(x_1) = \Phi(x) -1
\end{equation*}
and similarly $x_2 > x$ by
\begin{equation*}
\Phi(x_2) = \Phi(x) +1.
\end{equation*}
Since $\Phi_n$ converges uniformly on $[x_1, x_2]$, there is an $N \in \mathbb{N}$ so that
\begin{equation*}
|\Phi_n(x) - \Phi(x)| < 1
\end{equation*} 
for every $x \in [x_1, x_2]$ and $n \geq N$. So take any $n \geq N$ and see that
\begin{equation*}
\Phi_n(x_2) \geq \Phi(x_2) - 1 = \Phi(x),
\end{equation*}
which by the strict monotonicity of $\Phi_n$ means that $x_n \leq x_2$. A similar argument yields $x_n \geq x_1$, and so we have that for every $n \geq N$ that $x_n \in [x_1, x_2]$. This means that $x_n$ is a bounded sequence, and hence we can use the unform convergence of $\Phi_n$ on compact sets to conclude that $y_n$ converges to $y$. Finally we have that
\begin{equation*}
x_n = \Phi_n^{-1} (y) = \Phi^{-1}(y_n) \overset{y_n\to y}{\to} \Phi^{-1}(y) = x
\end{equation*}
\end{proof}

We can go even a step further with the last result of this section. It will require a bit more involved argument, but will operate on some familiar ideas from our latest proof.

\begin{proposition}
\label{prop:inverse_uniform}
$\Phi_n^{-1}$ converges to $\Phi^{-1}$ uniformly on compact sets.
\end{proposition}

\begin{proof}
Let $K \subset \mathbb{R}$ be a compact set, and fix $\epsilon > 0$. Our aim will be to establish that there is $N \in \mathbb{N}$ so that 
\begin{equation*}
|x-x_n| = |\Phi^{-1}(y) - \Phi_n^{-1}(y)| < \epsilon
\end{equation*}
for every $y \in K$ and $n \geq N$. 

Define $\tilde{K} = \Phi^{-1}(K)$ which is compact since $\Phi^{-1}$ is continuous. We will want to consider
\begin{equation*}
\Phi(x+\epsilon) - \Phi(x) > 0
\end{equation*}
for $x \in \tilde{K}$, so let $L$ be a Lipschitz constant for $\Phi$ over $\tilde{K}+[0,\epsilon]$. Our first claim is that
\begin{equation*}
\mathrm{inf}_{x \in \tilde{K}} \Phi(x+\epsilon) - \Phi(x) > 0.
\end{equation*}
It is obvious that the infimum is nonnegative, so we need to show that it is not zero. Assume to the contrary that for every $n \in \mathbb{N}$ there is an $x_n$ so that
\begin{equation*}
0 < \Phi(x_n + \epsilon) - \Phi(x_n) < \frac{1}{n}
\end{equation*}
which yields a sequence $(x_n)_n$ in $\tilde{K}$. Since $\tilde{K}$ is compact we have a subsequence $(x_{n_j})_j$ that converges to some $ \tilde{x} \in \tilde{K}$. But then
\begin{align*}
|\Phi(\tilde{x} + \epsilon) - \Phi(\tilde{x})| & \leq |\Phi(\tilde{x}+\epsilon) - \Phi(x_{n_j} + \epsilon)| + |\Phi(x_{n_j}+\epsilon) - \Phi(x_{n_j})| \\
&+ |\Phi(x_{n_j}) - \Phi(\tilde{x})| \\
&\leq 2L|\tilde{x}-x_{n_j}| + \frac{1}{n_j},
\end{align*}
Which can be made arbitrarily small. Hence, $\Phi(\tilde{x}+\epsilon) - \Phi(\tilde{x}) = 0$, which is in stark contrast to $\Phi$ being strictly increasing. This proves our claim and we can with some confidence say that there is a $\delta_+>0$ so that $\Phi(x+\epsilon) - \Phi(x) \geq \delta_+$ for every $x \in \tilde{K}$.

We can do quite a similar argument on
\begin{equation*}
\Phi(x) - \Phi(x-\epsilon) > 0,
\end{equation*}
to establish a $\delta_- > 0$ so that $\Phi(x) - \Phi(x-\epsilon) \geq \delta_-$ for every $x\in \tilde{K}$.

Let now $\delta = \mathrm{min}\{\delta_+, \delta_- \}>0$, and there is by the uniform convergence of $\Phi_n$ an $N \in \mathbb{N}$ so that
\begin{equation*}
|\Phi_n(x) - \Phi(x)| < \delta,
\end{equation*}
for every $x \in \tilde{K}+[\epsilon, \epsilon]$ and $n \geq N$. Take now any $y \in K$ and let $x$ and $x_n$ be so that
\begin{equation*}
\Phi(x) = \Phi_n(x_n) = y.
\end{equation*}
See now that
\begin{align*}
\Phi_n(x+\epsilon) &> \Phi(x+\epsilon) -\delta \\
&\geq    \Phi(x+\epsilon) -\delta_+ \\
&\geq \Phi(x+\epsilon) - (\Phi(x+\epsilon) - \Phi(x)) \\
&= \Phi(x).
\end{align*}
So we have $\Phi_n(x+\epsilon) < y$, which implies that $x_n < x+\epsilon$ since $\Phi_n$ is strictly increasing. Quite similarly we can establish that
\begin{align*}
\Phi_n(x-\epsilon) &< \Phi(x-\epsilon) + \delta \\
&\leq \Phi(x-\epsilon) + \delta_- \\
&\leq \Phi(x-\epsilon) + (\Phi(x) - \Phi(x-\epsilon)) \\
&= \Phi(x),
\end{align*}
which implies that $x_n > x-\epsilon$. In toto we have that whenever $n\geq N$
\begin{equation*}
|\Phi^{-1}(y) - \Phi_n^{-1}(y)| = |x - x_n| < \epsilon
\end{equation*}
for every $y \in K$, which is exactly what we wanted to prove.
\end{proof}

Looking back at this proof, the only things we needed were that $\Phi$ is strictly increasing and continuous, and is locally Lipschitz, and that $\Phi_n$ converges to $\Phi$ uniformly on compact sets. So we can generalise our result in the following lemma.

\begin{lemma}
\label{lem:Inverse_uniform_general}
Let $f: \mathbb{R} \to \mathbb{R}$ be a strictly monotone, continuous function, and let $\{f_n\}$ be a family of strictly monotone, continuous functions that converge to $f$ uniformly on compact sets. Then $f_n^{-1}$ converges to $f^{-1}$ uniformly on compact sets.
\end{lemma}

\begin{proof}
The proof is identical to that of proposition \ref{prop:inverse_uniform} with only minor alterations in notation.
\end{proof}

\subsection{Properties of $Z$ and its approximations}
In this section we assume $\Phi$ to be absolutely continuous, i.e.
\begin{equation*}
\Phi(u) = \int_0^u \Phi'(s)ds,
\end{equation*}
and $\Phi' \in L^1_{loc}(\mathbb{R})$. We also assume $\Phi$ to be strictly increasing and  $\Phi(\pm \infty) = \pm \infty$.

We define 
\begin{equation}
Z(u) = \int_0^u \mathrm{min}\{1, \Phi'(s)\} ds,
\end{equation}
and similarly
\begin{equation}
Z_n(u) = \int_0^u\mathrm{min}\{1, \Phi_n'(s)\}ds.
\end{equation}

As in the previous section we'll establish that $Z_n$ converges to $Z$ uniformly on compact sets, and continue on with the inverses and a similar convergence results for them. We'll start off with the following proposition:

\begin{proposition}
$Z_n$ converges pointwise to $Z$.
\end{proposition}
\begin{proof}
A useful identity to have in mind at this point is
\begin{equation*}
\mathrm{min}\{a,b\} = \frac{a+b}{2} - \frac{|a-b|}{2},
\end{equation*}
which makes
\begin{align*}
|\mathrm{min}\{1,a\} - \mathrm{min}\{1,b\}| &= \left| \frac{1+a}{2} -\frac{1+b}{2} + \frac{|1-b|}{2} - \frac{|1-a|}{2}\right| \\
&\leq \frac{1}{2}|a-b| + \frac{1}{2}\big| |1-b| - |1-a| \big| \\
&\leq \frac{1}{2}|a-b| + \frac{1}{2}|a-b| \\
&= |a-b|.
\end{align*}

For any $u \in \mathbb{R}$ we then have
\begin{align*}
|Z_n(u) - Z(u)| &\leq \int_0^{|u|} |\mathrm{min}\{1,\Phi'_n(s)\} - \mathrm{min}\{1,\Phi'(s)\}|ds \\
&\leq \int_0^{|u|} |\Phi'_n(s)-\Phi'(s)|ds,
\end{align*}
by using the above derived inequality. Following the definition of $\Phi_n$, its derivative is
\begin{equation*}
\Phi'_n(s) = \Phi' * \varphi_{\frac{1}{n}}(s) + \frac{1}{n},
\end{equation*}
and so
\begin{equation*}
|Z_n(u) - Z(u)| \leq \int_0^{|u|}|\Phi'*\varphi_{\frac{1}{n}}(s) - \Phi'(s)|ds + \frac{|u|}{n}.
\end{equation*}
We have of course complete control of the latter of these terms by making $n$ sufficiently large. For the first term we use part iv) of theorem 7 of appendix C in \citep{evans}, to state that $\Phi'*\varphi_{\frac{1}{n}}$ converges to $\Phi$ in $L^1_{loc}(\mathbb{R})$. With this we deduce that $Z_n$ converges pointwise to $Z$.
\end{proof}

Next we will prove that $Z_n$ converges to $Z$ uniformly on compact sets, but for this we need the Arzela-Ascoli theorem, presented below for ease of reference, (cf. e.g. \citep[p 718]{evans}). This theorem hinges on the concept of equicontinuity, for which a definition is presented directly below.
\begin{mydef}
A sequence $(f_n)$ of continuous, real-valued functions defined on an interval $[a,b]$ is said to be \textbf{equicontinuous} if for every $\epsilon >0$ there is a $\delta >0 $, depending only on $\epsilon$, so that for $x,y \in [a,b]$
\begin{equation*}
|x-y| < \delta \Rightarrow |f_n(x) - f_n(y)| < \epsilon
\end{equation*}
for every $n \in \mathbb{N}$.
\end{mydef}
We are now prepared for the Arzela-Ascoli theorem.
\begin{theorem}[Arzela-Ascoli]
\label{thm:Ascoli}
A bounded, equicontinuous sequence, $(f_n)$, of continuous, real-valued functions has a subsequence which converges uniformly.
\end{theorem}

Using this powerful theorem we can prove the following:
\begin{proposition}
\label{prop:Z_uniform_convergence}
$Z_n$ converges to $Z$ uniformly on compact sets.
\end{proposition}
\begin{proof}
The proof will be done by contradiction. Take $K \subset \mathbb{R}$ to be a compact set, and consider a closed interval of the form $[-M,M]$, with $M$ so large as to contain $K$. Obviously, $Z_n$ is a bounded sequence, with
\begin{equation*}
|Z_n(x)| \leq M.
\end{equation*}
In addition, the sequence is equicontinuous, since all $Z_n$'s are globally Lipschitz continuous with the same Lipschitz constant $1$.

We can from this conclude that $(Z_n)$ has a subsequence that converges uniformly on $[-M,M]$, and that the limit is $Z$, using theorem \ref{thm:Ascoli} and that $Z_n$ converges pointwise to $Z$. We can, with and identitical argument say that any subsequence of $(Z_n)$ also contains a subsequence converging uniformly to $Z$.

Suppose now that $(Z_n)$ does not converge to $Z$ uniformly on $[-M,M]$. Then there is an $\epsilon > 0$ so that for every $k \in \mathbb{N}$ there is an $n \geq k$ so that
\begin{equation*}
\underset{x\in[-M,M]}{\mathrm{sup}}|Z_n(x) - Z(x)| \geq \epsilon.
\end{equation*}
From this we can construct a subsequence $(Z_{n_k})$ so that
\begin{equation*}
\underset{x\in[-M,M]}{\mathrm{sup}}|Z_{n_k}(x) - Z(x)| \geq \epsilon,
\end{equation*}
for every $k$, but this is in clear contradiction with the fact that $(Z_{n_k})$ must have a subsequence converging uniformly to $Z$. Hence we can conclude that $(Z_n)$ converges uniformly to $Z$ on $[-M,M]$, and as a consequence also on $K$.
\end{proof}

We'll continue on with the last result of this section.
\begin{proposition}
\label{prop:Zinv_uniform_convergence}
With $\Phi$ strictly increasing and continuos, $Z$ is strictly increasing and continuous, and thus permits a strictly increasing and continuous inverse. Furthermore $Z_n^{-1}$ converges to $Z^{-1}$ uniformly on compact sets.
\end{proposition}
\begin{proof}
 We note that $Z_n' \geq \frac{1}{n}$, and is thus stricly increasing, permitting a continuous inverse $Z_n^{-1}$. 

That $Z$ is also strictly increasing follows from that $\Phi$ is strictly increasing. Indeed, suppose $t<u$ and see that
\begin{equation*}
Z(u) - Z(t) = \int_t^u \mathrm{min}\{1,\Phi'(s)\}ds \geq 0.
\end{equation*}
The only way this can be $0$ is if $\Phi' = 0$ a.e. everywhere on $(t,u)$, but if that were the case then $\Phi(u) = \Phi(t)$, in contradiction with $\Phi$ being strictly increasing. So $Z$ is strictly increasing, and has a continuous and strictly increasing inverse $Z^{-1}$.

We have from proposition \ref{prop:Z_uniform_convergence} that $Z_n$ converges uniformly to $Z$ on compact sets, and so we can simply use lemma \ref{lem:Inverse_uniform_general} to conclude that $Z_n^{-1}$ converges uniformly to $Z^{-1}$ on compact sets.
\end{proof}

\end{appendix}
\end{document}